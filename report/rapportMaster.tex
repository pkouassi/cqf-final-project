\documentclass[a4paper]{book}
\setlength{\textwidth}{16cm}
\usepackage[french]{babel}
\usepackage[french]{minitoc}
\usepackage[cyr]{aeguill}
\usepackage{fancyhdr}
\pagestyle{fancy}
\usepackage{url}
\usepackage{amssymb,amstext,amsmath}
\usepackage[dvips]{graphicx}
\usepackage{color}
\usepackage{theorem}
\usepackage{textcomp}
\usepackage{braket}
\usepackage{stmaryrd}
\usepackage{eurosym}
\usepackage{pst-text}
\usepackage[absolute]{textpos}
\usepackage[T1]{fontenc}
\markright{16cm}
\fancyhead{}
\fancyfoot{}
\lhead{\resizebox{1in}{!}{\includegraphics{logo_upmc}} \\ \vfill \resizebox{1in}{!}{\includegraphics{casa_logo}}}
\rhead{Amina BOUMEDIENE}
\rfoot{\thepage}

\title{
	\begin{figure}[t!]
		\begin{center}
			\begin{tabular}{lr}
				\includegraphics[scale=0.30]{logo_upmc} & \hfill \includegraphics[scale=0.85]{casa_logo}\\
				\hline
			\end{tabular}
		\end{center}
	\end{figure}
	\fbox{ 
		\begin{minipage}{\textwidth}
			\begin{center}
					\begin{tabular}{c}
						\\
					{\Large \bf Applications Math\'ematiques pour la Finance de march\'e }\\
					\\
					\end{tabular} 
			\end{center}
		\end{minipage} 
	}
}
\author{
		\begin{tabular}{c}
			{\Large \bf Amina BOUMEDIENE}\\
			\\
			\textit{ Master Math\'ematiques et Applications, sp\'ecialit\'e Statistique}\\
			\hline
		\end{tabular}
} 
\date{
	\vfill
	\begin{flushleft}
			Stage effectu\'e du 2 avril au 28 septembre 2007 au sein de la Gestion Financi\`ere, 
service Investissement et Couverture  de Cr\'edit Agricole S.A. et encadr\'e par M. Karim Mzoughi (ing\'enieur financier)\\
	\end{flushleft}
	\vfill
	\begin{flushleft}
		Laboratoire de Statistiques Th\'eoriques et Appliqu\'ees \\
		Universit\'e Pierre et Marie Curie, Paris VI \\
		\vfill
		\begin{tabular}{ll}
			Responsable de formation : &Paul Deheuvels (professeur)\\
			Encadrants : &Delphine Blanke (ma\^{\i}tre de conf\'erences)\\
						&Giovanni Peccati (ma\^{\i}tre de conf\'erences)
		\end{tabular}
	\end{flushleft}
}

\textwidth=16cm
\textheight=21cm

\oddsidemargin = -0.5cm
\evensidemargin=-0.5cm

\setcounter{tocdepth}{2}
\setcounter{secnumdepth}{3}

\theorembodyfont{\itshape}
\theoremstyle{break}
\newtheorem{df}{D\'efinition}[chapter]
{\theorembodyfont{\rmfamily}\theoremstyle{break} \newtheorem{ex}{Exemple}[chapter]}
{\theorembodyfont{\rmfamily}\theoremstyle{break}\newtheorem{rem}{Remarque}[chapter]}
\newtheorem{theo}{Th\'eor\`eme}[chapter]
\newtheorem{prop}{Proposition}[chapter]
{\theorembodyfont{\rmfamily}\theoremstyle{break}
\newtheorem{dem}{D\'emonstration}[chapter]}
\newtheorem{hyp}{Hypoth\`ese}[chapter]
\newtheorem{Not}{Notation}[chapter]

\newcommand{\E}{\mathbb{E}} % Op�rateur Esp�rance
\newcommand{\V}{\mathbb{V}} % Op�rateur Variance
\newcommand{\C}{\mathbb{C}ov} %Op�rateur Covariance
\renewcommand{\L}{\mathbb{L}}
\renewcommand{\P}{\mathbb{P}}
\newcommand{\R}{\mathbb{R}}
\newcommand{\Q}{\mathbb{Q}}
\newcommand{\noi}{\noindent} 
\newcommand{\F}{\mathcal{F}}

\newcommand{\N}{\mathcal{N}}
\newcommand{\X}{\underline{X}}
\newcommand{\dif}{\mathrm{d}}
\renewcommand{\thesubsubsection}{\alph{subsubsection})}
\newcommand{\1}{\text{1\hspace{-.55ex}\mbox{l}}} %fonction indicatrice
\begin{document}	
\maketitle

%***********************	
\chapter*{Remerciements}
%***********************
	\paragraph*{}
	Je tiens \`a remercier mon ma\^{\i}tre de stage Karim Mzoughi pour avoir su me guider tout au long de ce stage aussi bien dans les approches 
math\'ematique et informatique que dans le cadre de la Finance de march\'e. Je le remercie \'egalement pour m'avoir fait confiance, 
pour sa grande disponibilit\'e, sa patience et ses conseils dans la r\'edaction de ce rapport. 
	\paragraph*{}
	Merci \'egalement \`a Virginie Krakowiak pour m'avoir d\'ecrit les instruments financiers et les strat\'egies sous-jacentes.
	\paragraph*{} 
	Merci aussi \`a  Yves Amardeil, Philippe Gaspard, Patrick Mavro et Eric Morvan 
	pour m'avoir appris comment obtenir et utiliser des donn\'ees diponibles sur le march\'e. 
	\paragraph*{}
	Merci \`a toute l'\'equipe de la Direction Financi\`ere pour son accueil.
	\paragraph*{}
	Je remercie enfin tous les enseignants du Master Math\'ematiques et Applications, sp\'ecialit\'e Statistique, 
pour la richesse de leur formation.
	

%**********************
\chapter*{Introduction}
%**********************
\section*{}
	Cr\'e\'e en 2001, Cr\'edit Agricole SA est l'organe central du groupe Cr\'edit Agricole.  Il garantit l'unit\'e financi\`ere du groupe et 
coordonne, en liaison avec ses filiales sp\'ecialis\'ees, les strat\'egies des diff\'erents m\'etiers du groupe. \\
\indent Le service Investissement et couverture de Cr\'edit Agricole S.A. est en particulier charg\'e de de la gestion 
du risque de taux d'int\'er\^et global.\\
\\
\indent	Ce rapport comporte deux volets quasi-ind\'ependants correspondant aux deux missions dont j'ai \'et\'e charg\'ee :
	\begin{itemize}
		\item[$\bullet \quad$]	mettre en place le mod\`ele de d\'eformation de la courbe des taux de Brace, Gatarek 
\& Musiela  
		\item[$\bullet  \quad$]	valoriser des produits structur\'es sur le p\'etrole 
	\end{itemize}	
Il se voudra surtout th\'eorique et laissera peu de place \`a la 
partie informatique. Cependant, le d\'eveloppement en C++ a \'ete tr\`es important au cours de ce stage tant au niveau des 
difficult\'es d'impl\'ementation rencontr\'ees qu'au temps consacr\'e.

\section*{Mod\`ele de Brace, Gatarek et Musiela}
	Le secteur de la Finance de march\'e  en pleine expansion, il donne naissance \`a des produits d\'eriv\'es de plus 
en plus complexes et diversifi\'es. C'est en particulier le cas pour les produits d\'eriv\'es sur taux d'int\'er\^et. \\
\indent Les flux sous--jacents \`a ces derniers sont g\'en\'eralement des fonctions non lin\'eaires des futurs taux 
interbancaires (IBOR), c'est pourquoi il est devenu indispensable de disposer d'un mod\`ele simulant les taux 
d'int\'er\^ets \`a diff\'erentes dates dans le futur. \\
\\  \indent
	Apr\`es avoir fait un  tour d'horizon des mod\`eles disponibles dans la litt\'erature sur le sujet, 
nous avons rapidement opt\'e pour le mod\`ele de march\'e de Brace, Gatarek et Musiela. 
Le mod\`ele BGM a l'avantage de donner les m\^emes formules de 
prix que les op\'erationnels du march\'e pour les caps et les swaptions, produits liquides sur le march\'e des taux 
d'int\'er\^et.
\\ \indent
	Une premi\`ere partie de ce rapport sera consacr\'ee \`a des rappels sur les taux d'int\'er\^et.\\ 
En se pla\c cant dans le cadre unidimensionnel, nous exprimerons dans la deuxi\`eme partie, la dynamique des taux IBOR sous diff\'erentes 
mesures de probabilit\'e. \\
Nous proposerons une calibration d\'etaill\'ee, bas\'ee sur les cotations des caps \`a la monnaie apr\`es avoir 
d\'etermin\'e les volatilit\'es implicites des caplets sous-jacents.\\
Nous simulerons enfin les trajectoires des taux IBOR et \'etudierons un exemple de produit pouvant \^etre valoris\'e par 
notre mod\`ele.
\newpage
\section*{Valorisation de produits structur\'es sur le p\'etrole}

	Lorsqu'on valorise des produits sur le p\'etrole, une des premi\`eres difficult\'es est de prendre en compte la 
coexistance de deux risques :  \\
	\begin{itemize}
		\item[$\bullet \quad$] le risque li\'e aux fluctuations du prix du p\'etrole exprim\'e en \$ US \\
		\item[$\bullet \quad$] le risque li\'e aux fluctuations du taux de change \$ / \officialeuro \\
	\end{itemize}
A partir des diffusions historiques, nous exprimerons les dynamiques de ces deux al\'eas sous la probabilit\'e risque-neutre 
domestique en tenant compte des param\`etres du march\'e \'etranger. Nous \'etendrons alors la formule de Black \& Scholes aux
options soumises \`a un risque de change, dites options  \guillemotleft \ quanto \guillemotright . 
Nous \'etudierons enfin plusieurs exemples de structures optionnelles et verrons comment il est possible d'appliquer nos formules. 

\tableofcontents

%*****************************************************************
\part{Produits d\'eriv\'es sur taux d'int\'er\^et \\ Mod\`ele BGM }
%*****************************************************************
%************************************
\chapter*{Abr\'eviations et notations}
%************************************

	\begin{description}
		\item [I. Notations G\'en\'erales] 
		\item
		\item [$\quad t$] Date courante
		\item [$\quad \tau( s,t )$] Mesure du temps, fraction d'ann\'ee entre la date s et la date t compte tenu 
			du choix de la base.   
		\item [$\quad \mathrm N$] Nominal 
		\item [$\quad \theta  > 0 $] Une dur\'ee (en ann\'ee)
		\item [$\quad \beta( t )$] Facteur d'accumulation cf. d\'efinition \ref{facteur} p. \pageref{facteur} 
		\item [$\quad \text{DF}( t, T )$] Facteur d'actualisation stochastique cf. d\'efinition \ref{factActSto} p.
			\pageref{factActSto}
		\item [$\quad B( t,T )$] Prix en t d'un z\'ero-coupon de maturit\'e T cf. d\'efinition \ref{ZC} p. \pageref{ZC}
		\item [$\quad L( t, T )$] Taux de composition simple cf. d\'efinition \ref{TauxSimple} p. \pageref{TauxSimple}
		\item [$\quad Y( t, T )$] Taux compos\'e annuellement cf. d\'efinition \ref{TauxCompose} p. \pageref{TauxCompose}
		\item [$\quad F( t, T, S) $] Taux forward cf. d\'efinition \ref{TauxForward} p. \pageref{TauxForward}
		\item [$\quad f( t,T )$] Taux forward instantan\'e cf. d\'efinition \ref{TauxForwardInstantane} 
			p. \pageref{TauxForwardInstantane}
		\item [$\quad \left \{ T_1 , \cdots , T_n \right \}$] Ech\'eancier
		\item [$\quad i$] Indice parcourant l'\'ech\'eancier
		\item [$\quad r( t, \theta )$] Taux forward instantan\'e de maturit\'e $\theta$ cf. d\'efinition 
			\ref{TauxForwardInstantane} p. \pageref{TauxForwardInstantane}
		\item [$\quad L( t,0 ; \delta ) $] Taux IBOR spot en t de maturit\'e $\delta$
		\item [$\quad L( t,\theta ; \delta ) $]Taux IBOR forward en t (maturit\'e $\delta$) d'\'ech\'eance $t+\theta$
		\item [$\quad  Cap( t,T_n, K )$] Le prix en $t$ d'un cap de strike $K$ et d'\'ech\'eance $T_n$
		\item [$\quad Caplet( t, T_j, K )$] le prix en $t$ d'un caplet de strike $K$ et d'\'ech\'eance $T_j$ 
		\item [$\quad \N( 0, 1 )$]  La loi Normale centr\'ee et r\'eduite
		\item [$\quad \left< X \right> = \left< X, X \right > $] Le crochet d'un processus $X$ 
			cf. Annexe \ref{CalculSto} p. \pageref{CalculSto}
		\item [$\quad \Delta_t$] Pas de discr\'etisation
		\item [$\quad \left \{ t_k = t + k \Delta_t , k \right \} $ ] Maillage discr\'etisant l'EDS
		\item [$\quad k$] Indice parcourant le maillage 
		\item 
		\item [II. D\'efinition des processus]
		\item
		\item [$\quad \{B( t,T ), \quad t \geq 0\}$] Processus du z\'ero-coupon cf. d\'efinition 
			\ref{dif zc} p. \pageref{dif zc}
		\item [$\quad \{f( t,T ), \quad t \: \in \: \lbrack 0, \,T \rbrack \}$] Processus du taux forward 
			instantan\'e cf. \'equation \ref{dif fw} p. \pageref{dif fw}
		\item[$\quad \{L( t,\theta; \delta ), \quad t \geq 0\}$] Processus du taux IBOR cf. proposition \ref{dif IBOR} 
			p. \pageref{dif IBOR} 
		\item
	\end{description}
	\newpage
	\begin{description}
		\item [III. D\'efinition des fonctions]
		\item
		\item [$\quad \N( x ) = \int_{-\infty}^x \frac 1{\sqrt{2\pi}} e^{- \frac {u^2} 2 \dif u }$] La fonction de 
			r\'epartition de la loi Normale centr\'ee et r\'eduite  
		\item [$\quad \phi( x ) = \frac 1 {\sqrt{2\pi}} e^{- \frac {x^2} 2}$] La fonction de densit\'e 
			de la loi Normale centr\'ee et r\'eduite
		\item [$\quad \sigma( t,T )$] La fonction de volatilit\'e du processus $\{B( t,T ), \quad t \geq 0\}$ 
		\item [$\quad \sigma_f( t,T )$] La fonction de volatilit\'e du processus $\{f( t,T ), \quad t \: \in \: 
			 \lbrack 0, \,T \rbrack \}$
		\item [$\quad  \gamma( t,\theta; \delta ) $] La fonction de volatilit\'e du processus 
			$\{L( t, \theta; \delta ), \quad t \geq 0\}$
		\item [$\quad g_{K,\sigma}( x ) =  x \N \left \lbrack \frac{1}{\sigma}\log( x/K )+\frac{1}{2}
						\sigma \right \rbrack
 					- K \N \left \lbrack \frac{1}{\sigma}\log( x/K )-\frac{1}{2}\sigma \right \rbrack $]
			La fonction de Black \& Scholes
		\item 			
	\end{description}
	
%***************************************************
\chapter{Pr\'eliminaires sur les Taux d'Int\'er\^et}
%***************************************************

	\begin{df}[Taux Court] $\quad$ 
		On d\'efinit le taux court $r(\cdot)$ comme le taux associ\'e \`a un placement sans risque 
instantan\'e. On parle de taux spot instantan\'e.
	\end{df}
	\begin{df}[Facteur d'Accumulation] $\quad$ \label{facteur}
		Il s'agit d'un investissement dont l'\'evolution est r\'egie par l'\'equation diff\'erentielle
ordinaire : 
		\begin{displaymath}
			\left \{
			\begin{array}{l}
				\dif \beta(t)=r(t) \beta (t) \dif t \\
				\beta(0) = 1
			\end{array}
			\right.
		\end{displaymath} 
		dont la solution est 
		\begin{displaymath}
			\beta(t) = \exp{\left \{ \int_0 ^ t r(s) \dif s \right \} }
		\end{displaymath}
	\end{df}
	\begin{rem}
		Autrement dit, pour tout instant petit $\Delta_t$ , 
		\begin{displaymath}
			 \frac {\beta(t+\Delta_t) - \beta(t)}{\beta(t)} = r(t) \Delta_t
		\end{displaymath}
	\end{rem}
	
	\begin{df}[Facteur d'actualisation stochastique] $\quad$ \label{factActSto}
		Dans le cas o\`u $ \{ r(t), t\geq 0 \} $ est un processus stochastique, la quantit\'e DF,
 facteur d'actualisation, d\'efinie par : 
		\begin{displaymath}
			\text{DF}(t,T):=\frac {\beta (t)} {\beta (T)} 
		\end{displaymath}
		est {\bf al\'eatoire}. Elle repr\'esente l'\'equivalent en $t$ d'une unit\'e de devise en $T$.
	\end{df}
	\begin{df}[Obligation Z\'ero-Coupon]$\quad$ \label{ZC}
		L'obligation Z\'ero-coupon de maturit\'e $T$ est un contrat qui garantit le paiement d'une unit\'e de devise 
\`a la date $T$, sans versement interm\'ediaire. \\
Le prix de ce contrat est appel\'e {\bf Z\'ero-coupon}, on le note :
		\begin{displaymath}
			B(t,T)
		\end{displaymath}
	\end{df}	
	\begin{rem}[Taux d\'eterministes]$\quad$
		Dans un mod\`ele o\`u les taux d'int\'er\^et sont d\'eterministes, on a : $ B(t,T)\: = \: \text{DF}(t,T)$
	\end{rem}
	\begin{rem}$\quad$
		Le z\'ero-coupon est un outil essentiel \`a la valorisation de produits d\'eriv\'es sur taux d'int\'er\^et.\\ 
	En effet, leurs flux s'expriment souvent en fonction de z\'ero-coupons.
	\end{rem}
	\begin{df}[Taux de composition simple]$\quad$\label{TauxSimple}
		Le taux spot de composition simple de dur\'ee de composition $\tau(t,T)$ est le taux constant auquel il 
faut investir pour disposer d'une unit\'e de devise en $T$ partant d'un nominal $B(t,T)$. \\
On le note $L(t,T)$ tel que :
		\begin{displaymath}
			L(t,T) := \frac{1 - B(t,T)}{\tau(t,T) B(t,T)}
		\end{displaymath}
Autrement \'ecrit :
		\begin{displaymath}
			B(t,T) = \frac 1 {1 + \tau(t,T)L(t,T)}
		\end{displaymath}
	\end{df}

	\begin{rem}[Justification de la notation] $\quad$
		Ce taux est not\'e $L$ car les taux IBOR sont contruits selon cette m\'ethodologie.
		\footnote{La base de calcul utilis\'ee est $\frac{\text{Exact}}{360}$}
	\end{rem}
	\begin{rem} $\quad$
		Ce taux est utilis\'e pour des dur\'ee de composition inf\'erieures \`a un an. Dans les autres cas,
on utilise le taux suivant :
	\end{rem}
	\begin{df}[Taux compos\'e annuellement]$\quad$\label{TauxCompose}
		Il s'agit du taux auquel il faut investir pour obtenir une unit\'e de devise en $T$ partant de $B(t,T)$.\\ 
On le note $Y(t,T)$ tel que :
		\begin{displaymath}
			Y(t,T) := \frac 1 { \left [ B(t,T) \right ] ^{\frac 1 {\tau(t,T)}}}
		\end{displaymath}
	\end{df}
	\begin{rem}[Equivalence]$\quad$
		Ces deux d\'efinitions sont \'equivalentes quand on fait tendre $T \rightarrow t^+$. On retrouve alors le taux 
spot instantan\'e :
		\begin{eqnarray*}
			r(t)&=& \lim_{T\mapsto t^+}L(t,T)\\
			&=& \lim_{T\mapsto t^+}Y(t,T)\\
		\end{eqnarray*}
	\end{rem}
	\begin{df}[Courbe des taux]$\quad$
		La courbe des taux z\'ero-coupon, ou courbe des taux actuariels, est le graphe de la fonction :
		\begin{displaymath}
			T \mapsto \left\{
				\begin{array}{l}
				L(t,T) \qquad \forall \: t<T \leq t+1  \\
				Y(t,T) \qquad \forall \: T>t+1
				\end{array}
				\right.
		\end{displaymath}
Il s'agit de la courbe \`a un instant t du taux actuariel pour toutes les maturit\'es espac\'ees d'un an. 
	\end{df}

	\begin{rem}[Construction]$\quad$
		Les banques construisent une courbe z\'ero-coupon \`a partir de diff\'erents instruments financiers 
tels que les taux courts et les taux de swap.
	\end{rem}
	\begin{df}[Forward Rate Agreement] $\quad$
		Le Forward Rate agrement (FRA) est une op\'eration de garantie de taux \`a terme conclue
 entre deux contreparties qui s'engagent \`a \'echanger \`a une date future connue $S$, la diff\'erence 
entre un taux variable de r\'ef\'erence $L(T,S)$ constat\'e sur le march\'e \`a la date $T$, et un taux fixe $K$ garanti \`a 
l'origine t. \\
La valeur du contrat pour un nominal $N$ est :
		\begin{displaymath}
			N \, \tau(T,S) \, (K-L(T,S))
		\end{displaymath} 
	\end{df}
	\begin{rem}$\quad$
		Le contrat FRA est sans paiement de prime au d\'epart.
	\end{rem}
	\begin{df}[Taux Forward]$\quad$ \label{TauxForward}
		Le taux forward $F(t,T,S)$ est le taux qui rend nulle la valeur du contrat $FRA(t,T,S)$.\\
		 Il est donn\'e par :
		\begin{displaymath}
			F(t,T,S) := \frac 1 {\tau(T,S)} \left ( \frac{B(t,T)}{B(t,S)} - 1 \right )
		\end{displaymath}
	\end{df}
	\begin{rem}$\quad$
		Ce taux forward $F(t,T,S)$ peut \^etre vu comme une estimation du futur taux spot $L(T,S)$
	\end{rem}
	\begin{df}[Taux Forward instantan\'e]$\quad$ \label{TauxForwardInstantane}
		Le taux forward instantan\'e $f(t,T)$ correspond au taux forward pour une dur\'ee de composition\\ 
infinit\'esimale .
		\begin{eqnarray*}
			f(t,T) := lim_{S\mapsto T^+} F(t,T,S)
		\end{eqnarray*} \\
$\quad$ On notera $r(t,\theta)$, le taux forward instantan\'e en $t$, de maturit\'e $t+\theta$ 
	\end{df}
		
	\begin{prop}\label{zc-fw}$\quad$
		\begin{equation*}
			\fbox{$
				\begin{array}{rcl}
					B(t,T) &=&\exp{\left (  -\int_t^T f(t,u) \dif u \right )}\\
						\\
						 &=&\exp{\left (-\int^{T-t}_0 r(t,u) \dif u \right )} 
				\end{array}
			$}
		\end{equation*}
	\end{prop}		      
	\begin{dem}$\quad$
		
		\begin{itemize}
			\item[$\bullet$] Nous avons :
				\begin{eqnarray*}
					f(t,T) &:= & lim_{S\mapsto T^+} F(t,T,S) \\
					&=& - lim_{S\mapsto T^+} \frac 1 {B(t,S)} \frac {B(t,S) - B(t,T)}{S-T}\\
					&=& -  \frac 1 {B(t,T)} \frac {\partial B(t,T)}{\partial T}\\
					&=& - \frac {\partial \ln{B(t,T)}}{\partial T}
				\end{eqnarray*}
			\item[$\bullet$] La seconde partie s'obtient en effectuant un changement de variable et en remarquant
 que :
				\begin{eqnarray*}
					r(t,T-t) &=& f(t,T)
				\end{eqnarray*}
		\end{itemize}
		\begin{flushright}
			$\boxslash$
		\end{flushright}
	\end{dem}
\newpage
	\begin{rem}$\quad$
		D'apr\`es la proposition \ref{zc-fw} p. \pageref{zc-fw} , on voit que mod\'eliser le comportement du taux forward instantan\'e revient 
\`a mod\'eliser celui du z\'ero-coupon.		
	\end{rem}

%********************************************************************************************************
\chapter{Le mod\`ele de Brace, Gatarek \& Musiela \\ un cas particulier du mod\`ele de Heath, Jarrow \& Merton }
%********************************************************************************************************
\section{Hypoth\`eses du mod\`ele HJM}

\subsection*{Diffusion du Z\'ero-coupon}
	On suppose que le processus du z\'ero-coupon $\left \{ B(t,T) , \, t \geq 0 \right \} $ suit une diffusion lognormale :
	\begin{equation} \label{dif zc}
	\boxed{
		\dif B(t,T) = B(t,T) \left ( r(t,0) \dif t - \sigma(t,T) \dif W_t  \right )}
	\end{equation} 
	\begin{rem}[Sym\'etrie du mouvement brownien]$\quad$
		Si $W_t$ est un mouvement brownien, alors $-W_t$ est aussi un mouvement brownien.\\ 
Le signe $-$ apparaissant dans l'\'equation (\ref{dif zc}) p. \pageref{dif zc}  n'a qu'une valeur technique   		
	\end{rem}

\subsection*{Diffusion du processus des taux forward instantan\'e $\lbrace f(t,T),\, t \in [0,T] \rbrace$}

	\begin{prop}[Diffusion du taux forward instantan\'e]$\quad$\label{dif fw}
		En posant : 
		\begin{eqnarray*}
			 \sigma_f(t,T) &=& \frac {\partial \sigma(t,T)}{\partial T} \\
		\end{eqnarray*}
On a la diffusion suivante pour le taux forward instantan\'e :
		\begin{equation*}
			\boxed{\dif f(t,T)=\sigma_f(t,T)\left( \int_t^T \sigma_f(t,u) \dif u\right) \dif t
				+ \sigma_f(t,T) \dif W_t \\}
		\end{equation*}
	\end{prop}
\newpage
	\begin{dem}
		Nous avons: 
		\begin{displaymath}
			f(t,T)=-\frac{\partial \ln B(t,T)}{\partial T}|_T
		\end{displaymath}
Ainsi:
		\begin{eqnarray*}
			\dif f(t,T) &=& \dif \left (-\frac{\partial \ln B(t,T)}{\partial T}|_T \right)\\
			   	 &=& - \dif \left ( \frac{\partial \ln B(t,T)}{\partial T}|_T \right ) \\
			   	 &=& -\frac{\partial \dif \ln B(t,T)}{\partial T}|_T \\
		\end{eqnarray*}
Appliquons la formule de It\^o \footnote{cf propositien \ref{LemmeIto} p. \pageref{LemmeIto} dans l'Annexe} 
au processus $\left \{ Y_t=g(B(t,T)) , t \geq 0 \right \} $ avec $g(x)=\ln(x)$ \\
On obtient :
		\begin{eqnarray*}
			Y_t &=& g(B(t,T))=g(B(0,T))+\int_{0}^{t}{\frac{1}{B(s,T)}dB(s,T)}+
			\frac{1}{2}\int_{0}^{t}{\frac{-1}{B(s,T)^2}}\dif \left \langle B( \cdot , \, T) \right \rangle _s
		\end{eqnarray*}
On a d'une part,
		\begin{eqnarray*}
			\dif B(t,T) &=&  B(t,T) \left ( r(t,0) \dif t - \sigma(t,T) \dif W_t  \right )\\
		\end{eqnarray*}
et d'autre part,
		\begin{eqnarray*}
			\left \langle B( \cdot , \, T) \right \rangle _s &=& \int_{0}^{s}{\sigma^2(u,T)B(u,T)^2}du
		\end{eqnarray*} 
Ainsi :
		\begin{eqnarray*}
			Y_t &=& Y_0+\int_0^t{\left ( r(s,0) \dif s-\sigma(s,T) \dif W_s \right )}
			-\frac{1}{2}\int_0^t{\sigma^2(s,T)}\dif s \\
		\end{eqnarray*}
Autrement \'ecrit:
		\begin{eqnarray*}
			\dif Y_t &=& \left ( r(t,0)-\frac{1}{2}\sigma^2(t,T) \right ) \dif t-\sigma(t,T) \dif W_t
		\end{eqnarray*}
Nous obtenons la diffusion suivante pour  $\left \{f(t,T);t \in [0,T] \right \}$
		\begin{eqnarray*}
			\dif f(t,T) &=& \sigma(t,T) \partial_T \sigma(t,T) \dif t + \partial_T \sigma(t,T) \dif W_t \\
		\end{eqnarray*}
Le r\'esultat suit imm\'ediatement en posant : $\sigma_f(t,T)=\partial_T \sigma(t,T)$  
		\begin{flushright}
			$\boxslash$
		\end{flushright}
	\end{dem}

\newpage	

\section{Vers le mod\`ele BGM}
	
	\begin{df}[Taux IBOR] $\quad$
	Les taux d'int\'er\^et IBOR (Interbank Offered Rate)  sont des indices usuels du march\'e mon\'etaire.
Ils servent de r\'ef\'erence pour le prix de l'argent emprunt\'e pendant des dur\'ees inf\'erieures \`a un an.\\
Ils sont fix\'es
		\begin{itemize}
			\item	soit \`a Franckfort pour l'euro, on les appelle alors {\bf EURIBOR}
			\item	soit \`a Londres pour le dollar US, la livre sterling, le franc suisse ... 
				on les appelle {\bf LIBOR} 
		\end{itemize}
	\end{df}
	\begin{Not}$\quad$
		On note {\bf \  $L(t, \theta;  \delta)$ \ } le taux IBOR forward en t pour une composition entre 
$\, t+\theta \, $ et $\ t+x+\delta\,$ ; 
il s'agit du taux auquel les pr\^ets et emprunts seront effectu\'es entre $t+\theta$ et $t+\theta+\delta$ . \newline
\indent $\delta$ repr\'esente la dur\'ee de composition des taux entre les deux dates.\newline
Par exemple $ \delta = 3$ mois pour l'EURIBOR 3 mois, $\delta = 6$ mois pour l'EURIBOR 6 mois ...   
	\end{Not}
	\begin{prop}[Diffusion du taux IBOR] $\quad$ \label{dif IBOR}
		Le processus du taux IBOR d\'efini par :
		\begin{eqnarray}\label{IBOR zc}
			1 + \delta L(t,\theta;  \delta) & := & \exp{ \left ( \int_\theta^{\theta+\delta} r(t,u) \dif u \right )}\\
			\nonumber
		\end{eqnarray}
suit la diffusion suivante :
		\begin{equation*}
			\fbox{
			$\begin{array}{lrc}
				\dif L(t,\theta;  \delta)&=& \left (
						\frac{\partial}{\partial \theta} L(t,\theta;  \delta) + L(t,\theta;  \delta)\gamma(t,\theta;  \delta)
						\sigma(t,t+\theta) + \frac{\delta L^2(t,\theta;  \delta)}{1+\delta L(t,\theta;  \delta)}
						\gamma^2(t,\theta;  \delta)  \right )\dif t 
						+ L(t,\theta;  \delta) \gamma(t, \theta;  \delta) \dif W_t \\
			\end{array}$
			}
		\end{equation*}
	\end{prop}
	\begin{rem}$\quad$
		La d\'emonstration de ce r\'esultat s'appuiera fortement sur la diffusion du taux forward instantan\'e \`a 
\'ech\'eance glissante que nous allons pr\'ealablement pr\'eciser.
	\end{rem}
	\begin{prop}[Diffusion de $\left \{ r(t,\theta), t \geq 0 \right\}$] $\quad$ \label{dif fw r}
		La diffusion du taux forward \`a \'ech\'eance glissante est donn\'ee par :
		\begin{equation*}
			\fbox{
			$\begin{array}{lcr}
				\dif r(t,\theta) &=& \frac{\partial}{\partial \theta} \left ( r(t,\theta) + \frac 1 2 \left (
					\int_t^T \sigma_f(t,u) \dif u \right )^2 \right )
					+ \sigma_f(t, t+\theta) \dif W_t  
			\end{array}$}
		\end{equation*}	
	\end{prop}
	\begin{dem}[Diffusion de $\left \{ r(t,\theta), t \geq 0 \right\}$] $\quad$
		En reprenant le r\'esulat de  la proposition \ref{dif fw} p. \pageref{dif fw} nous avons :
 		\begin{eqnarray*}
			\dif r(t,T-t) &=& \dif f(t,T)+\frac{\partial}{\partial T}f(t,T) \dif t\\
				      &=& \sigma_f(t,T) \left(\int_t^T \sigma_f(t,u) \dif u \right) \dif t 
					+\sigma_f(t,T) \dif W_t + \partial_T f(t,T) \dif t
		\end{eqnarray*}
En posant $\theta=T-t$, nous obtenons :
		\begin{eqnarray*}
			\dif r(t,\theta) &=& \frac{\partial}{\partial \theta}\left(r(t,\theta)+\frac{1}{2}
					\left(\int_t^{t+\theta} \sigma_f(t,u) \dif u\right)^2\right) \dif t 
					+\sigma_f(t,t+\theta) \dif W_t \\
				&&\text{\tiny{en appliquant la formule de d\'erivation des fonctions compos\'ees 
				\`a l'expression $\frac{\partial}{\partial x}\left [ \left(\int_t^{t+x} \sigma_f(t,u) \dif u 
					\right )^2 \right ]$}}\\
		\end{eqnarray*}
Cette expression nous donne donc la diffusion du taux forward avec \'ech\'eance glissante sous la probabilit\'e risque neutre.
		\begin{flushright}
			$\boxslash$
		\end{flushright}
	\end{dem}
	\begin{dem}[Diffusion du taux IBOR]$\quad$
		Pour d\'efinir le processus de volatilit\'e $\sigma_f(t,t+\theta)$, nous fixons $\delta>0$  \, 
(par exemple $\delta=0.25$ ce qui correspond \`a des taux trimestriels) et on suppose que pour tout $x\geq 0$ 
le processus du taux -IBOR $\{L(t,\theta;  \delta);t\geq 0\}$  d\'efini par
		\begin{equation} \label{IBOR fw inst}
			1+\delta L(t,\theta;  \delta)=\exp\,(\int_\theta^{\theta+\delta} r(t,u)\,du)
		\end{equation}
\newline
\noi
a une structure de volatilit\'e lognormale i.e.
		\begin{equation} \label{hyp dif IBOR}
			\boxed{\dif L(t,\theta;  \delta)=\cdots dt+L(t,\theta;  \delta) 
				\gamma(t,\theta;  \delta) \dif W_t \, } 
		\end{equation}
\noi o\`u la fonction  $\gamma : \R_+^2\mapsto\R^d$ est d\'eterministe\\
		\begin{eqnarray*}
			\dif \left\lbrace\int_\theta^{\theta+\delta}{r(t,u) \dif u}\right\rbrace &=& 
			\int_\theta^{\theta+\delta}{dr(t,u) \dif u}\\
			&=& \int_\theta^{\theta+\delta}\left\lbrace \frac{\partial}{\partial u}\left((r(t,u)+\frac{1}{2}
				\left(\int_t^{t+u} \sigma_f(t,v)\dif v\right)^2\right) \dif t
				+ \sigma_f(t,t+u) \dif W_t\right\rbrace \dif u \\	
			&& \text{\tiny{ en utilisant la proposition \ref{dif fw r} p. \pageref{dif fw r}}}	\\		 	
			&=& \left\lbrack 
					\int_\theta^{\theta+\delta} \frac{\partial}{\partial u}
					\left \lbrace 
						r(t,u) + \frac{1}{2}
							\left(
								\int_t^{t+u}\sigma_f(t,v)\dif v
							\right)^2
					\right\rbrace 
					\dif u 
				\right\rbrack \dif t +
				\left\lbrack 
					\int_\theta^{\theta+\delta} \sigma_f(t,t+u)\dif u 
				\right\rbrack \dif W_t \\
			&=& \left \lbrack 
				r(t,u) + \frac{1}{2} 
					\left ( 
						\int_t^{t+u} \sigma_f(t,v) \dif v 
					\right ) ^2 
				\right \rbrack 
				_{u=\theta}^{u=\theta +\delta} \dif t \\
			&&+ \left \lbrack \int_0^{\theta+\delta} \sigma_f(t,t+u) \dif u - \int_0^{\theta} \sigma_f(t,t+u) \dif u 
				\right \rbrack \dif W_t\\
			&=& \left[r(t,\theta+\delta)-r(t,\theta)+ \frac{1}{2} \left( \int_t^{t+\theta+\delta} \sigma_f(t,s) \dif s\right)^2 
				-\frac{1}{2} \left( \int_t^{t+\theta} \sigma_f(t,s) \dif s\right)^2\right]\dif t\\
			&& +\left[\int_0^{\theta+\delta}\sigma_f(t,t+u)\dif u-\int_0^{\theta}\sigma_f(t,t+u)\dif u \right] \dif W_t \\
			&=& \left[r(t,\theta+\delta)-r(t,\theta)+\frac{1}{2} \left( \sigma(t,t+\theta+\delta)^2-\sigma(t,t+\theta)^2 \right)
				\right] \dif t \\
			&& + \left( \sigma(t,t+\theta+\delta)-\sigma(t,t+\theta)\right) \dif W_t	\\
		\end{eqnarray*}
\noi On obtient alors : 
		\begin{eqnarray*}
			\dif L(t,\theta;  \delta) &=& \dif \left[\frac{\exp\{\int_\theta^{\theta+\delta} r(t,u)\dif u \}-1}{\delta} \right]\\
			&=& \frac{1}{\delta} \exp \left\{ \int_\theta^{\theta+\delta} r(t,u) \dif u\right\} 
				\dif \left( \int_\theta^{\theta+\delta} r(t,u) \dif u \right)\\
			&& +\frac{1}{2\delta} \exp \left\{ \int_\theta^{\theta+\delta}r(t,u)\dif u \right\} 
				\dif \left< \int_\theta^{\theta+\delta} r(t,u) \dif u \right> \\
		\end{eqnarray*}
\noi En utilisant la diffusion de $\left \{ r(t,\theta), t \geq 0 \right\}$ donn\'ee par la proposition \ref{dif fw r} p. \pageref{dif fw r}, 
et en remarquant que :
		\begin{eqnarray*}
			\frac{\partial}{\partial \theta} L(t,\theta ; \delta) &=& \frac{\partial}{\partial \theta}
					\left[\frac{\exp\{\int_\theta^{\theta+\delta} r(t,u) \dif u \}-1}{\delta} \right] \\
			&=& \frac{1}{\delta}[1+\delta L(t,\theta ; \delta )][r(t,\theta+\delta)-r(t,\theta)] \\
		\end{eqnarray*}
\noi Nous obtenons l'\'equation suivante
		\begin{eqnarray*}
			\dif L(t,\theta;  \delta) &=& \left(\frac{\partial}{\partial \theta}L(t,\theta;  \delta) 
						+ \delta^{-1}(1+\delta L(t,\theta;  \delta))
				\sigma(t,t+\theta+\delta) (\sigma(t,t+\theta+\delta)-\sigma(t,t+\theta))\right) \dif t \\
			&& + \delta^{-1}(1+\delta L(t,\theta;  \delta))(\sigma(t,t+\theta+\delta)-\sigma(t,t+\theta)) \dif W_t \\
		\end{eqnarray*}
\noi Cette derni\`ere \'equation  et l'expression (\ref{hyp dif IBOR}) p. \pageref{hyp dif IBOR} impliquent que pour tout $\theta\geq 0$
		\begin{eqnarray}\label{rec vol zc}
			\sigma(t,t+ \theta+\delta)-\sigma(t,t+\theta)&=&\frac{\delta L(t,\theta;  \delta)}
				{1+\delta L(t,\theta;  \delta)}\gamma(t,\theta;  \delta)
		\end{eqnarray}
\noi On obtient finalement :
		\begin{eqnarray*}
			\dif L(t,\theta;  \delta)&=& \left (
						\frac{\partial}{\partial \theta} L(t,\theta;  \delta) + L(t,\theta;  \delta)
						\gamma(t,\theta;  \delta)
						\sigma(t,t+\theta) + \frac{\delta L^2(t,\theta;  \delta)}{1+\delta L(t,\theta;  \delta)}
						\gamma^2(t,\theta;  \delta)  \right )\dif t 
						+ L(t,\theta;  \delta) \gamma(t, \theta;  \delta) \dif W_t \\
		\end{eqnarray*}
		\begin{flushright}
			$\boxslash$
		\end{flushright}
	\end{dem}
\bigskip
	\fbox{
		\begin{minipage}{15cm} 
			
			\begin{rem}$\quad$ \label{IBORFwZc}
				Le taux IBOR forward s'exprime en fonction des z\'ero-coupons spots de la mani\`ere suivante :\\
					\begin{eqnarray*}
						L(t, \theta ; \delta ) &=& \frac 1 {\delta}
						\left ( \frac {B(t, t+\theta) - B(t, t + \theta + \delta)}
						{B(t, t + \theta + \delta)}  \right ) \label{IBORFw-Spot} 
					\end{eqnarray*}
			\end{rem}
			\begin{rem}[Position du processus]$\quad$ \label{PositionIBOR}
				Etant donn\'ee la position en $t$ du processus du taux IBOR 
$L(t,\theta ; \delta)$, en posant $x := t' - t $, dur\'ee \'ecoul\'ee entre les dates $t$ et $t'$, sa position en $t'$ est : 
$L(t', \theta - x ; \delta)$
			\end{rem}
		\end{minipage}
	}	


%***********************************************************************
\chapter{Mod\`ele BGM : calibration par rapport aux caps}
%***********************************************************************



\section{Valorisation des caps sur le march\'e}

\subsection{Les caps : des produits liquides sur le march\'e des taux}


\subsubsection*{Ech\'eancier}
	On se donne un ensemble de dates possibles pour le taux forward : 
$\left \{  T_1, T_2, \cdot \cdot \cdot, T_n\right \}$ \newline
\indent	On pose :
	\begin{itemize}
		\item[$\bullet \quad$]	$\delta := T_i - T_{i-1}  \quad$ en supposant que la dur\'ee \'ecoul\'ee entre 
			deux dates cons\'ecutives est constante. \\ 
		\item[$\bullet \quad$]	$T_0 := t \quad$ c'est \`a dire la date d'aujourd'hui. \\
		\item[$\bullet \quad$]	$T_i := t + i \, \delta \, , \quad i \in \llbracket  1 , n \rrbracket \quad$
	\end{itemize}
	\begin{df}[Cap-Caplets]$\quad$ 
		Un cap peut \^etre vu comme un portefeuille de calls dont le sous-jacent est le taux IBOR. 
En effet, chaque versement n'est effectu\'e en $T_i$ que si le flux $(L(T_{i-1}, 0;  \delta) -K) $ est positif. \\
Typiquement, pour un nominal $N$, le contrat verse \`a chaque date $T_i , \ i \in \llbracket 1,n \rrbracket $, le flux : \\
$ N \delta (L(T_{i-1}, 0;  \delta) -K)_+ $, c'est \`a dire le diff\'erentiel entre :\\
		\begin{itemize}
			\item 	le taux IBOR entre $T_{i-1}$ et $T_{i-1} + \delta = T_i $ fix\'e en $T_{i-1}$ \\
			\item un taux fixe $K$ \'etabli \`a l'avance \\
		\end{itemize}
La quantit\'e $\delta$ correspond \`a la dur\'ee de composition des taux. \\
\indent	 Un cap peut \^etre d\'ecompos\'e en la somme de n contrats versant le flux d\'ecrit pr\'ec\'edemment.\\ 
Chacun de ces contrats, appel\'e {\bf Caplet}, verse le flux pr\'ecedent.  
	\end{df}
	\begin{rem}[Cadre de travail] $\quad$
		Notre \'etude se limitera \`a consid\'erer des caps dont le sous-jacent est le taux IBOR 6 mois, ie :
$\delta = 0.5 $ \\
	Les \'ech\'eances des caps seront des multiples d'une ann\'ee (1 an , 2 ans, $\cdots$ , 10 ans), ainsi l'indice 
intervenant dans la date \'ech\'eance du cap sera pair.\\
	\end{rem}
	\unitlength = 1 cm
	\begin{pspicture}(20,10)(0,0)
		\rput(2,7){\bf Mise en place des notations :}
		\psline{-}(0,5)(6,5)
		\psline[linestyle=dashed](6,5)(8,5)
		\psline{->}(8,5)(15,5)
		\psline{-}(0,4.75)(0,5.25)
		\rput(0,5.5){\text{\tiny{$T_0 = t$}}}
		\psline{-}(2,4.75)(2,5.25)
		\rput(2,5.5){\text{\tiny{$T_1 = 6$ mois}}}		
		\psline{-}(4,4.75)(4,5.25)
		\rput(4,5.5){\text{\tiny{$T_2 = 1$ an}}}
		\psline{<->}(0,4.70)(2,4.70)
		\rput(1,4.80){\text{\tiny{$\delta=0.5$}}}	
		\psline{->}(2,5)(2,3)
		\rput(2,2.75){\text{\tiny{$\delta(L(T_0,0;\delta)-K)_+$}}}
		\psline{<-}(0,5.65)(0,6)
		\psline(0,6)(2,6)
		\psline{->}(2,6)(2,5.65)
		\rput(1,6.2){\text{\tiny{$\sigma_{T_0}$}}}
		\psline{->}(4,5)(4,4)
		\rput(4,3.75){\text{\tiny{$\delta(L(T_1,0;\delta)-K)_+$}}}
		\psline{<->}(2,4.7)(4,4.7)
		\rput(3,4.8){\text{\tiny{$\delta=0.5$}}}
		\psline{<-}(2,5.65)(2,6)
		\psline(2,6)(4,6)
		\psline{->}(4,6)(4,5.65)
		\rput(3,6.2){\text{\tiny{$\sigma_{T_1}$}}}
		\psline(9,4.75)(9,5.25)
		\rput(9,5.5){\text{\tiny{$T_{2i-1}$}}}
		\psline(11,4.75)(11,5.25)
		\rput(11,5.5){\text{\tiny{$T_{2i} = i$ ans}}}
		\psline(13,4.75)(13,5.25)
		\rput(13,5.5){\text{\tiny{$T_{2i+1}$}}}
		\psline{->}(13,5)(13,4)
		\rput(13,3.75){\text{\tiny{$\delta(L(T_{2i},0;\delta)-K)_+$}}}
		\psline{->}(11,5)(11,3)
		\rput(11,2.75){\text{\tiny{$\delta(L(T_{2i-1},0;\delta)-K)_+$}}}
		\psline{->}(9,5)(9,2)
		\rput(9,1.75){\text{\tiny{$\delta(L(T_{2i-2},0;\delta)-K)_+$}}}	
		\psline{<->}(9,4.7)(11,4.7)
		\rput(10,4.8){\text{\tiny{$\delta=0.5$}}}
		\psline{<->}(11,4.7)(13,4.7)
		\rput(12,4.8){\text{\tiny{$\delta=0.5$}}}
		\psline{<-}(9,5.65)(9,6)
		\psline(9,6)(11,6)
		\psline{->}(11,6)(11,5.65)
		\rput(10,6.2){\text{\tiny{$\sigma_{T_{2i-1}}$}}}
		\psline{<-}(11,5.65)(11,6)
		\psline(11,6)(13,6)
		\psline{->}(13,6)(13,5.65)
		\rput(12,6.2){\text{\tiny{$\sigma_{T_{2i}}$}}}
	\end{pspicture}
	\begin{rem}[Nominal]$\quad$
		Valoriser un cap bas\'e sur un nominal $N$ est \'equivalent \`a valoriser un cap sur une unit\'e 
de devise puis \`a multiplier ce prix par $N$. Ainsi, pour all\'eger les notations, nous consid\'ererons des caps sur 
une unit\'e de devise. 
	\end{rem}


\subsection{Pratiques du march\'e - Cotation en volatilit\'e}

	Le prix d'un cap est obtenu en sommant les prix des n caplets le constituant. En fait, le premier caplet \'etant 
enti\`erement d\'eterministe, il s'agit plut\^ot d'une somme \`a n-1 termes. Chacun des caplets est valoris\'e par la  
formule de Black :
	\begin{eqnarray}
		Caplet^{Market}(t, T_i, K) &=& 
		B(t, T_i) g_{K, \sigma_{T_{i-1}}\sqrt{T_{i-1}-t}} \Big ( L(t, (i-1) \delta ; \delta) \Big ) 
	\end{eqnarray}
	Sur le march\'e, seules les volatilit\'es des caps sont cot\'ees pour les diff\'erentes maturit\'es.\\
	Les volatilit\'es des caplets constituant un cap sont suppos\'ees \'egales \`a la volatilit\'e du cap cot\'ee 
sur le march\'e.
\newpage
	\unitlength = 1 cm
	\begin{pspicture}(20,10)(0,0)
		\psline{-}(0,5)(6,5)
		\psline[linestyle=dashed](6,5)(8,5)
		\psline{->}(8,5)(15,5)
		\psline{-}(0,4.75)(0,5.25)
		\rput(0,5.5){\text{\tiny{$T_0 = t$}}}
		\psline{-}(2,4.75)(2,5.25)
		\rput(2,5.5){\text{\tiny{$T_1 = 6$ mois}}}		
		\psline{-}(4,4.75)(4,5.25)
		\rput(4,5.5){\text{\tiny{$T_2 = 1$ an}}}
		\psline{<->}(0,4.70)(2,4.70)
		\rput(1,4.80){\text{\tiny{$\delta=0.5$}}}	
		\psline{->}(2,5)(2,3)
		\rput(2,2.75){\text{\tiny{$\delta(L(T_0,0;\delta)-K)_+$}}}
		\psline{<-}(0,5.65)(0,6)
		\psline(0,6)(2,6)
		\psline{->}(2,6)(2,5.65)
		\rput(1,6.2){\text{\tiny{$\sigma^{Market}_{T_n}$}}}
		\psline{->}(4,5)(4,4)
		\rput(4,3.75){\text{\tiny{$\delta(L(T_1,0;\delta)-K)_+$}}}
		\psline{<->}(2,4.7)(4,4.7)
		\rput(3,4.8){\text{\tiny{$\delta=0.5$}}}
		\psline{<-}(2,5.65)(2,6)
		\psline(2,6)(4,6)
		\psline{->}(4,6)(4,5.65)
		\rput(3,6.2){\text{\tiny{$\sigma^{Market}_{T_n}$}}}
		\psline(9,4.75)(9,5.25)
		\rput(9,5.5){\text{\tiny{$T_{n-1}$}}}
		\psline(11,4.75)(11,5.25)
		\rput(11,5.5){\text{\tiny{$T_{n} =$ \'ech\'eance du cap}}}
		\psline{->}(11,5)(11,3)
		\rput(11,2.75){\text{\tiny{$\delta(L(T_{n-1},0;\delta)-K)_+$}}}
		\psline{->}(9,5)(9,2)
		\rput(9,1.75){\text{\tiny{$\delta(L(T_{n-2},0;\delta)-K)_+$}}}	
		\psline{<->}(9,4.7)(11,4.7)
		\rput(10,4.8){\text{\tiny{$\delta=0.5$}}}
		\psline{<-}(9,5.65)(9,6)
		\psline(9,6)(11,6)
		\psline{->}(11,6)(11,5.65)
		\rput(10,6.2){\text{\tiny{$\sigma^{Market}_{T_{n}}$}}}
	\end{pspicture}
	 Ainsi, la formule du prix du march\'e pour le cap de maturit\'e $T_n$ (avec n = 2k) est :
	\begin{eqnarray}
		Cap^{Market} ( t, T_n, K) &=& 
		\sum_{i=1}^{i=n} B(t, T_j) g_{K, \sigma_{T_n} \sqrt{T{i-1}-t}} \Big (L(t, (i-1) \delta ; \delta) \Big )
	\end{eqnarray}
avec $\sigma_{T_n}$, la volatilit\'e cot\'ee sur le march\'e du cap $\frac n 2$ ans .

\subsection{Volatilit\'e implicite des caplets}

	N\'eanmoins, \`a partir des diff\'erentes volatilit\'es des caps cot\'ees sur le march\'e, on peut d\'eterminer par 
inversion de la formule de prix, la volatilit\'e implicite de chaque caplet.\\
Pour cela, nous retiendrons deux m\'ethodes :\\
	\begin{itemize}
		\item	La m\'ethode de stripping \\
		\item	L'interpolation lin\'eaire \\
	\end{itemize} 
\newpage
\subsubsection{Stripping}
	
	Il s'agit de consid\'erer les volatilit\'es des caps cot\'ees sur le march\'e  pour diff\'erentes maturit\'es : 
1 an, 2 an, 3 an, ... jusqu'\`a 10 ans et de supposer que deux caplets intervenant la m\^eme  ann\'ee ont m\^eme 
volatilit\'e. 
	\unitlength = 1 cm
	\begin{pspicture}(20,10)(0,0)
		\psline{-}(0,5)(6,5)
		\psline[linestyle=dashed](6,5)(8,5)
		\psline{->}(8,5)(15,5)
		\psline{-}(0,4.75)(0,5.25)
		\rput(0,5.5){\text{\tiny{$T_0 = t$}}}
		\psline{-}(2,4.75)(2,5.25)
		\rput(2,5.5){\text{\tiny{$T_1 = 6$ mois}}}		
		\psline{-}(4,4.75)(4,5.25)
		\rput(4,5.5){\text{\tiny{$T_2 = 1$ an}}}
		\psline{<->}(0,4.70)(2,4.70)
		\rput(1,4.80){\text{\tiny{$\delta=0.5$}}}	
		\psline{->}(2,5)(2,3)
		\rput(2,2.75){\text{\tiny{$\delta(L(T_0,0;\delta)-K)_+$}}}
		\psline{<-}(0,5.65)(0,6)
		\psline(0,6)(2,6)
		\psline{->}(2,6)(2,5.65)
		\rput(1,6.2){\text{\tiny{$\sigma_{T_0}$}}}
		\psline{->}(4,5)(4,4)
		\rput(4,3.75){\text{\tiny{$\delta(L(T_1,0;\delta)-K)_+$}}}
		\psline{<->}(2,4.7)(4,4.7)
		\rput(3,4.8){\text{\tiny{$\delta=0.5$}}}
		\psline{<-}(2,5.65)(2,6)
		\psline(2,6)(4,6)
		\psline{->}(4,6)(4,5.65)
		\rput(3,6.2){\text{\tiny{$\sigma_{T_0}$}}}
		\psline(9,4.75)(9,5.25)
		\rput(9,5.5){\text{\tiny{$T_{2i-1}$}}}
		\psline(11,4.75)(11,5.25)
		\rput(11,5.5){\text{\tiny{$T_{2i} = j$ ans}}}
		\psline(13,4.75)(13,5.25)
		\rput(13,5.5){\text{\tiny{$T_{2i+1}$}}}
		\psline{->}(13,5)(13,4)
		\rput(13,3.75){\text{\tiny{$\delta(L(T_{2i},0;\delta)-K)_+$}}}
		\psline{->}(11,5)(11,3)
		\rput(11,2.75){\text{\tiny{$\delta(L(T_{2i-1},0;\delta)-K)_+$}}}
		\psline{->}(9,5)(9,2)
		\rput(9,1.75){\text{\tiny{$\delta(L(T_{2i-2},0;\delta)-K)_+$}}}	
		\psline{<->}(9,4.7)(11,4.7)
		\rput(10,4.8){\text{\tiny{$\delta=0.5$}}}
		\psline{<->}(11,4.7)(13,4.7)
		\rput(12,4.8){\text{\tiny{$\delta=0.5$}}}
		\psline{<-}(9,5.65)(9,6)
		\psline(9,6)(11,6)
		\psline{->}(11,6)(11,5.65)
		\rput(10,6.2){\text{\tiny{$\sigma_{T_{2i-1}}$}}}
		\psline{<-}(11,5.65)(11,6)
		\psline(11,6)(13,6)
		\psline{->}(13,6)(13,5.65)
		\rput(12,6.2){\text{\tiny{$\sigma_{T_{2i-1}}$}}}
	\end{pspicture}
	Il s'agit alors de r\'esoudre les syst\`emes \'equations :
	\begin{itemize}
		\item[$\bullet \quad$] \textbf{\underline{Etape 1 :}} Initialisation
			\begin{eqnarray}
				\sigma_{T_0} & =& \sigma_{T_1} \\
				&=& \sigma^{Market}_{T_2} \\
			\end{eqnarray} 
		\item[$\bullet \quad$] 	\textbf{\underline{Etape 2 :}}
			\begin{equation}
				\left \{ 
					\begin{array}{rcl}
						Cap^{Market}(t, T_4, K) &=& 
						\sum_{i=1}^{i=4} B(t, T_i) g_{K,\sigma_{T_{i-1}} \sqrt{T_{i-1}-t}} 
						\Big (L(t, (i-1) \delta ;\delta)\Big ) \\
						\sigma_{T_2} &=& \sigma_{T_3}  \\
						&\vdots&
				\end{array}\right.
			\end{equation}
		\item[$\bullet \quad$] 	\textbf{\underline{Etape n - 2:}}
			\begin{equation}
				\left \{
					\begin{array}{rcl}
							Cap^{Market}(t, T_n, K) &=& \sum_{i=1}^{i=n} B(t, T_i) 
							g_{K,\sigma_{T_{i-1}} \sqrt{T_{i-1}-t}} 
							\Big (L(t, (i-1) \delta ; \delta) \Big ) \\
							\sigma_{T_{n-2}} &=& \sigma_{T_{n-1}} \\
					\end{array}\right.
			\end{equation}
	\end{itemize}
	o\`u $\sigma_{T_i}$ correspond \`a la volatilit\'e du caplet d'\'ech\'eance $T_{i+1}$ \\
On r\'esoud ces syst\`emes de mani\`ere it\'erative en inversant les formules de prix par un algorithme du type 
Newton-Raphson .\\
\\
\indent	Le stripping suivant a \'et\'e effectu\'e avec les caps \`a la monnaie, produits les plus liquides sur le march\'e des 
taux d'int\'er\^et.
	\begin{figure}[!!hhh]
		  	\begin{center}
			  	\rotatebox{0}{ 
		   		\includegraphics[scale=0.5]{volStripCaplet}}
			  	\caption{\label{VolStripCaplet } Volatilit\'es des caplets par stripping}
		  	\end{center}
	\end{figure}
	\newpage
	
\subsubsection{Interpolation lin\'eaire}

	Cette m\'ethode repose sur le fait que la diff\'erence entre le prix de deux caps de m\^eme strike $K$ et 
de maturit\'es cons\'ecutives $T_{2i}$ et $T_{2i+2}$ est \'egale \`a la somme des prix des caplets de maturit\'es $T_{2i+1}$ 
et $T_{2i+2}$.\footnote{par la proposition \ref{PrixCap} p. \pageref{PrixCap}} 
\newline
Autrement \'ecrit :
	\begin{eqnarray*}
		Cap^{Market}(t,T_{2i+2},K) - Cap^{Market}(t, T_{2i}, K) &=& 
		Caplet^{Market}(t, T_{2i+1},K) + Cap^{Market}(t,T_{2i+2}, K) \\
	\end{eqnarray*}  
En remarquant que le cap de maturit\'e 1 an est uniquement li\'e  \`a la volatilit\'e du caplet de maturit\'e 1 an 
\footnote{Le premier flux \'etant compl\`etement d\'eterministe} et en ajoutant une interpolation lin\'eaire, on peut 
d\'eterminer toutes les volatilit\'es des caplets \footnote{via un algorithme du type Newton-Raphson}. \\
Il s'agit de r\'esoudre r\'ecursivement le syst\`eme : \\
	\begin{equation*}
		\left	\{
		\begin{array}{rcl}
			\sigma_{T_0} &=& \sigma^{Market}_{T_2} \\
			Cap^{Market} (t, T_{2i+2}, K) - Cap^{Market} (t, T_{2i}, K) &=& 
			\delta\,  B(t, T_{2i+2}) g_{K,\sigma_{T_{2i+1}}\sqrt{T_{2i+1}-t}} \Big( K(t, T_{2i+1}) \Big) \\
			&& + \, \delta\,  B(t, T_{2i+1}) g_{K,\sigma_{T_{2i}} \sqrt{T_{2i}-t}} \Big ( K(t, T_{2i}) \Big) \\
			 \sigma_{T_{2i+1}} &=& 2 \sigma_{T_{2i}}  - \sigma_{T_{2i-1}} \\
			
		\end{array}
		\right.
	\end{equation*}
	\begin{figure}[!!hhh]
		  	\begin{center}
			  	\rotatebox{0}{ 
		   		\includegraphics[scale=0.5]{VolCapletInterpol}}
			  	\caption{\label{VolCapletInterpol } Volatilit\'es des caplets par la m\'ethode d'interpolation lin\'eaire}
		  	\end{center}
	\end{figure}
	\newpage 
	\begin{rem}[Comparaison des deux m\'ethodes] $\quad$
		Les r\'esultats obtenus par chacune des deux m\'ethodes sont tr\`es similaires.\\
On observe la m\^eme tendance : la volatilit\'e des caplets est fortement croissante sur le court terme et l\'eg\`erement 
d\'ecroissante sur le long terme.
	\end{rem}



\section{Valorisation des caps dans le mod\`ele BGM}
\subsection{Probabilit\'e Forward Neutre et Propri\'et\'e de Martingale}
	
	\begin{df}[Probabilit\'e Forward Neutre ]$\quad$
		On appelle probabilit\'e $T_i$-forward neutre et on note $\Q_{T_i}$, la mesure sous laquelle tout actif 
$S$ actualis\'e par le z\'ero-coupon de maturit\'e $T_i$ est martingale. \\
	\end{df}
	\begin{prop}[Martingale] $\quad$ \label{dif IBOR fw}
		Sous la probabilit\'e $T_{i+1}$-forward neutre, le taux IBOR de maturi\'e $T_i$ satisfait :
		\begin{eqnarray*}
			\dif L(t,i \delta;  \delta) &=& \gamma(t,i \delta;  \delta) L(t, i \delta;  \delta) \dif W_t^{T_{i+1}} (t)
		\end{eqnarray*} 
	avec $W_t^{T_{i+1}}$  un $\Q_{T_{i+1}}$-mouvement brownien.
	\end{prop}
	\begin{dem}$\quad$
		On a d\'ej\`a vu (cf. remarque \ref{IBORFwZc} p. \pageref{IBORFwZc}) que :
			\begin{eqnarray*}
					L(t, \theta ; \delta ) &=& \frac 1 {\delta}
					\left ( \frac {B(t, t+\theta) - B(t, t + \theta + \delta)}
					{B(t, t + \theta + \delta)}  \right ) \label{IBORFw-Spot}
			\end{eqnarray*}
	Ainsi, $L(t,i \delta ; \delta)$ peut \^etre vu comme un actif actualis\'e par le z\'ero-coupon de maturit\'e $T_{i+1}$ ,
c'est donc une martingale sous $\Q_{T_{i+1}}$.
Son terme de tendance \'etant alors nul sous  $\Q_{T_{i+1}}$ et la volatilit\'e \'etant invariante par changement 
de num\'eraire (cf. chapitre \ref{Numeraire} p. \pageref{Numeraire} dans l' Annexe), on obtient bien :
	\begin{eqnarray*}
			\dif L(t,i \delta;  \delta) &=& 
			\gamma(t,i \delta;  \delta) L(t, i \delta;  \delta) \dif W_t^{T_{i+1}} (t) \\
		\end{eqnarray*} 
	\begin{flushright}
			$\boxslash$
		\end{flushright}
	\end{dem}
	\begin{rem}[Important]$\quad$
		Sous la probabilit\'e  $T_{i+1}$-forward neutre et en posant : \\
		\begin{center}
			$\Sigma^2_i := \frac 1 {T_i-t} \int_t^{T_i} \gamma^2(s, T_i - s ; \delta) \dif s $
		\end{center}
\indent	Le logarithme du rapport  $\frac {L(T_i, 0, \delta)}{L(t, i \delta, \delta)} $ suit une loi normale \\
		\begin{itemize}
			\item	de moyenne $\quad -\frac 1 2 \Sigma^2_i (T_i - t)$ \\
			\item	de variance $\quad \Sigma_i^2 (T_i-t)$ .\\
		\end{itemize}
On peut \'ecrire : 
		\begin{eqnarray}
			L(T_i, 0 ; \delta) &=& L(t, i \delta; \delta) \exp \left \{ 
			-\frac 1 2 \Sigma^2_j (T_i-t)	+ \Sigma_i \sqrt{T_i-t} \, U		
						 \right \} \\
			&&\text{\tiny{cf. remarque \ref{PositionIBOR} page \pageref{PositionIBOR}
			pour la position du processus}} \nonumber
		\end{eqnarray}
avec $U \stackrel{\mathcal L}{\sim} \N(0,1) $
	\end{rem}
	\begin{prop}[Mouvements Browniens Forwards]$\quad$\label{mbf}
		Nous avons la relation suivante entre les diff\'erents mouvements browniens forwards :
		\begin{eqnarray*}
			\dif W_t^{T_{i+1}} &=& \dif W_t^{T_{i}} + \frac{\delta L(t,i \delta;  \delta)}{1+\delta L(t, i \delta;  \delta)}
						\gamma(t,i \delta;  \delta) \dif t \\
		\end{eqnarray*}
	\end{prop}
	\begin{dem}[Mouvements Browniens Forwards]\label{mbf} $\quad$
		On \'ecrit le quotient de Radon-Nikodym associ\'e \`a ce changement de probabilit\'e :\\
		\begin{eqnarray*}
			Z &=& \frac {\dif \Q_{T_i}}{\dif \Q_{T_{i+1}}} \\
			&=& \frac {B(t, T_i)} {B(t, T_{i+1})}  \frac{B(T_i,T_{i+1})} {B(T_i, T_i)} 
		\end{eqnarray*}
	Par le Th\'eor\`eme de Girsanov \footnote{cf. proposition \ref{TheoremGirsanov} p. \pageref{TheoremGirsanov} 
	dans l'Annexe}, on sait que $Z$ est une martingale exponentielle sous $\Q_{T_{i+1}}$, 
ce qui peut s'\'ecrire :
		\begin{eqnarray*}
			\frac {\dif Z}{Z} &=& \psi \dif W_t^{T_{i+1}} \\
		\end{eqnarray*}
\indent	avec  
		\begin{eqnarray*}
			\dif W_t^{T_{i+1}} &=& \dif W_t^{T_i} + \psi \dif t \\
		\end{eqnarray*} 
		
On a :
	\begin{eqnarray*}
		\frac{\dif Z}{Z} &=& \frac {\dif \left ( \frac {B(t,T_i)}{B(t,T_{i+1})} \right )}
					{\frac{B(t,T_i)}{B(t,T_{i+1})}} \\
			&=& \frac {\dif \left ( 1+\delta L(t,i \delta ; \delta) \right )}{1+\delta L(t,i \delta ; \delta)}\\
	\end{eqnarray*}	
	En substituant la diffusion du taux IBOR forward donn\'ee \`a la 
				proposition \ref{dif IBOR fw} p. \pageref{dif IBOR fw}, on obtient :
	\begin{eqnarray*}	
			\frac{\dif Z}{Z}&=& \frac{\delta \gamma(t, i \delta ; \delta) L(t, i \delta ; \delta)}
				{1+\delta L(t,i \delta ; \delta)} \dif W_t^{T_{i+1}} \\
	\end{eqnarray*}
Par identification, on trouve :
	\begin{eqnarray*}
		\psi &=& \frac{\delta\,  \gamma(t, i \delta ; \delta) L(t, i \delta ; \delta)}
				{1+\delta L(t,i \delta ; \delta)} \\
	\end{eqnarray*}
Et, par suite :
	\begin{eqnarray*}
		\dif W_t^{T_{i+1}} &=& \dif W_t^{T_i} \, + \, \frac{\delta \, \gamma(t, i \delta ; \delta) 
					L(t, i \delta ; \delta)} {1 + \delta L(t,i \delta ; \delta)} \dif t \\
	\end{eqnarray*}
	\begin{flushright}
			$\boxslash$
		\end{flushright}
	\end{dem}
\fbox{
	\begin{minipage}{\textwidth}
{\bf {\underline {Valorisation par AOA : }}}

	On rappelle le principe de valorisation en absence d'opportunit\'e d'arbitrage : \\
	\begin{prop}[Pricing des produits d\'eriv\'es]\label{pricing} $\quad$
		Consid\'erons un flux $\phi(S_T)$ vers\'e \`a la date $T$ .\\
		Soit $\Q^{T^*}$ , la probabilit\'e $T^*$ -forward. \\
		La valeur du flux en $t < T$ est :\\
			\begin{equation*}
				\boxed{\text{Prix}(t, \phi(S_T)) \, = \, \E \left [ \frac{\beta(t)}{\beta(T)} \phi(S_T) \,
				/ \F_t \, \right ]  \, =\, B(t,T^*)\, \E_{\Q^{T^*}} \left [ \phi(S_T) \, / \F_t \,\right ]}
			\end{equation*}
\\
		\end{prop}
	\end{minipage}
}

\newpage	
\subsection{Pricing des caps dans le mod\`ele BGM : objectif Calibration}

	\begin{prop}[Prix d'un caplet] $\quad$ \label{PrixCaplet}
		Le prix en $t$ du  caplet versant en $T_i$ le diff\'erentiel entre le taux IBOR $L(T_{i-1}, 0 ; \delta)$ observ\'e en $T_{j-1}$ et 
le taux fixe $K$  est donn\'e par :
		\begin{equation*}
			\fbox{
			$\begin{array}{lrc}
				Caplet^{BGM}(t, T_i, K ) &=& \delta B(t, T_i) g_{K,\Sigma_{i-1}\sqrt{T_{i-1}-t}} \Big( L(t, (i-1) \delta ; \delta) \Big ) \\
			\end{array}$
			}
		\end{equation*}
\indent avec $\Sigma_{i-1} = \frac 1 {\sqrt{T_{i-1}-t}} \sqrt{ \int_t^{T_{i-1}} \gamma^2(s, T_{i-1}-s;  \delta) \dif s } $
	\end{prop}
	\begin{dem}[Prix d'un caplet]$\quad$
		Nous avons :
		\begin{eqnarray*}
			Caplet^{BGM} (t, T_i, K) &=& \E \left [  \frac{\beta(t)}{\beta(T_i)} 
						\left (  L(T_{i-1},  0;  \delta)  -  K \right )_+ \delta
						/ \, \F_t \, \right ] \\
					&=& B(t,T_i) \E _{\Q_{T_i}} \left [ 
						\left (  L(T_{i-1},  0 ;  \delta)  -  K \right )_+ \delta
						/ \, \F_t \, \right ] \\
		\end{eqnarray*}
		avec $\Q_{T_i}$ la probabilit\'e $T_i$-forward neutre. \\
\indent Or, en utilisant la diffusion  proposition \ref{dif IBOR fw} p. \pageref{dif IBOR fw}, on voit que l'on peut  
appliquer la formule de Black \& Scholes pour un call europ\'een avec drift nul\footnote{cf. proposition \ref{PrixCallBS}
p. \pageref{PrixCallBS}} :
		\begin{eqnarray*}
			E _{\Q_{T_i}} \left [ \left (  L(T_{i-1},  0 ;  \delta)  -  K \right )_+ \delta / 
			\, \F_t \, \right ]  &=& L(t, (i-1) \delta ; \delta) \N \left ( h(t, T_{i-1}) \right ) -
				 K \N \left ( h(t, T_{i-1}) - 
				\Sigma_{T_{i-1}} \sqrt{T_{i-1}-t} \right ) \\
		\end{eqnarray*}
\indent avec : 
		\begin{eqnarray*}
			\Sigma_{i-1} &:=& \frac 1 {\sqrt{T_{i-1}-t}}  
					\sqrt{ \int_t^{T_{i-1}} \gamma^2(s, T_{i-1}-s;  \delta) \dif s } \\
				&& \\
			h(t, T_{i-1}) &:=& \frac { 
						\log \left [ \frac {L(t, (i-1) \delta ; \delta )}{K} \right ] + 
						\frac 1 2 \Sigma^2_{i-1} (T_{i-1} - t)}
						{ \Sigma_{i-1} \sqrt{T_{i-1} - t}} \\
		\end{eqnarray*}
		\begin{flushright}
			$\boxslash$
		\end{flushright}
	\end{dem}


	\begin{prop}[Prix d'un cap]$\quad$ \label{PrixCap}
		Le prix d'un cap est donn\'e par :\\
			\begin{equation*}
				\fbox{
				$\begin{array}{lrc}
					Cap^{BGM}(t, T_n, K) &=& \sum_{i=1}^{i=n} B(t, T_i) 
						g_{K,\Sigma_{i-1} \sqrt{T_{i-1}-t}} \Big ( K(t, T_{i-1}) \Big ) \\
				\end{array}$
				}
			\end{equation*}		
	\end{prop}
	\begin{dem}[Prix d'un cap]$\quad$
		Il suffit d'\'ecrire :
		\begin{eqnarray*}
			Cap^{BGM} (t , T_n, K ) &=& \sum_{i=1}^{i=n} Caplet^{BGM}(t, T_i , K ) \\
		\end{eqnarray*}
		\begin{flushright}
			$\boxslash$
		\end{flushright}
	\end{dem}

\section{Calibration du mod\`ele BGM \`a un facteur}

\subsection{M\'ethode}
	
	Sur le march\'e des taux d'int\'er\^et, les caps sont des produits d\'eriv\'es parmi les plus liquides.   
Par coh\'erence avec le march\'e , la calibration du mod\`ele s'effectuera de mani\`ere \`a retrouver les prix de ces produits. 
	Pour calibrer notre mod\`ele, nous allons chercher les param\`etres $\gamma$ qui minimisent 
l'\'ecart entre les prix des caps sur le march\'e et les prix des caps donn\'es par le mod\`ele . \\
Autrement dit, on va chercher les param\`etres tels que :
	\begin{eqnarray}
		Cap^{Market} (t, T_n, K) &=& Cap^{BGM} (t, T_n , K) 
	\end{eqnarray}
	Compte-tenu des r\'esultats pr\'ec\'edents, on peut \'ecrire 
$\forall \, i \, \in \,  \llbracket \,  1 , n \, \rrbracket$ , 
	\begin{eqnarray*}
		B(t, T_i) g_{K, \sigma_{T_{i-1}} \sqrt{T_{i-1}-t}} \Big ( L(t, (i-1) \delta ; \delta) \Big ) &=& 
		B(t, T_i) g_{K, \Sigma_{T_{i-1}} \sqrt{T_{i-1}-t}} \Big ( L(t, (i-1) \delta ; \delta) \Big ) \\
	\end{eqnarray*}
On voit apparaitre les contraintes de calibration du mod\`ele \\
\\
 \\
	\fbox{
		\\
		\begin{minipage}{\textwidth}
			 {\underline{\bf Contraintes de Calibration}} \\ 
			\\
		$\forall \, i \, \in \,  \llbracket \,  0 , n-1 \, \rrbracket$ ,
			\begin{eqnarray}
				\frac 1 {T_i - t } \int_t^{T_i} \gamma^2(s, T_i - s ; \delta) \dif s &=& \sigma^2_{T_i} \label{Contraintes}
			\end{eqnarray}
		\end{minipage}
			
	}



\subsection{Hypoth\`eses sur la volatilit\'e}


	On suppose dans cette section que la fonction $\gamma$ ne d\'epend que de l'\'ecart \`a l'\'ech\'eance :
	\begin{eqnarray*}
		\gamma(t, \theta ; \delta) &=& \gamma(\theta ; \delta) \\
	\end{eqnarray*} 
\indent	On suppose \'egalement que $\gamma$ est une fonction en escalier sur 
$\left \{ T_0 , T_1 , \cdots , T_{n-1} \right \} $. \\
Autrement dit, la  volatilit\'e est constante sur chaque intervalle $\left [  T_i , T_{i+1} \right ] $ .\\
En notant $\gamma_i$ la valeur de la fonction sur l'intervalle $[T_i, T_{i+1}]$ avec $i \in \llbracket 0 , n-1 \rrbracket$, 
ceci s'\'ecrit :
	\begin{eqnarray}
		\gamma(\theta, \delta ) &=& \sum_{i=0}^{n-1} \gamma_i \1_{\{ i \delta \leq \theta < (i+1) \delta \}}   \label{Gamma}\\
		\nonumber
	\end{eqnarray}
La contrainte de calibration (\ref{Contraintes}) p. \pageref{Contraintes} s'\'ecrit alors :
	\begin{eqnarray*}
		\frac 1 {T_i - t} \int_0^{i \delta} \gamma^2(\tau, \delta) \dif \tau &=& \sigma^2_{T_i} \\
	\end{eqnarray*}
En remarquant que :	
	\begin{eqnarray*}
		\int_{i \delta}^{(i+1) \delta} \gamma^2(\tau, \delta) \dif \tau &=& 
		\int_0^{(i+1) \delta} \gamma^2(\tau , \delta) \dif \tau  - \int_0^{i \delta} \gamma^2(\tau , \delta) \dif tau \\
		\Rightarrow \delta \, \gamma^2_i &=& (i+1) \, \delta \, \sigma^2_{T_{i+1}} - i  \, \delta \, \sigma^2_{T_i} \\ 
	\end{eqnarray*}
on voit qu'il est possible de calibrer le mod\`ele de mani\`ere it\'erative en utilisant :
		\begin{displaymath}
			\left \{
				\begin{array}{rcl}
					\gamma_0 &=& \sigma_{T_0} \\
					\gamma^2_i &=& (i+1) \, \sigma^2_{T_{i+1}} \, - \, i \, \sigma^2_{T_i} \\
				\end{array}
			\right.
		\end{displaymath}

%*******************
\chapter{Simulation}
%*******************

	Pour valoriser des produits d\'eriv\'es sur taux d'int\'er\^et pour lesquels il n'existe pas de formules ferm\'ees, 
nous aurons recours \`a des simulation de Monte Carlo. Il est alors n\'ecessaire de fixer la mesure de probabilit\'e sous 
laquelle ces simulations seront effectu\'ees.\\
\indent	Nous utiliserons des versions discr\'etis\'ees des \'equations suivantes : \\
	\\
	\fbox{
		\begin{minipage}{\textwidth}
			\begin{eqnarray}
				\dif W_t^{T_{i+1}} &=& \dif W_t^{T_{i}} + \frac{\delta L(t,i \delta;  \delta)}
							{1+\delta L(t, i \delta;  \delta)}
							\gamma(i \delta;  \delta) \dif t \, , 
			\quad i \in \llbracket 1 , n \rrbracket \label{difBrFw}\\
				&& \nonumber\\
				\dif L(t,i \delta;  \delta) &=& 
				\underbrace{\gamma(i \delta;  \delta)}_{\text{\tiny{$= \, \gamma_i $ par (\ref{Gamma}) 
				p. \pageref{Gamma}}}} L(t, i \delta;  \delta) \dif W_t^{T_{i+1}} (t) \, , 
			\quad i \in \llbracket 0 , n-1 \rrbracket \label{difIborFw} \\
				\nonumber
			\end{eqnarray}
		\end{minipage}
	}
	\begin{rem}[Pas de discr\'etisation] $\quad$
		Le pas de discr\'etisation en temps de ces \'equations introduira un biais 
dans la simulation dont il faudra tenir compte.
	\end{rem}


\section{Choix d'une mesure pour la simulation}

	\begin{df}[Mesure Terminale] $\quad$
		On appelle mesure terminale la mesure $T_n$ -forward neutre \\
	\end{df}
	\begin{prop}[Taux IBOR Forward sous la mesure Terminale] $\quad$ \label{difBrTerm}
		$\forall i \in \llbracket 0 , n-1 \rrbracket$ :
		\begin{eqnarray*}
			\dif L(t, i \delta ; \delta ) &=& L(t, i \delta ; \delta) \left [ -
			\sum_{j=i+1}^{n-1} \left ( \frac{\delta L(t, j \delta ; \delta)}{1 + \delta L(t, j \delta ; \delta )}
			\, \gamma_j \, \gamma_i \right ) \dif t \, 
			+ \, \gamma_i \, \dif W^{T_n}_t 
			\right ]
		\end{eqnarray*}
	\end{prop}
\newpage
	\begin{dem} $\quad$
		On raisonne par r\'ecurrence sur $n$ \\
		\begin{description}
			\item[\underline{\bf{si $ n= 1 $}}] $\quad$
				La propri\'et\'e est trivialement vraie en utilisant 
				l'\'equation (\ref{difIborFw}) p. \pageref{difIborFw} \\
			\item
			\item[\underline{\bf{si la propri\'et\'e est vraie au rang $n$}}]$\quad$ on a alors :
				\begin{eqnarray*}
					\dif L(t, i \delta ; \delta ) &=& L(t, i \delta ; \delta) \left [ -
							\sum_{j=i+1}^{n-1} \left ( 
							\frac{\delta L(t, j \delta ; \delta)}
							{1 + \delta L(t, j \delta ; \delta )}
							\, \gamma_j\, 
							\gamma_i \right ) \dif t \, 
							+ \, \gamma_i \, \dif W^{T_n}_t \right ]
				\end{eqnarray*}
				On peut \'ecrire que :
				\begin{eqnarray*}
					\frac{\dif L(t, i \delta ; \delta )}{L(t, i \delta ; \delta)} &=& -
					\sum_{j=i+1}^{n} \left ( 
					\frac{\delta L(t, j \delta ; \delta)}{1 + \delta L(t, j \delta ; \delta )}
					\, \gamma_j \, 
					\gamma_i \right ) \dif t \, \\
					 &&+ \, \underbrace{ \frac {\delta \, L(t, n \delta ; \delta )}
					 {1 + \delta \, L(t, n \delta ; \delta)} \, 
					\gamma_n \, 
					\gamma_i \, \dif t  
					+ \, \gamma_i\, \dif W^{T_n}_t }_
					{= \, \gamma_i \dif W_t^{T_{n+1}} \quad \text{\tiny{par (\ref{difBrFw}) p. \pageref{difBrFw}}}}\\
				\end{eqnarray*}
			La propri\'et\'e est donc aussi vraie au rang $n+1$
		\end{description}
		\begin{flushright}
			$\boxslash$
		\end{flushright}
	\end{dem}
	\begin{rem}$\quad$
		On voit alors que l'on peut simuler tous les taux forwards sous cette mesure de probabilit\'e.
	\end{rem}

\section{Algorithme de Simulation}

\subsection*{$\rightarrow \quad$ Mise en place d'un maillage}
	
	\begin{description}
		\item	On consid\`ere un maillage r\'egulier $\left \{ t_0 = t = T_0 , \cdots , t_k , \cdots , t_m \right \}$ avec $m \in \mathbb{N}^*$\\  
		\item	On note $\Delta_t$, le pas de discr\'etisation en temps.\\
		\item 	On a $t_k = t + k \Delta_t$
	\end{description}

\subsection*{$\rightarrow \quad$ Discr\'etisation des \'equations}
	En notant $\Delta W^{T_{i+1}} (t_k)$, l'accroissement du brownien sous la probabilit\'e $T_{i+1}$- forward sur 
l'intervalle $\left [ t_k ,  t_{k+1}  \right ]$, l'algorithme de simulation s'appuiera sur les \'equation suivantes 
\footnote{L'expression des taux IBOR sous forme exponentielle leur assure d'\^etre positifs} : \\
	\newline
	\fbox{	
		\begin{minipage}{\textwidth}
			\begin{eqnarray}
				\Delta W^{T_i}(t_k) &=& \Delta W^{T_{i+1}}(t_k) - \frac {\delta \, L(t_k , i \delta ; \delta)}
					{1 + \delta \, L(t_k , i \delta ; \delta)} \gamma_i \Delta_t  \, , 
				\quad i \in \llbracket 1 , n \rrbracket  \quad \text{et} \quad t_k \leq T_i\\
				L(t_{k+1} , i \delta ; \delta ) &=& L(t_k , i \delta ; \delta) 
					\exp{\left \{ \gamma_i \Delta W^{T_{i+1}}(t_k) - \frac 1 2 \gamma^2_i \Delta_t \right \}} 
					\, , \quad i \in \llbracket 1 , n \rrbracket  \quad \text{et} \quad t_k \leq T_i
			\end{eqnarray}
		
		\end{minipage}
	}
\newpage
\subsection*{$\rightarrow \quad$ \Large{Simulation pas \`a pas}}

	\fbox{\bf Inputs du mod\`ele : $\quad$ La courbe des taux forwards 
$\left \{ L(t, i \delta ; \delta) , i \in \llbracket 0 , n-1 \rrbracket \right \} $ }\\
	
	\begin{enumerate}
		\item \fbox{\bf $1^{\text{er}}$ point du maillage} \\
\\
			\bf{On cherche les $L(t_1 , i \delta ; \delta)$ pour $i \in \llbracket 0 , n-1 \rrbracket $} \\
			\begin{enumerate}
				\item \underline{Accroissements du brownien} \\
 					\begin{itemize}
						\item	On commence par simuler $\Delta W^{T_n}(t)$ comme une variable 
							al\'eatoire normale centr\'ee et de variance $\Delta_t$ 
							\footnote{cf. algorithme de Box \& Muller en annexe} 
						\item 	On obtient tous les $\Delta W^{T_i}(t)$ par la relation :
							\begin{eqnarray*}
								\Delta W^{T_i}(t) &=& \Delta W^{T_{i+1}}(t) - 
								\frac{\delta \, L(t , i \delta ; \delta)}
								{1 + \delta \, L(t , i \delta ; \delta)} \, \gamma_i \Delta_t
								\, , \quad i = 0 , \cdots , n-1 \\
							\end{eqnarray*}	
					\end{itemize} 
				\item \underline{Taux IBOR en $t_1$} \\
					\\
					On peut alors \'evaluer : \\
					\begin{eqnarray*}
						L( t_1 , i \delta ; \delta ) &=& L(t , i \delta ; \delta ) 
						\exp {\left \{ \gamma_i \Delta W^{T_{i+1}}(t) - \frac 1 2 \gamma^2_i \Delta_t
						\right \}} \, , \quad i = 0 , \cdots , n-1
					\end{eqnarray*}
			\end{enumerate}
		\item \fbox{\bf $2^{\text{\`eme}}$ point du maillage}\\
\\
			\bf{On cherche les $L(t_2 , i \delta ; \delta)$ pour $i \in \llbracket 0 , n-1 \rrbracket $} \\
			On connait \`a ce stade les $L(t_1 , i \delta ; \delta)$\\
			\begin{enumerate}
				\item \underline{Accroissements du brownien} \\
 					\begin{itemize}
						\item	On commence par simuler $\Delta W^{T_n}(t_1)$ comme une variable 
							al\'eatoire normale centr\'ee et de variance $\Delta_t$ 
						\item 	On obtient tous les $\Delta W^{T_i}(t_1)$ par la relation :
							\begin{eqnarray*}
								\Delta W^{T_i}(t_1) &=& \Delta W^{T_{i+1}}(t_1) - 
								\frac{\delta \, L(t_1 , i \delta ; \delta)}
								{1 + \delta \, L(t_1 , i \delta ; \delta)} \, \gamma_i \Delta_t
								\, , \quad i = 0 , \cdots , n-1 \\
							\end{eqnarray*}	
					\end{itemize} 
				\item \underline{Taux IBOR en $t_2$} \\
					\\
					On peut alors  \'evaluer : \\
					\begin{eqnarray*}
						L( t_2 , i \delta ; \delta ) &=& L(t_1 , i \delta ; \delta ) 
						\exp {\left \{ \gamma_i \Delta W^{T_{i+1}}(t_1) - \frac 1 2 \gamma^2_i \Delta_t
						\right \}} \, , \quad i = 0 , \cdots , n-1
					\end{eqnarray*}
			\end{enumerate}
		\item[]	\begin{center}
				\vdots 
			\end{center}   
		\item[k.]\fbox{\bf $k^{\text{\`eme}}$ point du maillage}\\ 
\\
			\bf{On cherche les $L(t_k , i \delta ; \delta)$ pour $i \in \llbracket 0 , n-1 \rrbracket $}  \\
			On connait \`a ce stade les $L(t_{k-1} , i \delta ; \delta)$ \\
			\begin{enumerate}
				\item \underline{Accroissements du brownien} \\
 					\begin{itemize}
						\item	On commence par simuler $\Delta W^{T_n}(t_{k-1})$ comme une variable 
							al\'eatoire normale centr\'ee et de variance $\Delta_t$ 
						\item 	On obtient tous les $\Delta W^{T_i}(t_{k-1})$ par la relation :
							\begin{eqnarray*}
								\Delta W^{T_i}(t_{k-1}) &=& \Delta W^{T_{i+1}}(t_{k-1}) - 
								\frac{\delta \, L(t_1 , i \delta ; \delta)}
								{1 + \delta \, L(t_1 , i \delta ; \delta)} \, \gamma_i \Delta_t
								\, , \quad i = 0 , \cdots , n-1 \\
							\end{eqnarray*}	
					\end{itemize} 
				\item \underline{Taux IBOR en $t_k$} \\
					\\
					 On peut alors \'evaluer : \\
					\begin{eqnarray*}
						L( t_k , i \delta ; \delta ) &=& L(t_{k-1} , i \delta ; \delta ) 
						\exp {\left \{ \gamma_i \Delta W^{T_{i+1}}(t_{k-1}) - \frac 1 2 \gamma^2_i \Delta_t
						\right \}} \, , \quad i = 0 , \cdots , n-1
					\end{eqnarray*}
			\end{enumerate}
		\item[] \begin{center} 
				\vdots 
			\end{center}  
	\end{enumerate}

	
\subsection*{}
	Nous r\'ecapitulons les \'etapes de simulation dans le tableau suivant :
		\begin{displaymath}
			\begin{array}{|c|c|c|c|c|c|}
		\hline 
			k/i&0&1& \cdots&n-2&n-1 \\
		\hline
			&\Delta W^{T_1}(t)&\Delta W^{T_2}(t)&\cdots &\Delta W^{T_{n-1}}(t)&\Delta W^{T_n}(t) \\
			1&L(t_1,0; \delta)& L(t_1,\delta; \delta)&\cdots&L(t_1,(n-2) \delta; \delta)&
			L(t_1,(n-1) \delta; \delta) \\
		\hline 
			&\Delta W^{T_1}(t_1)&\Delta W^{T_2}(t_1)&\cdots &\Delta W^{T_{n-1}}(t_1)&\Delta W^{T_n}(t_1) \\
			2&L(t_2,0; \delta)& L(t_2,\delta; \delta)&\cdots&L(t_2,(n-2) \delta; \delta)&
			L(t_2,(n-1) \delta; \delta) \\
		\hline
			\vdots&\vdots&\vdots&\vdots&\vdots&\vdots \\
		\hline
			&\Delta W^{T_1}(t_{k-1})&\Delta W^{T_2}(t_{k-1})&\cdots &\Delta W^{T_{n-1}}(t_{k-1})
			&\Delta W^{T_n}(t_{k-1}) \\
			k&L(t_k,0; \delta)& L(t_k,\delta; \delta)&\cdots&L(t_k,(n-2) \delta; \delta)&
			L(t_k,(n-1) \delta; \delta) \\
		\hline
			\vdots&\vdots&\vdots&\vdots&\vdots&\vdots \\
		\hline
			&\Delta W^{T_1}(t_{m-1})&\Delta W^{T_2}(t_{m-1})&\cdots &\Delta W^{T_{n-1}}(t_{m-1})
			&\Delta W^{T_n}(t_{m-1}) \\
			m&L(t_m,0; \delta)& L(t_m,\delta; \delta)&\cdots&L(t_m,(n-2) \delta; \delta)&
			L(t_m,(n-1) \delta; \delta) \\
		\hline
			\end{array}
	\end{displaymath}
\\
\indent Ainsi, il nous est maintenant possible de d\'eterminer toutes les trajectoires des taux IBOR forwards 
et par suite d'obtenir les $L(T_i, 0 ; \delta)$ qui vont nous permettre de valoriser nos produits.

%**************************************************************************
\chapter{Valorisation de produits dans le mod\`ele BGM }
%**************************************************************************

\section{Taux IBOR forward de maturit\'e 1 an}

	\fbox{
	\begin{minipage}{\textwidth}
	\begin{prop}$\quad$\label{IBOR1an}
		Les taux IBOR forwards de maturit� 1 an $\left \{ L(t, i\delta; 2 \delta) , i \right\}$ s'expriment 
en fonctions des taux IBOR de maturit\'e 6 mois de la mani\`ere suivante : \\
		\begin{eqnarray*}
			L(t , i \delta ; 2 \delta) &=& \frac{L(t,i\delta ; \delta) + L(t,(i+1)\delta ; \delta)}{2} + 
						\frac{L(t,i\delta ; \delta)L(t,(i+1)\delta ; \delta)}{4} \\
		\end{eqnarray*}
	\end{prop}
	\end{minipage}}

	\begin{dem}$\quad$
		Nous avons :
		\begin{description}
			\item[(i)] d'une part :
				\begin{eqnarray*}
					L(t, i \delta ; \delta ) &=& \frac 1 {\delta} \left [ 
					\frac{B(t, t + i \delta)}{B(t, t + (i+1) \delta)} - 1\right ] \\
					&&\text{\tiny{cf. remarque \ref{IBORFwZc} p. \pageref{IBORFwZc}}} \\
					\Rightarrow \quad \frac{B(t, T_i)}{B(t, T_{i+1})}
					&=& \delta L(t, i \delta , \delta ) -1
				\end{eqnarray*} 
					
			\item[(ii)] de la m\^eme fa\c con, 
				\begin{eqnarray*}
					 \frac{B(t, T_{i+1})}{B(t,T_{i+2})}
					&=& \delta L(t, (i+1) \delta ; \delta ) -1
				\end{eqnarray*}
			\item[(iii)] Utilisons la remarque \ref{IBORFwZc} p. \pageref{IBORFwZc} pour \'evaluer 
				$L(t, i\delta ; 2 \delta)$ :
				\begin{eqnarray*}
					L(t, i \delta ; \delta ) &=& \frac 1 {2 \delta} \left [ 
					\frac{B(t, t + i \delta)}{B(t, t + (i+2) \delta)} - 1 \right ] \\
							&\text{\tiny{ $\delta = 0.5$}}& \\
					&=&\frac{B(t, T_i)}{B(t, T_{i+2})} - 1\\
					&=& \frac{B(t,T_i)}{B(t, T_{i+1})} \frac{B(t, T_{i+1})}{B(t, T_{i+2})} - 1 \\
					&=& \left ( \delta L(t, i \delta , \delta ) -1\right )
					\left ( \delta L(t, (i+1) \delta , \delta ) -1\right ) - 1 \\
					&\text{\tiny{par (i) et (ii)}}& \\
					&=& \frac{L(t,i\delta ; \delta) + L(t,(i+1)\delta ; \delta)}{2} + 
						\frac{L(t,i\delta ; \delta)L(t,(i+1)\delta ; \delta)}{4} \\
				\end{eqnarray*} 
		\end{description}
		\begin{flushright}
			$\boxslash$
		\end{flushright}
	\end{dem}

	\begin{rem}$\quad$
		\begin{description}
			\item[$\bullet$]	On voit par cette relation qu'en simulant les trajectoires des taux IBOR 6 mois, 
						on va pouvoir valoriser des produits dont le sous-jacent est l'IBOR 1 an. 
			\item[$\bullet$]	D'autre part, le taux IBOR de maturit\'e 1 an apparait comme 
						une combinaison des IBOR 6 mois.\\ En g\'en\'eral, ces processus ne pourront 
						\^etre consid\'er\'es simultan\'ement comme log-normaux.
			\item[$\bullet$] 	Dans le cas de produit {\bf ne d\'ependant que} du IBOR 1 an, 
						il est possible de donner une diffusion au processus et de calibrer 
						le mod\`ele en consid\'erant des produits liquides sur IBOR 1 an. 
			\item[$\bullet$]	Pour valoriser des produits dont les sous-jacents sont l'IBOR 
						pour diff\'erentes maturit\'es, il est plus pertinent d'utiliser le mod\`ele 
						BGM pour le taux \`a la plus petite maturit\'e, puis d'en d\'eduire les
						suivants \`a l'aide de la proposition \ref{IBOR1an} p. \pageref{IBOR1an}.  
		\end{description}		 
	\end{rem}


	
\section{M\'ethode de Monte-Carlo}
	
	Pour valoriser nos produits avec la m\'ethode de Monte-Carlo, nous devons \'evaluer les  payoff sous la mesure 
choisie pour effectuer les simulation : soit la mesure terminale $\Q^{T_n}$.
Pour ce faire nous proc\'ederons  comme suit :
	\begin{description}
		\item[a.]$\quad$ Evaluation des z\'ero coupon de maturit\'e $T_n$ en chaque date de tomb\'ee des flux 
				avec les taux simul\'es par :
				\begin{eqnarray*}
					B(T_i, T_n) &=& \prod_{j=i}^{n-1} \left [ 1 + \delta L(T_i, j \delta ; \delta)
					\right ]^{-1}
				\end{eqnarray*}
		\item[b.]$\quad$ Division de chaque payoff par la valeur du z\'ero-coupon \`a la date de paiement du flux\\
				On a ainsi exprim\'e chaque payoff sous le num\'eraire $B(.,T_n)$ associ\'e \`a la mesure 
				terminale $\Q^{T_n}$
		\item[c.]$\quad$ Actualisation \`a la date de valorisation en multipliant par $B(t,T_n)$
	\end{description}	

\section{Mise en place et valorisation d'un swap de volatilit\'es dans le mod\`ele BGM}
\subsection{Swap et Taux de swap}
				
	\begin{df}[Swap de taux]$\quad$ \label{swap}
		Un swap de taux sur une unit\'e de devise est un contrat versant \`a une fr\'equence donn\'ee 
le diff\'erentiel entre un taux $K$ pr\'ealablement fix\'e et un taux variable. \\
	\end{df}
\indent	Typiquement, pour un swap dont les dates de versements sont semestrielles de la forme 
	$T_i = T_0 + i \delta$ avec $i = 1 , \cdots  n$, le taux re\c cu \`a la date $T_i$ est le taux IBOR :
		\begin{eqnarray*} 
			L(T_{i-1}, 0 , \delta) &=& \frac 1 {\delta} \left ( \frac 1 {B(T_{i-1}, T_i)} - 1 \right )
		\end{eqnarray*}	
	Nous nous placerons dans ce cadre.
	\begin{prop}[Valorisation du swap]$\quad$\label{PrixSwap}
		Le prix du swap en $t$ est donn\'e par :
		\begin{displaymath}
			\boxed{swap(t,T_n,K) \,=\, B(t,T_0) - B(t, T_n) - K \delta \sum_{i=1}^{n} B(t, T_i) }
		\end{displaymath}
	\end{prop}
	\begin{dem}
		\begin{description}
			\item[$\bullet \quad$ Valorisation de la jambe fixe :]  
				\begin{eqnarray*}
					\E \, \left[ \sum_{i=1}^{n} \frac{\beta(t)}{\beta(T_i)} 
					\delta \, K \, / \, \F_t \right ]
					&=& \delta \, K \, \sum_{i=1}^{n} \E \,\left[\frac{\beta(t)}{\beta(T_i)} 
					/ \, \F_t \right ]\\
					&=& \delta \, K \, \sum_{i=1}^{n} B(t, T_i) 
				\end{eqnarray*}
			\item[$\bullet \quad$ Valorisation de la jambe variable :]
				\begin{eqnarray*}
					\E \, \left[ \sum_{i=1}^{n} \frac{\beta(t)}{\beta(T_i)} 
					\delta L(t, (i-1)\delta ; \delta)/ \, \F_t \right ]
					&=& \sum_{i=1}^{n}\E \, \left[\frac{\beta(t)}{\beta(T_i)} 
					\delta L(t, (i-1)\delta ; \delta)/ \, \F_t \right ] \\
					\text{\tiny{par remarque \ref{IBORFwZc} p. \pageref{IBORFwZc}}}
					&=& \sum_{i=1}^{n}\E \, \left[\frac{\beta(t)}{\beta(T_i)}
						\left (\frac 1{B(T_{i-1}, T_i)} -1 \right ) / \, \F_t \right ] \\
					&=& \sum_{i=1}^{n}\E \, \left[\frac{\beta(t)}{\beta(T_i)}\frac 1{B(T_{i-1}, T_i)}
					 / \, \F_t \right ] - \sum_{i=1}^{n}\E \, \left[\frac{\beta(t)}{\beta(T_i)}
					/ \, \F_t \right ] \\
					&=& \sum_{i=1}^{n}\E \, \left[\frac{\beta(t)}{\beta(T_{i-1})B(T_{i-1}, T_i)}
					\frac{\beta(T_{i-1})}{\beta(T_i)}  / \, \F_t \right ]
					- \sum_{i=1}^{n} B(t, T_i) \\
					&=& \sum_{i=1}^{n}\E \, \left[\frac{\beta(t)}{\beta(T_{i-1})B(T_{i-1}, T_i)}
					\E \left[ \frac{\beta(T_{i-1})}{\beta(T_i)}  / \, \F_{T_{i-1}} \right ]
					/ \, \F_t \right ] - \sum_{i=1}^{n} B(t, T_i) \\
					&=& \sum_{i=1}^{n}\E \, \left[\frac{\beta(t)}{\beta(T_{i-1})B(T_{i-1}, T_i)}
					B(T_{i-1}, T_i)/ \, \F_t \right ] - \sum_{i=1}^{n} B(t, T_i) \\
					&=& \sum_{i=1}^{n} B(t, T_{i-1}) - \sum_{i=1}^{n} B(t, T_i) \\
					&=& B(t, T_0) - B(t, T_n)
				\end{eqnarray*}
			\item En se pla\c cant en position payeuse du taux fixe $K$ on a :
				\begin{eqnarray*}
					swap(t, T_n, K) &=& \, B(t,T_0) - B(t, T_n) - K \delta \sum_{i=1}^{n} B(t, T_i) 
				\end{eqnarray*}
		\end{description}
		\begin{flushright}
			$\boxslash$
		\end{flushright}
	\end{dem}
	\begin{df}[Taux de swap]$\quad$ \label{TauxSwap}
		Le taux de swap (associ\'e \`a une maturit\'e de swap) est la valeur $K$ qui annule le prix du swap en $t$.\\
		Il est donn\'e par :
		\begin{displaymath}
			\boxed{
			S_{0,n} \, = \, \frac{B(t,T_0) - B(t, T_n)}{\delta \sum_{i=1}^{n} B(t, T_i)}}
		\end{displaymath} 
	\end{df}
	\begin{rem}$\quad$
		Pour une maturit\'e donn\'ee, le taux de swap est ind\'ependant de la jambe variable
	\end{rem}
\newpage
\subsection{Structure du contrat}
	On consid\`ere la structure suivante :
	\begin{description}
		\item[Date de signature :]$\quad t$ \\
		\item[Date de d\'epart :]$\quad T_0$ \\
		\item[Nominal :]$\quad N$ \officialeuro \\
		\item[Ech\'eancier :]$\quad \left\{ T_1, \cdots , T_n \right\} \quad \text{avec}$ 
					\begin{center}
					$\left \{
					\begin{array}{cccc}
						T_i &=& T_0 + 2\, i \, \delta& \\
						\delta &=& 0.5  &\text{ie. 6 mois} \\
						T_n &=& T_0 + 20 \, \delta &\text{ie. 10 ans}
					\end{array} \right.$\\
					\end{center}
		\item
		\item[Jambe payeuse :]$\quad$  A chaque date $T_i$, avec $i \in \llbracket 1 , n \rrbracket $ 
			(ie. tous les semestres), on paie le flux : 
			\begin{displaymath}
				N \, \delta \left [ L(T_{i-1}, 0, \delta) + 2 bps \right ]
			\end{displaymath}
		\item
		\item[Jambe receveuse :]$\quad$ A chaque date $T_i$ avec $ i \in \{ 2 , 4, 6, 8, \cdots , 18, 20 \}$ 
			(ie. tous les ans), on re\c coit le flux :
			\begin{displaymath}
				x \, N \, \delta | S_{i,i + 20 \delta} - S_{i-1, i - 1 +20 \delta} |
			\end{displaymath}
			avec :
			\begin{displaymath}
				\left \{
				\begin{array}{cc}
					S_{i, i + 20 \delta}&\text{le taux de swap 10 ans d\'etermin\'e \`a la date} T_i \\
					x&  \text{une quantit\'e non-al\'eatoire annulant la valeur du contrat en } t 
				\end{array}
				\right.
			\end{displaymath}
			
	\end{description}
\subsection{Valorisation de la jambe payeuse}
	\begin{prop}[Prix en t de la jambe payeuse]$\quad$
		Nous avons :
		\begin{displaymath}
			\boxed{\text{prix( jambe payeuse )} \, = \,  N \, \delta \, \sum_{i=1}{20} 
					\left [ L(t,(i-1) ; \delta) + 2 bps  \right ]}
		\end{displaymath}
	\end{prop}
	\begin{dem}$\quad$
		On \'evalue :
		\begin{eqnarray*}
			\text{prix( jambe payeuse )} &=& \E \, \left[ \sum_{i=1}^{20} \frac{\beta(t)}{\beta(T_i)}		
			N \, \delta \left( L(T_{i-1}, 0 ; \delta) + 2 bps \right) / \F_t \right] \\
			&=& N \, \delta \sum_{i=1}^{20}\E \, \left[\frac{\beta(t)}{\beta(T_i)}
			\left( L(T_{i-1}, 0 ; \delta) + 2 bps \right) / \F_t \right] \\
			&=& N \, \delta \sum_{i=1}^{20} B(t, T_i) \E_{\Q_{T_i}}  \, 
			\left[L(T_{i-1}, 0 ; \delta) + 2 bps/ \F_t \right] \\
			&&\text{\tiny{par la proposition \ref{pricing} p. \pageref{pricing}}} \\
			&=& N \, \delta \sum_{i=1}^{20} B(t, T_i) \left [ L(t, (i-1) \delta ; \delta) + 2 bps\right ]\\
			&&\text{\tiny{par la proposition \ref{dif IBOR fw} p. \pageref{dif IBOR fw}}} 
		\end{eqnarray*}
		\begin{flushright}
			$\boxslash$
		\end{flushright}
	\end{dem}

\subsection{Expression et valorisation de la jambe receveuse}

	\begin{prop}$\quad$
		Le taux de swap s'exprime en fonction des IBOR de maturit\'e 1 an de la mani\`ere suivante :
		\begin{displaymath}
			\boxed{
			S_{i, i + 20 \delta} =  \frac{1-\prod\limits_{j=i+1}^{10} \frac 1 
			{1 + 2 \delta L(T_i, 2 j \delta ; 2 \delta)}}
			{\sum_{j=i+1}^{10} 2 \delta \prod\limits_{k=i+1}^{j} \frac 1 {1+ 2 \delta L(T_i, 2k \delta ; 2 \delta)}}
			}
		\end{displaymath}
	\end{prop} 
	\begin{dem}$\quad$
		Par la d\'efinition du taux de swap, on peut \'ecrire que :
		\begin{eqnarray*}
			S_{i, i + 20 \delta} &=&  \, \frac{1 - B(T_i, T_{i + 20 \delta})}
			{2 \delta \sum_{j=i+1}^{i+10} B(T_i, T_{2j})}
		\end{eqnarray*}
		Nous avons de plus pour $k > i$ :
		\begin{eqnarray*}
			B(T_i, T_k) &=& \frac{B(T_i, T_{k})}{B(T_i, T_{k-1})} \,  \frac{B(T_i, T_{k-1})}{B(T_i, T_{k-2})}
			\, \frac{B(T_i, T_{k-2})}{B(T_i, T_{k-3})} \cdots \frac{B(T_i, T_{i+1})}{B(T_i, T_i))}
		\end{eqnarray*}
		et on a vu \`a la remarque \ref{IBORFwZc} p. \pageref{IBORFwZc} que :
		\begin{eqnarray*}
				\frac{B(T_i, T_k)}{B(T_i, T_{k+1})}&=& 1 + \tau L(T_i,  \tau ; \tau)
		\end{eqnarray*}
			avec $ T_{k+1} - T_k = \tau $\\
		En adaptant ces r\'esultats \`a notre cadre de travail, on obtient :
		\begin{eqnarray*}
			B(T_i , T_{i + 20 \delta}) &=& \prod\limits_{j=i+1}^{i+10} 
				\frac 1 {1 + 2 \delta L(T_i, 2 j \delta ; 2 \delta)}	
		\end{eqnarray*} 
			\`a substituer au num\'erateur et de la m\^eme fa\c con :
		\begin{eqnarray*}
				B(T_i, T_2j) &=& \prod\limits_{k=i+1}^{j} \frac 1 {1 + 2 \delta L(T_i , 2 k \delta ; 2 \delta)}
		\end{eqnarray*}
		au d\'enominateur.\\
		Ainsi : 
		\begin{eqnarray*}
			S_{i, i + 20 \delta} &=& \frac{1-\prod\limits_{j=i+1}^{10} \frac 1 
			{1 + 2 \delta L(T_i, 2 j \delta ; 2 \delta)}}
			{\sum_{j=i+1}^{10} 2 \delta \prod\limits_{k=i+1}^{j} \frac 1 
			{1+ 2 \delta L(T_i, 2k \delta ; 2 \delta)}}
		\end{eqnarray*}
		\begin{flushright}
			$\boxslash$
		\end{flushright}
	\end{dem}
\newpage
	\begin{rem}[Expression de $S_{i, i + 20 \delta}$ en fonction des taux IBOR de maturit\'e 6 mois]
		En substituant les $L(T_i, 2j \delta ; 2 \delta)$ par leur expression en fonction des 
		$L(T_i, k \delta ; \delta)$ donn\'ee \`a la proposition \ref{IBOR1an} p. \pageref{IBOR1an} :
		\begin{eqnarray*}
			L(T_i, 2 j \delta ; 2 \delta) &=& \frac{L(T_i, 2 j \delta ; \delta) + 
			L(T_i, 2 (j+1) \delta ; \delta)} 2 +  \frac{L(T_i, 2 j \delta ; \delta) \, 
			L(T_i, 2 (j+1) \delta ; \delta)} 4 
		\end{eqnarray*}
	On obtient une expression pour le taux de swap en fonction des taux IBOR de maturit\'e 6 mois : 
		\begin{eqnarray*}
			S_{i, i + 20 \delta} &=& 
			\frac{
				1-\prod\limits_{j=i+1}^{10} 
				\left [ 
					{ 1 + 2 \, \delta 
					\left \{ 
					\frac{
						L(T_i, 2 j \delta ; \delta) + L(T_i, 2 (j+1) \delta ; \delta)}
						 2 +  
					\frac{
						L(T_i, 2 j \delta ; \delta) \, L(T_i, 2 (j+1) \delta ; \delta)} 
						4 
					\right \}
					}  \right]^{-1}
				}
				{
				\sum_{j=i+1}^{10} 2 \delta \prod\limits_{k=i+1}^{j} \left [
					{
					1+ 2 \, \delta 
				\left \{ 
					\frac{
						L(T_i, 2 k \delta ; \delta) + L(T_i, 2 (k+1) \delta ; \delta)} 
						2 +  
					\frac{
						L(T_i, 2 k \delta ; \delta) \, L(T_i, 2 (k+1) \delta ; \delta)}
					 4 
				\right \}
					} \right ]^{-1}
			}
		\end{eqnarray*}
	On voit alors qu'en simulant les trajectoires des taux IBOR de maturit\'e 6 mois avec notre mod\`ele, on va 
	pouvoir valoriser la jambe receveuse du contrat en utilisant leur position en $T_i$.
	\end{rem}
	\begin{rem}$\quad$
		La quantit\'e non al\'eatoire $x$ doit \^etre fix\'ee de mani\`ere \`a ce que jambe payeuse et jambe 
	receveuse se compensent \`a la signature du contrat. 
	\end{rem}
	\begin{rem}$\quad$
		Pour valoriser la jambe receveuse, il s'agit d'\'evaluer :
		\begin{displaymath}
			\E \left[ 
				\sum_{k=1}^{10} \frac{\beta(t)}{\beta(T_{2k})} \, x \, N \, \delta
				 \left |S_{2k, 2 (k+10) \delta} - S_{k-1 , 2((k-1) +10) \delta} \right | \, / \, \F_t 
			\right]
		\end{displaymath}
	\end{rem}

\subsection{M\'ethode employ\'ee}

	Pour \'evaluer le prix de la jambe receveuse, nous proc\'ederons comme suit : \\
	\begin{description}
		\item[(i)$\quad$]	Simulation des $L(T_{2k}, 2j \delta ; \delta)$ pour :
			\begin{center}
			$\left \{
				\begin{array}{ccc}
					k &=& 0 , \cdots , 10 \\
					j &=& k+1 , \cdots ,\, k+11 
				\end{array}
			\right.$
			\end{center}  
		\item
		\item[(ii)$\quad$]	Calcul des $L(T_{2k} , 2 j \delta ; 2 \delta)$ pour :
			\begin{center}
			$\left \{
				\begin{array}{ccc}
					k &=& 0 , \cdots , 10 \\
					j &=& k+1 , \cdots ,\, k+10  
				\end{array}
			\right.$
			\end{center}  
		\item
		\item[(iii)$\quad$]	Calculs des $S_{i,2(i+10) \delta}$ pour $i = 1, \cdots , 10$
		\item[(iv)$\quad$] Evaluation des payoffs sous  la mesure terminale comme \'evoqu\'e pr\'ec\'edemment. 
		\item[(v)$\quad$] Calculs de la valeur de $x$ \'egalisant les deux jambes \`a la signature du contrat.
	\end{description}

%*******************
\chapter*{Conclusion}
%*******************
	Le mod\`ele impl\'ement\'e nous permet de simuler les trajectoires des taux IBOR et par suite de valoriser 
tout produit d\'eriv\'e dont les flux sont fonction  de ces taux.\\
\indent C\^ot\'e informatique nous avons d\'evelopp\'e des DLL en C++ sous la plateforme Visual Studio Express Edition. 
Celles-ci sont alors int\'egr\'ees dans un pricer Excel. \\
\\
\indent N\'eanmoins, certaines am\'eliorations seront \`a apporter :
	
\section*{Probl\`eme de discr\'etisation}
	Une source de biais dans la simulation est d\^ue \`a la discr\'etisation des \'equations (\ref{difBrFw}) et 
(\ref{difIborFw}) p. \pageref{difBrFw}.\\
En effet l'int\'egrale :
	\begin{displaymath}	
		 \int_{t + k \Delta_t}^{t + (k+1) \Delta_t } \frac{\delta \, L(s, (i-1) \delta ; \delta)}
			{1 + \delta \, L(s, (i-1) \delta ; \delta)}\gamma(s, (i-1) \delta ; \delta) \, \dif s 
	\end{displaymath}
est approch\'ee sur l'intervalle $\left [  t_k, t_{k+1}\right]$ par :
	\begin{displaymath}
		\frac{\delta \, L( t_k , (i-1) \delta ; \delta) }{1 + \delta \, L( t_k , (i-1) \delta ; \delta)}
		\gamma(t_k, (i-1)  \delta ; \delta) \Delta_t
	\end{displaymath}
Lorqu'on simule sous la mesure Terminale $\Q^{T_n}$, le taux Ibor $L( \cdot \, ,\,  (n-1) \delta ; \delta)$ 
est bien une martingale sous cette mesure pour tout pas de discr\'etisation. 
Puis $L( \cdot \, ,\,  (n-2) \delta ; \delta)$ est martingale sous la probabilit\'e $\widehat{\Q}_{T_n-1}$ d\'efinie par la discr\'etisation.
 Or on a  $ \widehat{\Q}_{T_{n-1}} \neq \Q_{T_{n-1}} $ et la propri\'et\'e de martingale se perd.\\
A. Brace, M. Musiela \& E. Schl\"ogl ont trait\'e ce sujet dans \cite{5} et ont propos\'e comme alternative d'utiliser l'
\'equation suivante :
	\begin{eqnarray*}
		\Delta W_{T_{i+1}} (t_k) &=& \Delta W_{T_{i+2}}(t_k) - c_{i,i+1}(t_k) \Delta_t
	\end{eqnarray*}
avec 
	\begin{eqnarray*}
		c_{i,i+1}(t_k) &=& \frac 1{\Delta_t} \frac 1{\gamma_i} \ln{\left ( 
		\frac{1 + \delta L( t_k , (i+1) \delta ; \delta ) \exp{\left \{  \gamma_i \gamma_{i+1}  \Delta_t \right \}} }
		{1 + \delta L( t_k , (i+1) \delta ; \delta )}\right )}
	\end{eqnarray*}

\newpage
\section*{Corr\'elation entre les diff\'erents taux forwards}

	Nous avons mis en place un mod\`ele unidimensionnel bas\'e sur les prix des caps. Ce mod\`ele ne permet pas de tenir 
compte des corr\'elations entre les diff\'erents taux forwards. \\ Il est possible d'\'etendre notre mod\`ele \`a un mod\`ele 
multi-facteurs. Une Analyse en Composantes Principales montre qu'il faudrait au minimum trois facteurs pour assurer la 
fiabilit\'e du mod\`ele. Il pourra alors \^etre calibr\'e de mani\`ere \`a retrouver  simultan\'ement les prix des caps et 
des swaptions \`a la monnaie.\\
Ces sujets sont trait\'es dans \cite{1}.
  
%***********************************************
\part{Produits d\'eriv\'es sur le p\'etrole   \\
	March\'e \'etranger et risque de change}
%***********************************************
%***************************************
\chapter*{Notations et Cadre de travail}
%***************************************

\subsection*{On notera : }
	\begin{description}
		
		\item [$\quad t $] la date courante
		\item [$\quad T $] d\'esignera en g\'en\'eral l'\'ech\'eance d'un produit d\'eriv\'e 
		\item [$\quad r_t $]  le taux sans risque \`a la date t sur notre march\'e 
		\item [$\quad \beta(t) \, = \, e^{\int_0^t r_s \dif s} $]  le facteur d'accumulation entre 0 et $t$ 
		\item [$\quad B(t,T) $]  le prix en t d'un z\'ero-coupon de maturit\'e $T$
		\item [$\quad S $] un sous-jacent exprim\'e en monnaie domestique
		\item [$\quad F_t(S,T) $]   le $T$-prix forward de $S$ en $t$
		\item [$\quad Call_t(T,K,S) $] le prix en t d'un call europ\'een sur $S$ de strike $K$ et \'ech\'eance $T$
		\item [$\quad Put_t(T,K,S) $]  le prix en t d'un put europ\'een sur $S$ de strike $K$ et \'ech\'eance $T$\\
		\item [$\quad \N(0,1) $] la loi gaussienne centr\'ee et r\'eduite\\
		\item [$\quad \N(\cdot) $] la fonction de r\'epartition de la gaussienne centr\'ee et r\'eduite\\
		\item [$\quad \phi(\cdot) $] la densit\'e de la gaussienne centr\'ee et r\'eduite\\	
	\end{description}
					
\subsection*{Hypoth\`ese}
 On supposera l'absence d'opportunit\'e d'arbitrage \footnote{cf. chapitre \ref{AOA} p. \pageref{AOA} dans l'Annexe} \\

%***********************************************
\chapter{La formule de Black, Scholes et Merton}
%***********************************************		

\section{Diffusion d'un actif financier}
		
\subsection*{Diffusion \guillemotleft \ historique \guillemotright }

	On suppose que tout processus de prix $\{S_t,\, t \in [0,T]\}$ est r\'egi par une \'equation diff\'erentielle 
stochastique de la forme :\\
	\begin{eqnarray*}
		\frac{\dif S_t}{S_t} &=& \mu_S(t) \dif t + \sigma_S(t) \dif \widehat{W}_t
	\end{eqnarray*}			
	o\`u \\
	\begin{itemize}
		\item Le terme de tendance $\mu_S(\cdot)$ est une fonction d\'eterministe.
		\item Le terme de volatilit\'e $\sigma_S(\cdot)$ est une fonction d\'eterministe.
		\item $\widehat{W}_t$ est un mouvement brownien sous la probabilit\'e historique $\P$.\\	
	\end{itemize}	
	\bigskip
	D'apr\`es le lemme d'It\^o\footnote{cf. proposition \ref{LemmeIto} p. \pageref{LemmeIto} dans l'Annexe.} on a :\\
		\begin{eqnarray*}
			S_t &=& S_0 \exp \left \lbrace \, \int_0^t (\mu_S(s)-\frac{1}{2}\sigma_S^2(s))\, \dif s 
			+ \int_0^t \sigma_S(s) \dif \widehat{W}_s \, \right \rbrace
		\end{eqnarray*}	
	\noindent De mani\`ere \'equivalente on a :
	\begin{eqnarray*}
		S_T &=& S_t \exp \left \lbrace \, \int_t^T (\mu_S(s)-\frac{1}{2}\sigma_S^2(s))\, \dif s 
		+ \int_t^T \sigma_S(s) \dif \widehat{W}_s \, \right \rbrace
	\end{eqnarray*}

\subsection*{Diffusion "risque-neutre"}
	\begin{prop} $\quad$\label{dif RN} 
		La diffusion de  $S(\cdot)$ sous la probabilit\'e risque-neutre $\Q$ est donn\'ee par :\\
		\begin{equation*}
			\boxed{\frac{dS_t}{S_t} \, = \, r_t \, \dif t + \sigma_S(t) \, \dif W_t}
		\end{equation*}		
	\end{prop}	
	D'apr\`es le lemme d'It\^o\footnote{cf. proposition \ref{LemmeIto} p. \pageref{LemmeIto} dans l'Annexe.}, on obtient :\\
		\begin{eqnarray*} \label{exp RN}
			S_t &=& S_0 \exp \left \lbrace \, \int_0^t (r_s-\frac{1}{2}\sigma_S^2(s)) \, \dif s +
			 \int_0^t \sigma_S(s) \, \dif W_s \, \right \rbrace \\
		\end{eqnarray*}	
	\\
	De mani\`ere \'equivalente, on a :\\
	\begin{eqnarray*}\label{exp RN bis}
		S_T = S_t \exp \left \{ \int_t^T (r_u - \frac 1 2 \sigma_S^2(u) )\, \dif u 
		+ \int_t^T \sigma_S(u) \, \dif W_u \right \}
	\end{eqnarray*}
	\begin{prop}[Loi du logarithme n\'ep\'erien $\frac {S_T}{S_t} $]\label{loi du log} $\quad$
		Sous la probabilit\'e risque-neutre, et en posant :\\
		\begin{itemize}
			\item	$R(t,T) = \frac 1 {T-t} \int_t^T r_u \, \dif u $\\
			\item	$\Sigma^2_S(t,T) = \frac 1 {T-t} \int_t^T \sigma_S^2(u)\, \dif u $\\
		\end{itemize}
		le logarithme du rapport $\frac {S_T}{S_t} $ suit une loi normale :\\
		\begin{itemize}
			\item	de variance : $\Sigma_S(t,T) \sqrt{T-t} $\\
			\item	de moyenne :  $(R(t,T) - \frac 1 2 \Sigma^2_S(t,T) ) (T-t) $\\
		\end{itemize}
	\end{prop}
	\begin{rem} $\quad$
		$R(t,T)$ et $\Sigma^2_S(t,T)$ repr\'esentent respectivement le taux sans rique moyen et la volatilit\'e moyenne de
		 l'actif $S$ entre $t$ et $T$.
	\end{rem}
\newpage
	\begin{rem}\label{exp avec U} $\quad$
		On peut \'ecrire que sous la probabilit\'e risque-neutre $\Q$ :
		\begin{eqnarray} 
			S_T = S_t e^{\left \{ \, R(t,T) - \frac 1 2 \Sigma_S^2(t,T)\, \right \} (T-t)
			 + \Sigma_S(t,T)\sqrt{T-t} U} 
		\end{eqnarray}
		o\`u la variable al\'eatoire $U$  suit une loi gaussienne centr\'ee et r\'eduite et est ind\'ependante 
		de $\F _t$ :
		\begin{center}	
			$\left\lbrace 
				\begin{array}{c} 
 					U \stackrel{\mathcal L}{\sim} \N(0,1)\\
					U \amalg \F_t 
				\end{array}
			\right. $
		\end{center}
	\end{rem}
	\bigskip
	\begin{dem}\label{DemExpU}
		\begin{itemize}
			\item Par construction de l'int\'egrale de Wiener. 
			\item Par propri\'et\'e d'ind\'ependance des accroissements du mouvement brownien par rapport \`a
 				la filtration canonique.
		\end{itemize}
		\begin{flushright}
			$\boxslash$
		\end{flushright}			
	\end{dem}		
	\begin{prop} $\quad$ 
		\begin{displaymath}
			\forall \, t \quad \mu_S(t) = 0 \, \iff \,  \{S_t, \, t\} \, \text{est une} 
			\quad \{\Omega ,\, \F, \, \P\} -martingale
		\end{displaymath}
	\end{prop}
	\begin{dem}
		\begin{itemize}
			\item En utilisant les \'el\'ements de la d\'emonstration \ref{DemExpU} p. \pageref{DemExpU}
		\end{itemize}
		\begin{flushright}
			$\boxslash$
		\end{flushright}
	\end{dem}
	\begin{rem}[Changement de probabilit\'e] $\quad$ 
		Un changement de probabilit\'e ne modifie que le terme de tendance (facteur de $ \dif t $).\\
		\indent La volatilit\'e est invariante par changement de mesure de probabilit\'e 
\footnote{cf. Annexe \ref{CalculSto} p. \pageref{CalculSto}}.
	\end{rem}

\section{Formule de Black-Scholes-Merton pour un call europ\'een}
		
	\begin{df} $\quad$	
	      \indent Un call europ\'een sur un sous-jacent $S$, d'\'ech\'eance $T$ et de strike $K$ est le droit d'acheter
 au prix $K$, fix\'e \`a l'avance, le sous-jacent $S$, \`a la date $T$.\\
	      \\
	      \indent Le flux \`a l'\'ech\'eance est donc : $\quad \phi(S_T) \, = \, (S_T - K)_+$	
	\end{df}
	\begin{df} $\quad$	
	      \indent Un put europ\'een sur un sous-jacent $S$, d'\'ech\'eance $T$ et de strike $K$ est le droit de vendre
 au prix $K$, fix\'e \`a l'avance, le sous-jacent $S$, \`a la date $T$.\\
	      \\
	      \indent Le flux \`a l'\'ech\'eance est donc : $\quad \phi(S_T) \, = \, (K-S_T)_+$	
	\end{df}
\newpage
	\begin{prop}[Probabilit\'e d'exercice du call] $\quad$
		\indent La probabilit\'e d'exercice d'un call est donn\'e par la formule : 
		\begin{equation}\label{ProbaExercice}
			\boxed {\Q \, \left  \lbrack \, S_T \geq K \,/ \, \F_t \, \right \rbrack = \N(d_2)}
		\end{equation}
		avec 
		\begin{displaymath}
			d_2 := \frac{\ln \{\frac{S_t}{K}\} + (R(t,T) - \frac 1 2 \Sigma_S^2(t,T)) (T-t) }{\Sigma_S(t,T)
				\sqrt{T-t}} 
		\end{displaymath}								
	\end{prop}
	\bigskip
	\begin{dem}
		\begin{eqnarray*}
			\Q \, \lbrack \, S_T \geq K /\F_t \, \rbrack 
				&=& \Q \, \left \lbrack \, S_t e^{(R(t,T) - \frac 1 2 \Sigma_S^2(t,T)) (T-t) +
				 \Sigma_S(t,T)\sqrt{T-t} U} \geq K /\F_t \, \right \rbrack \\
				&& \text{\tiny{d'apr\`es la remarque \ref{exp avec U} p. \pageref{exp avec U}.}} \\
				&=& \Q \, \left \lbrack \, (R(t,T) - \frac 1 2 \Sigma_S^2(t,T)) (T-t) +
				 \Sigma_S(t,T)\sqrt{T-t} \, U \geq  \ln \frac{K}{S_t} /\F_t \, \right \rbrack \\
				&=&	\Q \, \left \lbrack \, U \geq \frac{\ln \{\frac{K}{S_t}\} - (R(t,T) - 
				\frac 1 2 \Sigma_S^2(t,T)) (T-t) }{\Sigma_S(t,T)\sqrt{T-t}}/\F_t \, \right \rbrack \\
				&=& \Q \, \left \lbrack \, - U \leq \frac{\ln \{\frac{S_t}{K}\} + (R(t,T) - \frac 1 2 
				\Sigma_S^2(t,T)) (T-t) }{\Sigma_S(t,T)\sqrt{T-t}}/\F_t \, \right \rbrack \\
				&=& \Q \, \left \lbrack \, - U \leq \frac{\ln \{\frac{S_t}{K}\} + (R(t,T) - \frac 1 2 
				\Sigma_S^2(t,T)) (T-t) }{\Sigma_S(t,T)\sqrt{T-t}} \, \right \rbrack \\
				&& \text{\tiny {d'apr\`es le lemme d'ind\'ependance cf. proposition \ref{independance} 
				p. \pageref{independance} dans l'Annexe.}} \\
				&& \text{\tiny{$U$ \'etant ind\'ependant de $\F_t$ et $S_t$ \'etant 
				$\F_t$-mesurable.}} \\
				&=& \Q \, \left \lbrack \, U \leq \frac{\ln \{\frac{S_t}{K}\} + (R(t,T) - \frac 1 2 
				\Sigma_S^2(t,T)) (T-t) }{\Sigma_S(t,T)\sqrt{T-t}} \, \right \rbrack\\
				&&  \text{\tiny{par sym\'etrie de la loi gaussienne $U \stackrel{\mathcal L}{\sim} 
				\N(0,1) \Leftrightarrow  -U \stackrel{\mathcal L}{\sim} \N(0,1) $}}		
		\end{eqnarray*}
		\begin{flushright}
			$\boxslash$
		\end{flushright}
	\end{dem}
	\begin{prop}[La formule de Black-Scholes-Merton]$\quad$ \label{PrixCallBS}
		\indent Le prix \`a la date $t$ d'un call europ\'een d'\'ech\'eance $T$, et de strike $K$ est donn\'e par :\\
		\begin{equation*}
			\boxed { Call_t(T,K,S) = S_t \N(d_1) - K e^{ - \int_t^T r_u \, \dif u} \N(d_2) }
		\end{equation*}
		avec : 
		\begin{eqnarray*} \label{d1 et d2}
			d_1 &:=& \frac{\ln \{\frac{S_t}{K}\} + (R(t,T) + \frac 1 2 \Sigma_S^2(t,T)) (T-t) }
			{\Sigma_S(t,T)\sqrt{T-t}} \\
	 		d_2 &:=& \frac{\ln \{\frac{S_t}{K}\} + (R(t,T) - \frac 1 2 \Sigma_S^2(t,T)) (T-t) }
			{\Sigma_S(t,T)\sqrt{T-t}} \\
			    &=& d_1 - \Sigma_S(t,T)\sqrt{T-t} 
		\end{eqnarray*}
	\end{prop}
	\begin{dem}$\quad$\label{DemBSM}
		\indent	Le prix du call \`a la date $t<T$ est donn\'e par :
		\begin{eqnarray}
			C_t(S,T,K) &=& \E_\Q \, \left \lbrack \, \frac{\beta(t)}{\beta(T)} (S_T - K)\1 _{\{S_T \geq K \}}
			 \, / \, \F_t \, \right \rbrack \nonumber\\
		 	&=& \E_\Q \, \left \lbrack \, \frac{\beta(t)}{\beta(T)} S_T \1 _{\{S_T \geq K \}} 
			\, / \, \F_t \, \right \rbrack - \frac{\beta(t)}{\beta(T)} \, K \E_\Q \, \left \lbrack \, \1_{\{S_T \geq K \}} \, / \, \F_t \, \right \rbrack \nonumber\\
			&& \text{\tiny{par lin\'earit\'e de l'esp\'erance conditionnelle}} \nonumber \\
		     	&=& \E_{\Q} \left \lbrack \, \frac{\beta(t)}{\beta(T)} S_T \1 _{\{S_T \geq K \}} 
			\, / \, \F_t \, \right \rbrack \, - \, \frac{\beta(t)}{\beta(T)} \, K \,\Q \, \left \lbrack 
			\, S_T \geq K \, / \, \F_t \, \right \rbrack \nonumber\\
		\end{eqnarray}
		\indent Il reste \`a d\'eterminer le premier membre : 
		\begin{eqnarray*}
			\E_{\Q} \left \lbrack \, \frac{\beta(t)}{\beta(T)} S_T \1_{\{S_T \geq K \}} \, / \, 
			\F_t \, \right \rbrack  
			&=& \E_{\Q}\, \left \lbrack\, \frac{\beta(t)}{\beta(T)}.S_t e^{(R(t,T) - \frac 1 2 
			\Sigma_S^2(t,T)) (T-t) + \Sigma_S(t,T)\sqrt{T-t}.U} \1_{\{S_T \geq K \}} /\F_t \, \right \rbrack \\
			&& \text{\tiny{d'apr\`es la remarque \ref{exp avec U} p. \pageref{exp avec U}.}} \\
			&=& S_t \,\E_{\Q}\, \left \lbrack\,e^{- \frac 1 2 \Sigma_S^2(t,T) (T-t) + \Sigma_S(t,T)\sqrt{T-t}
			\,U} \1_{\{S_T \geq K \}} /\F_t \, \right \rbrack \\
			&=& S_t \, \int_{\{S_T \geq K \}}e^{- \frac 1 2 \Sigma_S^2(t,T) (T-t) + \Sigma_S(t,T)\sqrt{T-t} U}
			 \mathrm d\Q \\
			&& \text{\tiny {d'apr\`es le lemme d'ind\'ependance cf. proposition \ref{independance} p. 
			\pageref{independance} dans l'Annexe.}} \\
			&=& S_t \, \int_{-d_2}^{\infty}e^{- \frac 1 2 \Sigma_S^2(t,T) (T-t) + \Sigma_S(t,T)\sqrt{T-t} U} 
			\frac 1 {\sqrt{2 \pi }}e^{\frac{-u^2}{2}}\, \dif u \\
			&& \text{\tiny{car $\Q$ est absolument continue par rapport \`a la mesure de Lebesgues et admet une 
			densit\'e.}} \\
			&=& S_t \, \int_{-d_2}^{\infty}e^{\frac{-u^2}{2}- \frac 1 2 \Sigma_S^2(t,T) (T-t) + \Sigma_S(t,T)
			\sqrt{T-t} U} \frac 1 {\sqrt{2 \pi }}\, \dif u \\
			&=& S_t \, \int_{-d_2}^{\infty}e^{-\frac{u^2 - 2 \Sigma_S(t,T) \sqrt{T-t} + \Sigma_S^2(t,T) (T-t)}
			{2}}\frac 1 {\sqrt{2 \pi }}\, \dif u \\
			&=& S_t \, \int_{-d_2}^{\infty}e^{-\frac{(u - \Sigma_S(t,T) \sqrt{T-t} )^2}{2}}\frac 1 
			{\sqrt{2 \pi }}\, \dif u \\
			&=& S_t \, \int_{-d_2 - \Sigma_S(t,T) \sqrt{T-t}}^{\infty}e^{\frac{-z^2}{2}}\frac 1 
			{\sqrt{2 \pi }}\, \dif z \\
			&& \text{\tiny{par changement de variable : $ z = u - \Sigma_S(t,T) \sqrt{T-t} $}} \\
			&=& S_t \N(d_1) \\
			&& \text{\tiny{en posant $\, d_1 = d_2 + \Sigma_S(t,T) \sqrt{T-t} \, $}} \\
			&& \text{\tiny{en utilisant la sym\'etrie de la loi gaussienne centr\'ee r\'eduite}}   
		\end{eqnarray*} 
		\begin{flushright}
			$\boxslash$
		\end{flushright}
	\end{dem}
	\begin{rem}$\quad$
		\indent Lorsque les taux d'int\'er\^et et la volatilit\'e sont constants, on a : 
			\begin{equation*}
				d_2 = \frac{\ln \{ \frac{S_t}{K} \} + (r - \frac 1 2 \sigma_S^2) (T-t)}{\sigma_S \sqrt{T-t}}
			\end{equation*}
	\end{rem}
	\begin{prop}[Formule de Black-Scholes-Merton pour un put]
		\begin{equation}
			\boxed{\, Put_t(T,K,S) \, = \, -S_t \,\N(-d_1) + K e^{ - \int_t^T r_u \, \dif u} \, \N(-d_2) }			
		\end{equation}
	\end{prop}
	\begin{dem}[Parit\'e Call-Put]$\quad$
		D'apr\`es la remarque \ref{call-put} p. \pageref{call-put} et la proposition \ref{PrixCallBS} p.
 \pageref{PrixCallBS} on a : 
		\begin{eqnarray*} 
			Put_t(T,K,S) &=& Call_t(T,K,S) - S_t + K \, B(t,T) \\
			&=& S_t \N(d_1) - K e^{ - \int_t^T r_u \, \dif u} \N(d_2) - S_t + K e^{ - \int_t^T r_u \, \dif u} \\
			&=& - S_t ( 1 - \N(d_1) ) + K e^{ - \int_t^T r_u \, \dif u} ( 1 - \N(d_2)) \\
			&& \text{\tiny{par sym\'etrie de la loi normale}} \\
			&=& -S_t \,\N(-d_1) + K e^{ - \int_t^T r_u \, \dif u} \, \N(-d_2) 
		\end{eqnarray*} 
		\begin{flushright}
			$\boxslash$
		\end{flushright}		
	\end{dem}

\section{Les \guillemotleft \ grecques \guillemotright }

\subsection{Le Delta}

	\begin{df}[Delta d'une option]$\quad$
	Le delta d'une option est la sensibilit\'e du prix de l'option par rapport au prix sous-jacent
	\end{df}
	\begin{rem}[Formulation math\'ematique]$\quad$
	Math\'ematiquement, le Delta se calcule en d\'erivant le prix de l'option par rapport au sous-jacent
	\end{rem}
	\begin{prop}[Delta d'un call europ\'een]$\quad$ \label{DeltaCall}
	Le delta d'un call europ\'een est donn\'e par :
		\begin{equation*}
			\boxed{\Delta^{Call_t(T,K,S)} = \N(d_1)}
		\end{equation*}
	\end{prop}

	\begin{prop}[R\'esultat utile] \label{res util} $\quad$
		On d\'esigne par $\phi(x) = \frac 1 {\sqrt{2\pi}} e^{-\frac {x^2} 2} $ , la fonction de densit\'e de la gaussienne
centr\'ee et r\'eduite. On a :
		\begin{equation*}
			\boxed{\phi(d_2) = \frac {S_t} K e^{-R(t,T) (T-t)} \phi(d_1)}
		\end{equation*}
	\end{prop}
	\begin{dem}$\quad$
	En effet : 
		\begin{eqnarray*}
				\phi(d_2) &=& \phi(d_1 - \Sigma_S(t,T) \, \sqrt{T-t}) \\
				&=& \frac 1 {\sqrt{2 \pi}} \exp{\left \{ -\frac 1 2 (d_1 - \Sigma_S(t,T) \, \sqrt{T-t})^2
					 \right \}} \\
				&=& \frac 1 {\sqrt{2 \pi}} \exp{\left \{  -\frac 1 2 ( d_1^2 - 2 d_1 \Sigma_S(t,T) \, 
				\sqrt{T-t} + \Sigma^2_S(t,T) \, (T-t)\right \}} \\
				&=& \phi(d_2) \,\exp{\left \{ d_1 \Sigma_S(t,T) \, \sqrt{T-t} - \frac 1 2 \Sigma^2_S(t,T) \, (T-t)
					\right \}} \\
				&=& \phi(d_1) \exp{\left \{ \ln \{\frac{S_t}{K}\} + (R(t,T) + \frac 1 2 \Sigma_S^2(t,T)) 
				(T-t) - \frac 1 2 \Sigma^2_S(t,T) \, (T-t) \right \}} \\
				&=& \phi(d_1) \frac{S_t}{K} e^{R(t,T)(T-t)}\\
			\end{eqnarray*}
		\begin{flushright}
			$\boxslash$
		\end{flushright}
	\end{dem}
	\begin{dem}[Delta d'un call europ\'een]\label{demDelta}$\quad$
		\begin{itemize}
			\item	Par d\'efinition, le Delta du call europ\'een s'\'ecrit :
			\begin{eqnarray*}
				\Delta^{Call_t(T,K,S)} &:=& \frac{\partial Call_t(T,K,S)}{\partial S_t}\\
				&=& \N(d_1) + S_t \frac{\partial \N(d_1)}{\partial S_t} -
				 Ke^{-R(t,T)(T-t)} \frac{\partial \N(d_2)}{\partial S_t} \\
				&=& \N(d_1) + \phi(d1) \frac{\partial d_1}{\partial S_t} -  
				 Ke^{-R(t,T)(T-t)} \phi(d_2) \frac{\partial d_2}{\partial S_t} \\
			\end{eqnarray*}
	o\`u $\phi(\cdot)$ d\'esigne la densit\'e de la gaussienne centr\'ee et r\'eduite \\
			\item	On a :
			\begin{eqnarray*}
				\frac{\partial d_1}{\partial S_t} &=& \frac{\partial d_2}{\partial S_t}\\
				&=& \frac 1 {S_t\, \Sigma_S(t,T) \, \sqrt{T-t}}\\
			\end{eqnarray*}
			\item	D'autre part, par la proposition \ref{res util} p. \pageref{res util} :
			\begin{eqnarray*}
				\phi(d_2)&=& \phi(d_1) \frac{S_t}{K} e^{R(t,T)(T-t)}\\
			\end{eqnarray*}
			\item	Ainsi :
			\begin{eqnarray*}
				\Delta^{Call_t(T,K,S)} &=& \N(d_1) \, + \, S_t \phi(d1) \frac{\partial d_1}{\partial S_t} \, 
				- \,  Ke^{-R(t,T)(T-t)} \frac{\partial d_2}{\partial S_t} \\
				&=& \N(d_1) \\
				&&  + \, S_t  \phi(d1) \, \frac 1 {S_t\, \Sigma_S(t,T) \, \sqrt{T-t}} \\ 
				&& - \, Ke^{-R(t,T)(T-t)} \,  \frac 1 {S_t\, \Sigma_S(t,T) \, \sqrt{T-t}} \,
				 \phi(d_1) \frac{S_t}{K} e^{R(t,T)(T-t)}\\
				&=& \N(d_1) \, + \, \phi(d_1) \left \lbrack  \frac 1 {\Sigma_S(t,T) \, \sqrt{T-t}} \, - 
				\, \frac 1 {\Sigma_S(t,T) \, \sqrt{T-t}} \right \rbrack \\
				&=& \N(d_1) \\
			\end{eqnarray*}
			
		\end{itemize}
		\begin{flushright}
			$\boxslash$
		\end{flushright}
	\end{dem}

\subsection{Le Gamma}

	\begin{df}[Gamma d'une option] $\quad$
	Le Gamma d'une option est la sensibilit\'e du Delta de l'option par rapport au sous-jacent.
	\end{df}
	\begin{rem}[Formulation Math\'ematique]$\quad$
	Il s'agit de calculer :
		\begin{eqnarray*}
			\Gamma^{Call_t(T,K,S)} &:=& \frac {\partial \Delta^{Call_t(T,K,S)}}{\partial S_t} \\
		\end{eqnarray*}
	\end{rem}
	\begin{prop}[Gamma d'un call europ\'een]$\quad$ \label{GammaCall}
	Le Gamma d'un call europ\'een est donn\'e par :
		\begin{equation*}
			\boxed{\Gamma^{Call_t(T,K,S)} \, = \, \frac {\phi(d_1)} {S_t\, \Sigma_S(t,T) \, \sqrt{T-t}}}
		\end{equation*}		
	\end{prop}
	\begin{dem}$\quad$
	Il suffit d'utiliser la formulation math\'ematique
		\begin{flushright}
			$\boxslash$
		\end{flushright}
	\end{dem}	

\subsection{Le Vega}

	\begin{df}[Vega d'une option]$\quad$
	Le Vega d'une option est la sensibilit\'e du prix de l'option par rapport \`a la volatilit\'e. 
	\end{df}
	\begin{prop}[Vega d'un call europ\'een] \label{VegaCall} $\quad$
	Le Vega d'un call europ\'een est donn\'e par :
		\begin{equation*}
			\boxed{Vega^{Call_t(T,K,S)} \, =\, S_t \, \sqrt{T-t} \, \phi(d_1) }
		\end{equation*}
	\end{prop}
	\begin{dem}$\quad$
		\begin{itemize}
			\item	On a :
			\begin{eqnarray*}
				Vega^{Call_t(T,K,S)} &:=& \frac{\partial Call_t(T,K,S) }{\partial \Sigma_S(t,T)} \\
				&=& S_t \phi(d_1) \frac{\partial d_1}{\partial \Sigma_S(t,T)} 
				- K e^{-R(t,T)(T-t)} \phi(d_2) \frac{\partial d_2}{\partial \Sigma_S(t,T)} \\
			\end{eqnarray*}
			\item 	De plus :
			\begin{eqnarray*}
				\frac{\partial d_2}{\partial \Sigma_S(t,T)} &=& \frac{\partial d_1}{\partial \Sigma_S(t,T)}
						- \sqrt{T-t}\\
			\end{eqnarray*}
			\item	Et , par la proposition \ref{res util} p. \pageref{res util} :
			\begin{eqnarray*}
				\phi(d_2)&=& \phi(d_1) \frac{S_t}{K} e^{R(t,T)(T-t)}\\
			\end{eqnarray*}
			\item	D'o\`u :
			\begin{eqnarray*}
				Vega^{Call_t(T,K,S)} &=& S_t \phi(d_1) \frac{\partial d_1}{\partial \Sigma_S(t,T)} 
				- K e^{-R(t,T)(T-t)}  \phi(d_1) \frac{S_t}{K} e^{R(t,T)(T-t)} (\frac{\partial d_1}{\partial \Sigma_S(t,T)}
				- \sqrt{T-t})\\
				&=& S_t \, \sqrt{T-t} \, \phi(d_1) \\
			\end{eqnarray*}
		\end{itemize}
		\begin{flushright}
			$\boxslash$
		\end{flushright}
	\end{dem}

\subsection{Le Rho}

	\begin{df}[Rho d'une option]$\quad$
	Le Rho d'une option est la sensibilit\'e du prix de l'option par rapport au taux d'int\'er\^et.
	\end{df}
	\begin{prop}[Rho d'un call europ\'een] \label{RhoCall} $\quad$
	Le Rho d'un call europ\'een est donn\'e par :
		\begin{equation*}
			\boxed{\rho^{Call_t(T,K,S)} \, = \, (T-t) K e^{-R(t,T)(T-t)} \N(d_2)}
		\end{equation*}
	\end{prop}
	\begin{dem}$\quad$
		On a :
		\begin{eqnarray*}
			\rho^{Call_t(T,K,S)} &:=& \frac{\partial Call_t(T,K,S)} {\partial R(t,T)} \\
			&=&  S_t \phi(d_1) \frac{\partial d_1}{\partial R(t,T)} 
				- K e^{-R(t,T)(T-t)} \phi(d_2) \frac{\partial d_2}{\partial R(t,T)} 
				+ (T-t)  K e^{-R(t,T)(T-t)} \N(d_2) \\
			&=& (T-t) K e^{-R(t,T)(T-t)} \N(d_2) \\
		\end{eqnarray*}
		\begin{flushright}
			$\boxslash$
		\end{flushright}
	\end{dem}

%**********************************************************************************************
\chapter{Valorisation d'une option  exotique : \\ \guillemotleft \ le put gap \guillemotright } 
%**********************************************************************************************

\section{D\'efinition}

 	\begin{df}[Put gap europ\'een] $\quad$
		Un put \guillemotleft \ gap \guillemotright \ est un put dont le strike $K$ 
		(dont d\'epend l'exercice ou non de l'option \`a l'\'ech\'eance) 
		est diff\'erent du montant $L$ utilis\'e pour calculer son payoff. \\
	\\
	Le payoff pour un put gap est :\\
	\\
		\begin{center}	
			$\left\lbrace 
				\begin{array}{ccc} 
 					L - S_T  & \text{si} \quad S_T < K &\\
					0  & \text{sinon}& 
				\end{array}
			\right. $
		\newline
		\end{center}
		C'est-\`a dire le flux peux s'\'ecrire : $( L -S_T ) \, \1_{\{S_T < K\}}$	
	\end{df}

\section{Evaluation}
	\begin{prop}\label{BS gap} $\quad$
		Le prix d'un put gap est donn\'e par : 
		\begin{equation*}
			\boxed{ L \, e^{-\int_t^T r_s \, \dif s } \, \N(-d_2) - S_t \, \N(-d_1) }
		\end{equation*}  
		avec :  
		\begin{eqnarray*}
			d_1 &=& \frac{\ln \{\frac{S_t}{K}\} + (R(t,T) + \frac 1 2 \Sigma_S^2(t,T)) (T-t) }
				{\Sigma_S(t,T)\sqrt{T-t}} \\
	 		d_2 &=& \frac{\ln \{\frac{S_t}{K}\} + (R(t,T) - \frac 1 2 \Sigma_S^2(t,T)) (T-t) }
				{\Sigma_S(t,T)\sqrt{T-t}} \\
			    &=& d_1 - \Sigma_S(t,T)\sqrt{T-t} 
		\end{eqnarray*}
	\end{prop}
\newpage
	\begin{dem}
		\begin{itemize}
			\item	Utiliser la probabilit\'e d'exercice d'un call vanilla \footnote{cf. 
				Proposition \ref{ProbaExercice} p. \pageref{ProbaExercice}.}.
			\item Utiliser la d\'emonstration \ref{DemBSM} p. \pageref{DemBSM}. 
		\end{itemize}
		\begin{flushright}
			$\boxslash$
		\end{flushright}
	\end{dem}

%****************************
\chapter{March\'e \'etranger}
%****************************
		
\section{Coexistance de deux risques}
	Lorsque l'on consid\`ere des options sur un  sous-jacent exprim\'e en monnaie \'etrang\`ere, on est simultan\'ement 
soumis \`a deux risques : 
	\begin{itemize}
		\item	le risque li\'e aux fluctuations du prix du sous-jacent\\
		\item	le risque li\'e aux fluctuations du taux de change \\
	\end{itemize}  
	\begin{ex} $\quad$
		Typiquement, une option sur le p\'etrole (exprim\'e en \$) avec strike ou nominal en \officialeuro.	
	\end{ex}	
	\begin{rem} $\quad$
		La coexistance de ces deux risques induit naturellement une corr\'elation dont il faut tenir 
		compte pour valoriser ce type de produit.
	\end{rem}
	\begin{rem}
		$\quad$ Il est \'egalement n\'ecessaire de consid\'erer les param\`etres du march\'e \'etranger.\\
	\end{rem}

\subsection*{Param\`etres du march\'e \'etranger}
	
	On notera : \\
	\begin{itemize}
		\item[$\bullet$] $\{ \, r_t^f , t \in \lbrack 0 , T \rbrack \, \} $, le processus (suppos\'e d\'eterministe) du taux 
			d'int\'er\^et sur le march\'e \'etranger \\
		\item[$\bullet$]	$\{ \, \beta^f(t) = e^{\int_0^t r_s ds } , t \in \lbrack 0 , T \rbrack \, \} $, le facteur 
			d'accumulation associ\'e \\
		\item[$\bullet$]	$\{ \, B^f(t,T) = e^{- \int_t^T r_s ds } , t \in \lbrack 0 , T \rbrack \, \} $, les z\'ero-coupons 
			associ\'es \\
		\item[$\bullet$]	$\Q^f $, la mesure de probabilit\'e risque-neutre \'etrang\`ere, i.e. associ\'ee au num\'eraire 
			du facteur d'accumulation \'etranger \\
	\end{itemize}	
	\begin{prop}[Changement de probabilit\'e]
		Pour tout flux exprim\'e en monnaie \'etrang\`ere $\phi^f$, on a la formule d'\'evaluation 
		\footnote{par changement de num\'eraire cf. Annexe \ref{Numeraire} p. \pageref{Numeraire}}:\\
		\begin{equation}\label{chgt dom_etr}
			\boxed{ \,  \E_{\Q}\,\left \lbrack  \, \frac{\beta(t)}{\beta(T)} X_T \phi^f(T) \, / \, 
			\F_t \, \right \rbrack \, = \,  X_t \,  \E_{\Q^f}\,\left \lbrack \, \frac{\beta^f(t)}{\beta^f(T)} 
			\phi^f(T) \, / \, \F_t \, \right \rbrack \, }
		\end{equation}
	\end{prop}

\section{Processus du taux de change}
	
	On note $\{ \, X_t, \, t \in \lbrack 0 , T \rbrack \, \}$, le processus du taux de change, i.e  \`a la date $t$ : \\ 
	\begin{center}
		\framebox{ $ 1 \, \$ = X_t $ \officialeuro } 
	\end{center}
	
\subsection*{Diffusion \guillemotleft \ historique \guillemotright }
	
	On place sur ce processus une diffusion classique :
	\begin{eqnarray*}
		\frac {dX_t}{X_t} = \mu_X(t) \, \dif t + \sigma_X(t) \, \dif \widehat{W}^x_t
	\end{eqnarray*}
	o\`u $\{ \widehat{W}^x_t , t \} $ est un  $\{\Omega ,\, \F,\,  \P\}$-mouvement brownien.
	
\subsection*{Diffusion risque-neutre domicile}
	
	\begin{prop}
		Sous la probabilit\'e risque-neutre domicile, la diffusion de $ \{ \, X_t , \, t \in \lbrack 0 , T \rbrack 
		\, \} $ est : \\
		\begin{equation}
			\boxed{\, \frac{\, \dif X_t}{X_t} \, = \,  ( r_t - r_t^f ) \, \dif t + \sigma_X(t) \, 
			\dif W^x_t  \, } \\
		\end{equation}
		avec $\{\, W^x_t , t \,\} $ un $\{\Omega ,\, \F, \, \Q \}$-mouvement brownien.
	\end{prop}
	\begin{dem}
		Supposons que la diffusion du processus du taux de change sous $\Q$ soit donn\'ee par :
		\begin{eqnarray*}
			\frac {dX_t}{X_t} = \mu'_X(t) \, \dif t + \sigma_X(t) \, \dif W^x_t
		\end{eqnarray*}
		o\`u $\{ W^x_t , t \} $ est un  $\{\Omega ,\, \F,\,  \Q\}$-mouvement brownien.\\
		\\
		On sait d'ores et d\'ej\`a que la volatilit\'e est invariante par changement de probabilit\'e, il suffit de 
d\'eterminer le terme de tendance $\mu'_X(t) $ sous la probabilit\'e risque-neutre domicile. \\
		\\
		\indent	Notons respectivement $ A(t) = \exp{ \{ - \int_0^t r_s \dif s  \} } $ et $ A^f(t) = \exp{ \{ - \int_0^t r^f_s \dif s  \} } $ , les facteurs 
d'actualisation domestique et \'etranger. \\
		\\
		\indent On consid\`ere le processus $  Y_t : = \frac{X_t}{A^f(t)} A(t) $.\\
$ Y(t) $  est le prix en $t$ d'un titre \'etranger (sans risque) $ \frac 1 {A^f(t)}  $ , \'echang\'e  en monnaie domestique  
$ \frac 1 {A^f(t)} X_t $ , et actualis\'e au taux d'int\'er\^et sans risque domestique  $ \frac {X_t}{A^f(t)} A(t)  $
\newpage
	\noindent	$\Rightarrow $ c'est une $\{\Omega ,\F, \Q\}$ - martingale \\
	\noindent	$\Rightarrow $ son terme de tendance est nul sous $\Q$ \\
		et, on a :
		\begin{eqnarray*}
			dY_t &=& \, \dif  \left \lbrack \, X_t \frac{A(t)}{A^f(t)} \, \right \rbrack \\
			     &=& \frac{A(t)}{A^f(t)} \, \dif X_t + X_t \, \dif \left ( \frac{A(t)}{A^f(t)} \right ) \\
		             &=& \frac{A(t)}{A^f(t)} X_t \left \{ \mu'_X(t) \, \dif t + \sigma_X(t) \, \dif W^x_t \right \} 
				+ X_t ( - r_t + r_t^f ) \frac{A(t)}{A^f(t)}\, \dif t \\
		            &=&  \frac{A(t)}{A^f(t)} X_t \left \{ (\mu'_X(t) - r_t + r_t^f )\, \dif t  + \sigma_X(t) \, 
				\dif W^x_t \right \}\\
		\end{eqnarray*}
		\begin{eqnarray*}
			\Rightarrow  &\frac{\, \dif Y_t}{Y_t} = (\mu'_X(t) - r_t + r_t^f ) \, \dif t  + \sigma_X(t) \, 
				\dif W^x_t &\\
			\Rightarrow  &\mu'_X(t) - r_t + r_t^f = 0 &\\
			\Rightarrow  &\mu'_X(t) = r_t - r_t^f &	\\
		\end{eqnarray*}	
		\begin{flushright}
			$\boxslash$
		\end{flushright}
	\end{dem}

\section{Diffusion du sous-jacent \'etranger}

	Soit $\{ \, S^f_t , t \in \lbrack 0 , T \rbrack \, \} $ , le processus de prix d'un sous-jacent exprim\'e en 
monnaie \'etrang\`ere.\\
	\begin{rem}$\quad$
		Le titre $ S^d_t  = S^f_t X_t $ est un titre domestique, c'est la valeur en monnaie domestique 
		d'un titre \'etranger.
	\end{rem}

\subsection*{Diffusion \guillemotleft \ historique \guillemotright }

	Comme d'habitude, on place une diffusion de la forme :\\
	\begin{eqnarray*}
		\frac{dS_t^f}{S_t^f} = \mu_{S^f}(t) \, \dif t + \sigma_{S^f}(t) \, \dif \widehat{W}^{S^f}_t\\
	\end{eqnarray*}
	avec $\{ \, \widehat{W}^{S^f}_t , t \,\} $ un $\{\Omega ,\, \F,\,  \P\}$-mouvement brownien.

\subsection*{Diffusion risque-neutre domicile}

	\begin{prop}$\quad$
		Sous la probabilit\'e risque-neutre domicile, la diffusion de $ \{ \, S_t^f , \, t \in \lbrack 0 , 
		T \rbrack \, \} $ est : \\
		\begin{equation}
			\boxed{ \, \frac{\, \dif S_t^f}{S_t^f} \, = \, \left \{ \, r_t  - \rho_t(X,S^f) \sigma_X(t)  
			\sigma_{S^f}(t) \, \right \} \, \dif t +
			\sigma_{S^f}(t) \, \dif W^{S^f}_t \,}
		\end{equation}
		avec $\{ \, W^{S^f}_t ,\,  t \,\} $ un $\{ \Omega ,\,\F, \,\Q \} $ -mouvement brownien.
	\end{prop}
	\begin{dem}$\quad$
		\begin{itemize}
			\item	On suppose que la diffusion du sous-jacent \'etranger sous $\Q$ est :
 				\begin{eqnarray*}
					\frac{dS_t^f}{S_t^f} = \mu'_{S^f}(t) \, \dif t + \sigma_{S^f}(t) \, \dif W^{S^f}_t\\
				\end{eqnarray*}
				avec $\{ \, W^{S^f}_t , t \,\} $ un $\{\Omega ,\, \F,\,  \Q\}$-mouvement brownien.\\
			\item	On consid\`ere le processus $\{\, Z_t : = S^f_t X_t A(t) ,\, t \in \lbrack 0 , 
				T \rbrack \,\} $.\\
$Z_t$ est le prix du sous-jacent \'etranger $S_t^f$, \'echang\'e en monnaie domestique $ S_t^f X_t $, et 
actualis\'e au taux  risque neutre domestique  $ S_t^f X_t  A(t) $ \\
\noindent $ \Rightarrow $ c'est une $\{\Omega ,\, \F, \, \Q\}$ - martingale \\
\noindent $ \Rightarrow $ son terme de tendance est nul sous $\Q$\\
\\
Or , 
				\begin{eqnarray*}
					\, \dif Z_t &=& \, \dif \left \lbrack \, S^f_t X_t A(t)  \, \right \rbrack \\
					&=& A(t) \, \dif S^f_t X_t + S^f_t X_t \, \dif A(t)\\
				\end{eqnarray*}

			\item	{\bf Prise en compte de la corr\'elation entre taux de change et sous-jacent :}\\
La formule de It\^o \footnote{cf. proposition \ref{LemmeIto} p. \pageref{LemmeIto} dans l'Annexe.}
 appliqu\'ee avec la fonction $ f(x,y) = x y $ nous donne :
				\begin{eqnarray*}
					\, \dif S^f_t X_t 	&=& X_t \, \dif S_t^f + S_t^f \, \dif X_t  + \frac 1 2 
					\C_t ( \, \dif S^f_t , \, \dif X_t ) + \frac 1 2 \C_t (\, \dif X_t , \, \dif S^f_t )\\
					&=& X_t \, \dif S_t^f + S_t^f \, \dif X_t + \C_t ( \, \dif S^f_t , \, \dif X_t )\\
					&=& S_t^f X_t \left \lbrack \, \frac{\, \dif S_t^f}{S_t^f} + \frac{\, 
					\dif X_t}{X_t} + \frac{1}{S_t^f X_t }  \C_t ( \, \dif S^f_t , \, \dif X_t ) \, \right \rbrack \\
					\frac{\, \dif S^f_t X_t}{S_t^f X_t} &=& \frac{\, \dif S_t^f}{S_t^f} + \frac{\, \dif X_t}{X_t} +  
					\C_t ( \frac{\, \dif S_t^f}{S_t^f} , \frac{\, \dif X_t}{X_t} )
				\end{eqnarray*}

Et, on a :
				\begin{eqnarray*}
					\C_t ( \frac{\, \dif S_t^f}{S_t^f} , \frac{\, \dif X_t}{X_t} ) 
								&=& \rho_t(X,S^f) \sigma_X(t)  \sigma_{S^f}(t) \, \dif t \\
				\end{eqnarray*}
o\`u $ \rho_t(X,S^f) $ est le coefficient de corr\'elation des logarithmes de $ X_t $ et $S_t^f $ . \\

			\item	La  diffusion de $Z_t$ sous $\Q$ est alors donn\'ee par :
				\begin{eqnarray*}
					dZ_t &=&  A(t) S^f_t X_t \frac{\, \dif S^f_t X_t}{S_t^f X_t} +  A(t) S^f_t X_t 
							( - r_t) \, \dif t \\
					\frac{dZ_t}{Z_t} &=& \frac{dS_t^f}{S_t^f} + \frac{dX_t}{X_t} + \rho_t(X,S^f) 
					\sigma_X(t)  \sigma_{S^f}(t )\, \dif t - r_t \, \dif t \\
					&=& \left ( \, \mu'_{S^f}(t) \, \dif t + \sigma_{S^f}(t) \, \dif W^{S^f}_t \, 
					\right ) +  \left (\,\mu'_X(t) \, \dif t + \sigma_X(t) \, \dif W^x_t) \, \right ) +  
					\rho_t(X,S^f) \sigma_X(t)  \sigma_{S^f}(t) dt - r_t \, \dif t \\
				\end{eqnarray*}

			\item	En ne s'int\'eressant qu'au terme de tendance, on voit que :
				\begin{eqnarray*}
					\mu'_{S^f}(t) + \mu'_X(t) + \rho_t(X,S^f) \sigma_X(t)  \sigma_{S^f}(t)  - r_t = 0 \\
					\end{eqnarray*}
\noindent $\Rightarrow \mu'_{S^f}(t) = - \mu'_X(t)  - \rho_t(X,S^f) \sigma_X(t)  \sigma_{S^f}(t)  + r_t $ \\
\noindent $\Rightarrow \mu'_{S^f}(t) = r_t  - \rho_t(X,S^f) \sigma_X(t)  \sigma_{S^f}(t) $ \\
		\end{itemize}
	\begin{flushright}
			$\boxslash$
		\end{flushright}
	\end{dem}
	\begin{rem}$\quad$
		Comme pr\'ec\'edemment, en posant :\\
		\begin{itemize}
			\item	$\Gamma_X^{S^f}(t,T) = \frac 1 {T - t} \int_t^T (\rho_u(X,S^f) \sigma_X(u) 
				 \sigma_{S^f}(u)) \, \dif u $\\
			\item	$\Sigma^2_{S^f}(t,T) = \frac 1 {T - t} \int_t^T \sigma_{S^f}^2(u) \, \dif u $ \\
			\item	$R^f(t,T) = \frac 1 {T - t} \int_t^T r_u \, \dif u $ \\
		\end{itemize} 
	le logarithme du rapport $ \frac{S^f_t}{S^f_T} $ suit une loi gaussienne : \\
		\begin{itemize}
			\item	de variance : $\Sigma_{S^f}(t,T) \sqrt{T-t}$\\
			\item	de moyenne : $(R^f(t,T)- \Gamma_X^{S^f}(t,T)-\frac 1 2 \Sigma^2_{S^f}(t,T) ) (T-t) $ \\
		\end{itemize}
\indent Et on peut \'ecrire que sous la probabilit\'e risque-neutre domicile $\Q$ :
		\begin{eqnarray}
		S_T^f = S_t^f \exp{ \left \{ \, (R^f(t,T)- \Gamma_X^{S^f}(t,T)-\frac 1 2 \Sigma^2_{S^f}(t,T) ) (T-t) + 
			\Sigma_{S^f}(t,T) \sqrt{T-t} \, U \, \right \}}
		\end{eqnarray}	
o\`u la variable al\'eatoire $U$  suit une loi gaussienne centr\'ee et r\'eduite et est ind\'ependante de $\F _t$:
		\begin{center}	
			$\left\lbrace 
				\begin{array}{c} 
 					U \stackrel{\mathcal L}{\sim} \N(0,1)\\
					U \amalg \F_t 
				\end{array}
			\right. $
		\end{center}
	\end{rem}
	
\section{Valorisation d'un call quanto sans garantie de change}
				
	\indent Par la proposition \ref{chgt dom_etr} p. \pageref{chgt dom_etr}, on voit que lorsque le taux de change 
appliqu\'e \`a l'\'ech\'eance est celui en cours, il n'y a pas de risque li\'e au changement de devise. 
Il suffit alors d'\'evaluer l'option sur le march\'e \'etranger puis de convertir 
en monnaie domestique.

\subsection*{Formule de Black-Scholes-Merton pour un call quanto}

	Soit $Call_t(T,K^f,S^f)$ le prix en $t$ d'un call domestique sur sous-jacent \'etranger $S^f$, de strike $K^f$ 
exprim\'e en monnaie \'etrang\`ere et d'\'ech\'eance $T$.\\
\indent	Si on note $Call^f_t(T,K^f,S^f)$ le prix d'un call \'etranger sur le sous-jacent $S^f$ de strike $K^f$ et d'\'ech\'eance $T$ , 
avec ce qui pr\'ec\`ede on peut \'ecrire que :
	\begin{eqnarray}
		Call_t(T,K^f,S^f) = X_t Call^f_t(T,K^f,S^f)
	\end{eqnarray} 
	
	\begin{prop}[Prix d'un call quanto] \label{PrixQuanto} $\quad$
		Le prix d'un call quanto \footnote{sans fixation du taux de change} est :
		\begin{eqnarray*}
			\boxed{Call_t(T,K^f,S^f) = X_t \left \{ \, S^f_t \N(d_1^f) - K^f e^{ - \int_t^T r^f_u du} \N(d^f_2) \, \right \}}
		\end{eqnarray*}
	
	avec : 
		\begin{eqnarray*}
			d^f_1 &=& \frac{\ln \{\frac{S^f_t}{K^f}\} + (R^f(t,T) + \frac 1 2 \Sigma_{S^f}^2(t,T)) (T-t) }{\Sigma_{S}^f(t,T)\sqrt{T-t}} \\
	 		d^f_2 &=& \frac{\ln \{\frac{S^f_t}{K^f}\} + (R^f(t,T) - \frac 1 2 \Sigma_{S^f}^2(t,T)) (T-t) }{\Sigma_{S^f}(t,T)\sqrt{T-t}} \\
				&=& d^f_1 - \Sigma_{S^f}(t,T)\sqrt{T-t} 
		\end{eqnarray*}
	\end{prop}
	\begin{dem}$\quad$
		Il suffit d'appliquer la formule de Black-Scholes-Merton \footnote{cf. proposition \ref{PrixCallBS} p. \pageref{PrixCallBS}}
		 sur le march\'e \'etranger.\\
		\begin{flushright}
			$\boxslash$
		\end{flushright}
	\end{dem}	
		

\section{Call quanto avec garantie de change}

\subsection{Evaluation d'un call quanto avec garantie de change}

	L'\'evaluation est plus compliqu\'ee lorsque le taux de change appliqu\'e \`a l'\'ech\'eance est 
stipul\'e dans le contrat.\\
On notera $\X $ \  ce taux. Il s'agit alors d'\'evaluer la valeur en $t$ du flux : 
		\begin{eqnarray*}
			\X (S^f_T - K^f)_+
		\end{eqnarray*}	
	
	\begin{prop}[Prix d'un call quanto avec garantie de change]$\quad$ \label{callQX}
		Le prix du call quanto avec taux de change fix\'e par le contrat est :
		\begin{equation}
			\boxed{\, Call_t^{\X}( T, K^f, S^f ) \, = \,  \X \,  \left \{ \, S_t^f e^{-\Gamma_X^{S^f}(t,T) (T-t)}
			 e^{-\int_t^T (r_u-r_u^f) du} \N(D_1) - K^f e^{-\int_t^T r_u du} \N(D_2) \, \right \} \, }
		\end{equation}
avec : 
		\begin{eqnarray}
			D_1 &=& \frac{ \ln{\{ \frac{S_t^f}{K^f} \}} + (R^f(t,T)- \Gamma_X^{S^f}(t,T) + \frac 1 2 
				\Sigma^2_{S^f}(t,T) ) (T-t)}
					{\Sigma_{S^f}(t,T) \sqrt{T-t}}\\
			D_2 &=& \frac{ \ln{\{ \frac{S_t^f}{K^f} \}} + (R^f(t,T)- \Gamma_X^{S^f}(t,T) - \frac 1 2 
				\Sigma^2_{S^f}(t,T) ) (T-t)}{\Sigma_{S^f}(t,T) \sqrt{T-t}}\\
			&=& D_1 - \Sigma_{S^f}(t,T) \sqrt{T-t}
		\end{eqnarray}
	\end{prop}
	\begin{dem}
		\begin{itemize}
			\item	On a :
			\begin{eqnarray*}
				\E_{\Q} \left \lbrack \, \X (S^f_T - K^f)_+ \, / \, \F_t \, \right \rbrack
			 	&=& \X \, \E_{\Q} \left \lbrack \,  (S^f_T - K^f)_+ \, / \, \F_t \, \right \rbrack \\
				&=& \X \, \E_{\Q} \left \lbrack \,  (S^f_T - K^f) \1_{\{ S^f_T \geq K^f\} } \, / \,
					 \F_t \, \right \rbrack \\
			\end{eqnarray*}	
			\item	En utilisant la diffusion de $\{ \, S^f_t , t \in \lbrack 0 , T \rbrack \, \} 
				$ sous $\Q$ et la m\'ethode d'\'evaluation d'un call classique, on obtient : 
			\begin{eqnarray*}
				Call_t^{\X}( T, K^f, S^f ) = \X \, e^{-\int_t^T r_u du} 
				\left \{ \,
				\E_{\Q} \left \lbrack \,  S^f_T \1_{\{ S^f_T \geq K^f\} }  \, / \, \F_t \, \right \rbrack 
				- K  \Q \left \lbrack \, S^f_T \geq K^f \, / \, \F_t \, \right \rbrack \right \} 									
			\end{eqnarray*}
			\item	{\bf Probabilit\'e d'exercice}\\
			Comme pr\'ec\'edemment, on est amen\'e \`a \'evaluer la probabilit\'e d'exercice de l'option :
			\begin{eqnarray*}
				\Q \left \lbrack \, S^f_T \geq K^f\ \, / \, \F_t \, \right \rbrack &=& 
				\Q \left \lbrack \,  S_t \exp{ \left \{ \, (R^f(t,T)- \Gamma_X^{S^f}(t,T)-\frac 1 2 
				\Sigma^2_{S^f}(t,T) ) (T-t) + \Sigma_{S^f}(t,T) \sqrt{T-t} \, U \, \right \}} \geq K^f\ 
				\, / \, \F_t \, \right \rbrack \\
				\end{eqnarray*}
			avec $U \stackrel{\mathcal L}{\sim} \N(0,1) $ \\
	\indent On trouve par des calculs similaires que la probabilit\'e d'exercice du call est :\\
	\begin{displaymath}
		\N(D_2^f)
	\end{displaymath}
	avec : 
	\begin{eqnarray}
			D_2^f &=& \frac{ \ln{\{ \frac{S_t^f}{K^f} \}} + (R^f(t,T)- \Gamma_X^{S^f}(t,T)-\frac 1 2 \Sigma^2_{S^f}(t,T) ) (T-t)}
					{\Sigma_{S^f}(t,T) \sqrt{T-t}}
	\end{eqnarray}

			\item Le traitement du premier terme :
			\begin{eqnarray*}
				\E_{\Q} \left \lbrack \,  S^f_T \1_{\{ S^f_T \geq K^f\} }  \, / \, 
				\F_t \, \right \rbrack 
			\end{eqnarray*}
fait apparaitre un facteur suppl\'ementaire. En effet, en  utilisant la m\^eme m\'ethode que pour la d\'emonstration 
de la formule de Black-Scholes-Merton \ref{DemBSM} p. \pageref{DemBSM}, on a :
			\begin{eqnarray*}
				&&\E_{\Q} \left \lbrack \,  S^f_T \1_{\{ S^f_T \geq K^f\} }  \, / \, \F_t \, \right \rbrack \\
				&&= \, \E_{\Q} \left \lbrack \, S_t^f \exp{ \left \{ \, (R^f(t,T)- \Gamma_X^{S^f}(t,T)
				-\frac 1 2 \Sigma^2_{S^f}(t,T) ) (T-t) + \Sigma_{S^f}(t,T) \sqrt{T-t} \, U \, 
				\right \}}\1_{\{ S^f_T \geq K^f\} }  \, / \, \F_t \, \right \rbrack \\
				&&=\,  S_t^f \exp{ \left \{ \, (R^f(t,T)- \Gamma_X^{S^f}(t,T)) (T-t)\, \right \} 
				}\, \E_{\Q} \left \lbrack \,\exp{ \left \{ \, -\frac 1 2 \Sigma^2_{S^f}(t,T)  
				(T-t) + \Sigma_{S^f}(t,T) \sqrt{T-t} \, U \, \right \}}\1_{\{ S^f_T \geq K^f\} 
				}  \, / \, \F_t \, \right \rbrack \\
				&&=\, \frac{S_t^f e^{-\Gamma_X^{S^f}(t,T) (T-t)}}{e^{-\int_t^T r^f_u du}} \, 
				\int_{-D_2^f}^{\infty} e^{-\frac 1 2 \Sigma^2_{S^f}(t,T)  (T-t) + \Sigma_{S^f}(t,T) 
				\sqrt{T-t} \, U } \, \frac 1 {\sqrt{2 \pi}} e^{\frac {-u^2} 2} \dif u \\
			\end{eqnarray*}
Pour les m\^emes raisons que pr\'ec\'edemment, ce terme est \'egal \`a :
			\begin{eqnarray*}
				\E_{\Q} \left \lbrack \,  S^f_T \1_{\{ S^f_T \geq K^f\} }  \, / \, 
				\F_t \, \right \rbrack &=& 	\frac{S_t^f e^{-\Gamma_X^{S^f}(t,T) (T-t)}}
				{e^{-\int_t^T r^f_u du}} \, \N(D_1^f) \\
				\end{eqnarray*}
avec :
			\begin {eqnarray*}
				D_1^f &=& D_2^f + \Sigma_{S^f}(t,T) \sqrt{T-t}\\
			\end{eqnarray*}
		\end{itemize}
		\begin{flushright}
			$\boxslash$
		\end{flushright}
	\end{dem}
	
\subsection*{Prix du put quanto avec garantie de change}

	Par la relation de parit\'e Call-Put \footnote{cf. remarque  \ref{Parite} p. \pageref{Parite} dans l'Annexe}, 
le r\'esultat suivant est imm\'ediat :

	\begin{prop}[Prix d'un put quanto avec garantie de change]$\quad$ \label{putQX}
		\begin{equation}
			\boxed{\, Put_t^{\X}( T, K^f, S^f ) =\X \, \,e^{-R(t,T) (T-t)}  \left \{
			 	K^f \, \N(-D_2^f) \, - \,  
				S_t^f \, e^{(R^f(t,T)+\Gamma_X^{S^f}(t,T)) (T-t)} \, \N(-D_1^f)
				\,\right \}\\ }
		\end{equation} 
	avec :
		\begin{eqnarray*}
			D_1^f &=& \frac{ \ln{\{ \frac{S_t^f}{K^f} \}} + (R^f(t,T)- \Gamma_X^{S^f}(t,T) + \frac 1 2 \Sigma^2_{S^f}(t,T) ) (T-t)}
					{\Sigma_{S^f}(t,T) \sqrt{T-t}}\\
			D_2^f &=& \frac{ \ln{\{ \frac{S_t^f}{K^f} \}} + (R^f(t,T)- \Gamma_X^{S^f}(t,T) - \frac 1 2 \Sigma^2_{S^f}(t,T) ) (T-t)}
					{\Sigma_{S^f}(t,T) \sqrt{T-t}}\\
			&=& D_1 - \Sigma_{S^f}(t,T) \sqrt{T-t}
		\end{eqnarray*}	
	\end{prop}
	
\subsection{Les \guillemotleft \ grecques \guillemotright \ d'un call quanto avec garantie de change}

	Les calculs des grecques d'un call quanto avec garantie de change s'appuieront sur le r\'esultat suivant :
	\begin{prop}[R\'esultat pr\'eliminaire] \label{res pre quanto} $\quad$
		\begin{equation*}
			\boxed{
			\phi(D_2^f) = \frac{S_t^f}{K^f} e^{(R^f(t,T)-\Gamma_{X}^{S^f}(t,T))(T-t)} \phi(D_1^f)
			}
		\end{equation*}
	o\`u $\phi(.)$ d\'esigne la densit\'e de la $\N(0,1)$ 
	\end{prop}
	\begin{dem}$\quad$
		En effet, on a :
		\begin{eqnarray*}
			\phi(D_2^f) &=& \frac1 {\sqrt{2 \pi}} \exp \left \{- \frac 1 2 \left ( D_1^f - \Sigma_{S^f}(t,T) \sqrt{T-t} \right )^2 \right \} \\
			&=& \frac1 {\sqrt{2 \pi}} \exp \left \{- \frac 1 2 \left ( (D_1^f)^2 - 2 D_1^f \Sigma_{S^f}(t,T) \sqrt{T-t} + \Sigma^2_{S^f}(t,T) (T-t) \right ) \right \} \\
			&=& \left [ \frac1 {\sqrt{2 \pi}} e^{-\frac 1 2 (D_1^f)^2 }  \right ] + 
			\exp \left \{  \ln \left( \frac{S_t^f}{K^f} \right ) + \left (R^f(t,T)-\Gamma_X^{S^f}(t,T) +
			 \frac 1 2 \Sigma^2_{S^f}(t,T) \right )(T-t) -  \frac 1 2 \Sigma^2_{S^f}(t,T) (T-t)\right \}\\
			&=& \phi(D_1^f) \frac{S_t^f}{K^f} e^{(R^f(t,T)-\Gamma_X^{S^f}(t,T))(T-t)}\\
		\end{eqnarray*}	
		\begin{flushright}
			$\boxslash$
		\end{flushright}
	\end{dem}

\subsubsection*{$\bullet \quad$ Le Delta}
	
	\begin{prop}[Delta d'un call quanto avec garantie de change] \label{delta quanto} $\quad$
	Le delta d'un call quanto avec garantie de change est donn\'e par :
		\begin{equation*}
			\boxed{
			\Delta^{Call_t^{\X}( T, K^f, S^f )} \,
			= \, \X e^{-(\Gamma_{X}^{S^f}(t,T) + R(t,T)-R^f(t,T))(T-t)} \N(D_1^f)
				}
		\end{equation*}	
	\end{prop}
	\begin{dem}[Delta d'un call quanto avec garntie de change] $\quad$
	Nous avons :
		\begin{eqnarray*}
			\Delta^{Call_t^{\X}( T, K^f, S^f )} &=& \frac{\partial Call_t^{\X}( T, K^f, S^f )}{\partial S_t^f} \\
			&=& \X  e^{-(\Gamma^{S^f}_{X}(t,T) + R(t,T)-R^f(t,T))(T-t)}\N(D_1^f) \\
			&& + \X S_t^f e^{-(\Gamma^{S^f}_{X}(t,T) + R(t,T)-R^f(t,T))(T-t)} \phi(D_1^f) \frac{\partial D_1^f}{\partial S_t^f}\\
			&& -\X K e^{R(t,T)(T-t)} \phi(D_2^f) \frac{\partial D_2^f}{\partial S_t^f} \\
		\end{eqnarray*}
		On a :
		\begin{eqnarray*}
			\frac{\partial D_2^f}{\partial S_t^f} &=& \frac{\partial D_1^f}{\partial S_t^f}	
		\end{eqnarray*}
		Ainsi, par application du  r\'esultat de la proposition \ref{res pre quanto} p. \pageref{res pre quanto}
		\begin{eqnarray*}
			\Delta^{Call_t^{\X}( T, K^f, S^f )} &=& \X e^{-(\Gamma^{S^f}_{X}(t,T) + R(t,T)-R^f(t,T))(T-t)}\N(D_1^f) \\
			&& + \X S_t^f e^{-(\Gamma^{S^f}_{X}(t,T) + R(t,T)-R^f(t,T))(T-t)} \phi(D_1^f) \frac{\partial D_1^f}{\partial S_t^f}\\
			&& - \X  K^f e^{R(t,T)(T-t)} \frac{S_t^f}{K} e^{(R^f(t,T)-\Gamma_{X}^{S^f}(t,T))(T-t)} \phi(D_1^f)\frac{\partial D_1^f}{\partial S_t^f} \\
			&=&  \X e^{-(\Gamma^{S^f}_{X}(t,T) + R(t,T)-R^f(t,T))(T-t)}\N(D_1^f) \\
		\end{eqnarray*}
		\begin{flushright}
			$\boxslash$
		\end{flushright}
	\end{dem}

\subsubsection*{$\bullet \quad$ Le Gamma}

	\begin{prop}[Gamma d'un call quanto avec garantie de change] \label{gamma quanto} $\quad$
		\begin{equation*}
			\boxed{
			\Gamma^{Call_t^{\X}( T, K^f, S^f )} \, = \,
			\frac {\X e^{-(\Gamma^{S^f}_{X}(t,T) + R(t,T)-R^f(t,T))(T-t)}}
			{ S_t^f\Sigma_{S^f}(t,T) \sqrt{T-t}} \phi(D_1^f)
			}
		\end{equation*}
	\end{prop}
	\begin{dem}[Gamma d'un call quanto avec garantie de change]  $\quad$
	En effet, 
		\begin{eqnarray*}
			\Gamma^{Call_t^{\X}( T, K^f, S^f )} &=& \frac {\partial \Delta^{Call_t^{\X}( T, K^f, S^f )}} {\partial S_t^f} \\
			&=& \X  e^{-(\Gamma^{S^f}_{X}(t,T) + R(t,T)-R^f(t,T))(T-t)} \phi(D_1^f) \frac{\partial D_1^f} {\partial S_t^f} \\
		\end{eqnarray*}
	Et, 
		\begin{eqnarray*}
			\frac{\partial D_1^f} {\partial S_t^f} &=& \frac {\partial} {\partial S_t^f} \left [
			\frac{ \ln{\{ \frac{S_t^f}{K^f} \}} + (R^f(t,T)- \Gamma_X^{S^f}(t,T) + \frac 1 2 \Sigma^2_{S^f}(t,T) ) (T-t)}
					{\Sigma_{S^f}(t,T) \sqrt{T-t}}
				\right ] \\
			&=& \frac 1 {S_t^f\Sigma_{S^f}(t,T) \sqrt{T-t}} \\
		\end{eqnarray*}
		\begin{flushright}
			$\boxslash$
		\end{flushright}
	\end{dem}

\subsubsection*{$\bullet \quad$ Sensibilit\'e par rapport aux taux d'int\'er\^et}

	La formule de prix d'un call quanto avec garantie de change faisant intervenir deux taux :\\
	\begin{itemize}
		\item	le taux  moyen $R(t,T)$ entre $t$ et $T$ sur le march\'e domestique \\
		\item 	le taux  moyen $R^f(t,T)$ entre $t$ et $T$ sur le march\'e \'etranger \\
	\end{itemize}
	on s'int\'eressera \`a la sensibilit\'e de l'option par rapport \`a chacun de ces taux.

	\begin{prop}[Sensibilit\'e par rapport au taux domestique] $\quad$
	En notant :\\ $\rho_1^{Call_t^{\X}( T, K^f, S^f )}$ , la sensibilit\'e par rapport au taux domestique, on a :
		\begin{equation*}
			\boxed{
			\rho_1^{Call_t^{\X}( T, K^f, S^f )} \, = \, 
			-(T-t) Call_t^{\X}( T, K^f, S^f )
			}
		\end{equation*}
	\end{prop}
	\begin{dem}[Sensibilit\'e par rapport au taux domestique]  $\quad$
	En effet, les formules pour $D_1^f$ et $D_2^f$ ne faisant intervenir que le taux \'etranger, on a :
		\begin{eqnarray*}
			\rho_1^{Call_t^{\X}( T, K^f, S^f )} &=& \frac {\partial Call_t^{\X}( T, K^f, S^f )} {\partial R(t,T)} \\
			&=& -(T-t)\X S_t^f e^{-(\Gamma^{S^f}_{X}(t,T) + R(t,T)-R^f(t,T))(T-t)} \N(D_1^f) 
			+\,  \X K^f (T-t) e^{-R(t,T)(T-t)} \N(D_2^f) \\
			&=&-(T-t) Call_t^{\X}( T, K^f, S^f ) \\
		\end{eqnarray*}
		\begin{flushright}
			$\boxslash$
		\end{flushright}
	\end{dem}
	\begin{prop}[Sensibilit\'e par rapport au taux \'etranger] $\quad$
		En notant : \\ $\rho_2^{Call_t^{\X}( T, K^f, S^f )}$ , la sensibilit\'e par rapport au taux \'etranger, on a :
		\begin{equation*}
			\boxed{
			\rho_2^{Call_t^{\X}( T, K^f, S^f )} \, = \, (T-t) \X S_t^f e^{-(\Gamma^{S^f}_{X}(t,T) +
			 R(t,T)-R^f(t,T))(T-t)} \N(D_1^f) 
			}
		\end{equation*}
	\end{prop}
	\begin{dem}[Sensibilit\'e par rapport au taux \'etranger] $\quad$
	On a :
		\begin{eqnarray*}
			\rho_2^{Call_t^{\X}( T, K^f, S^f )} &=& \frac{\partial Call_t^{\X}( T, K^f, S^f )} {\partial R^f(t,T)} \\
			&=& \X S_t^f (T-t) e^{-(\Gamma^{S^f}_{X}(t,T) + R(t,T)-R^f(t,T))(T-t)} \N(D_1^f) \\
			&& + \, \X S_t^f e^{-(\Gamma^{S^f}_{X}(t,T) + R(t,T)-R^f(t,T))(T-t)} \phi(D_1f) \frac{\partial D_1^f}{\partial R^f(t,T)}\\
			&& - \, \X K^f (T-t) e^{-R(t,T)(T-t)} \phi(D_2f) \frac{\partial D_2^f}{\partial R^f(t,T)}\\
		\end{eqnarray*}
	De plus :
		\begin{eqnarray*}
			\frac{\partial D_2^f}{\partial R^f(t,T)} &=& \frac{\partial D_1^f}{\partial R^f(t,T)}
		\end{eqnarray*}
	Donc :
		\begin{eqnarray*}
			\rho_2^{Call_t^{\X}( T, K^f, S^f )} &=& 
			\X S_t^f (T-t) e^{-(\Gamma^{S^f}_{X}(t,T) + R(t,T)-R^f(t,T))(T-t)} \N(D_1^f) \\
			&& + \, \X S_t^f e^{-(\Gamma^{S^f}_{X}(t,T) + R(t,T)-R^f(t,T))(T-t)} \phi(D_1f) \frac{\partial D_1^f}{\partial R^f(t,T)}\\
			&& - \, \X K^f (T-t) e^{-R(t,T)(T-t)} \frac{S_t^f}{K^f} e^{(R^f(t,T)-\Gamma_{X}^{S^f}(t,T))(T-t)} \phi(D_1^f)
				\frac{\partial D_1^f}{\partial R^f(t,T)} \\
			&=& \X S_t^f (T-t) e^{-(\Gamma^{S^f}_{X}(t,T) + R(t,T)-R^f(t,T))(T-t)} \N(D_1^f) \\
		\end{eqnarray*}
		\begin{flushright}
			$\boxslash$
		\end{flushright}
	\end{dem}
\newpage	
\section{Evaluation d'un put \guillemotleft \ gap \guillemotright \ sur sous-jacent \'etranger}

	\begin{df}[Put europ\'een \guillemotleft \ gap-quanto \guillemotright \ avec garanti de change] $\quad$ \label{putgapQX}
	Il s'agit d'un contrat donnant droit \`a son d\'etenteur de vendre  \`a la date $T$, un sous-jacent $S^f$ exprim\'e en 
monnaie \'etrang\`ere, \`a un prix $L^f$ et pour un taux de change $\X$ fix\'es \`a l'avance, et \`a condition que le prix
du sous-jacent ne d\'epasse pas un montant $K^f$ \'egalement fix\'e. \\
Le payoff du contrat est donn\'e par :\\
		\begin{center}
			$\X \, ( L^f - S^f_T) \, \1_{\{ S^T_f \leq K^f \}}$  
		\end{center}		
	\end{df}

 	\begin{prop}[Prix du put \guillemotleft \ gap \guillemotright \ sur sous-jacent \'etranger avec garantie de change]  $\quad$
	Pour un taux de change $\X$ fix\'e \`a l'avance, le prix du put gap sur un sous-jacent \'etranger est donn\'e par :\\
		\begin{equation}\label{put gap quanto}
			\X \, \,e^{-R(t,T) (T-t)}  \left \{
			 	L^f \, \N(-D_2^f) \, - \,  
				S_t^f \, e^{(R^f(t,T)- \Gamma_X^{S^f}(t,T)) (T-t)} \, \N(-D_1^f)
				\,\right \}\\ 
		\end{equation} 
	avec :
		\begin{eqnarray*}
			D_1 &=& \frac{ \ln{\{ \frac{S_t^f}{K^f} \}} + (R^f(t,T)- \Gamma_X^{S^f}(t,T) + 
				\frac 1 2 \Sigma^2_{S^f}(t,T) ) (T-t)}
				{\Sigma_{S^f}(t,T) \sqrt{T-t}}\\
			D_2 &=& \frac{ \ln{\{ \frac{S_t^f}{K^f} \}} + (R^f(t,T)- \Gamma_X^{S^f}(t,T) - 
				\frac 1 2 \Sigma^2_{S^f}(t,T) ) (T-t)}{\Sigma_{S^f}(t,T) \sqrt{T-t}}\\
			&=& D_1 - \Sigma_{S^f}(t,T) \sqrt{T-t} \\
		\end{eqnarray*}	
	\end{prop}
	\begin{dem}$\quad$
		On utilise la formule du put quanto avec  garantie de change  \footnote{cf. proposition 
\ref{putQX} p. \pageref{putQX}} 
en effectuant les changements n\'ecessaires pour distinguer les strikes $K^f$ et $L^f$ intervenant 
respectivement dans la probabilit\'e d'exercice de l'option et dans le flux \`a l'\'ech\'eance.\\
		\begin{flushright}
			$\boxslash$
		\end{flushright}	
	\end{dem}

\section{Les grecques du put \guillemotleft \ gap-quanto \guillemotright \ avec garantie de change}

	\begin{rem}$\quad$
		Les calculs seront tr\`es similaires \`a ceux des grecques pour les options quanto avec garantie de change, 
		ils s'appuieront \'egalement sur la proposition \ref{res pre quanto} p. \pageref{res pre quanto}.
	\end{rem}

\subsection*{Le delta}
	\begin{prop}[Delta du put \guillemotleft \ gap-quanto \guillemotright \ avec garantie de change] $\quad$
		\begin{equation*}
			\boxed{
			\Delta(Put^{quanto}_{gap}) \, =\, 
		\X e^{-[R(t,T)+\Gamma_X^{S^f}(t,T) - R^f(t,T)](T-t)} \left \{
		\frac {\phi(D_1^f)}{\Sigma_{S^f}(t,T) \sqrt{T-t}} \left [ 1 - \frac{L^f}{K^f}\right ]
			- \N(-D_1^f)	
			\right \}
			}
		\end{equation*}
	\end{prop}
	\begin{dem}[Delta du put \guillemotleft \ gap-quanto \guillemotright \ avec garantie de change] $\quad$
	En effet, en d\'erivant l'expression (\ref{put gap quanto}) p. \pageref{put gap quanto}, on obtient :
		\begin{eqnarray*}
			\Delta(Put^{quanto}_{gap}) &=& \X L^f e^{-R(t,T)(T-t)} \phi(-D_2^f) \left 
			( - \frac {\partial D_2^f}{\partial S_t^f}\right)\\
			&& - \, \X e^{-[R(t,T)+\Gamma_X^{S^f}(t,T)-R^f(t,T)](T-t)} \N(-D_1^f) \\
			&& + \, \X S_t^f e^{-[R(t,T)+\Gamma_X^{S^f}(t,T)-R^f(t,T)](T-t)} \phi(-D_1^f) \left ( 
				- \frac {\partial D_1^f}{\partial S_t^f}\right)\\ 
		\end{eqnarray*}
On a d'une part, 
		\begin{eqnarray*}
			\phi(-D_1^f) &=& \phi(D_1^f) \\
			\phi(-D_2^f) &=& \phi(D_2^f) \\
				&=& \frac{S_t^f}{K^f} e^{(R^f(t,T)-\Gamma_{X}^{S^f}(t,T))(T-t)} \phi(D_1^f) \\
				&& \text{\tiny{par la proposition \ref{res pre quanto} p. \pageref{res pre quanto}}}
		\end{eqnarray*}
et d'autre part, 
		\begin{eqnarray*}
			 \frac {\partial D_2^f}{\partial S_t^f} &=&  \frac {\partial D_1^f}{\partial S_t^f} \\
			&=& \frac 1 {S_t^f \Sigma_{S^f}(t,T) \sqrt{T-t}}\\
		\end{eqnarray*}
On obtient :
		\begin{eqnarray*}
			\Delta(Put^{quanto}_{gap}) &=&  - \, \X L^f e^{-R(t,T)(T-t)}\frac{S_t^f}{K^f} 
			e^{(R^f(t,T)-\Gamma_{X}^{S^f}(t,T))(T-t)} \phi(D_1^f)  \frac {\partial D_1^f}{\partial S_t^f}\\
			&& - \, \X e^{-[R(t,T)+\Gamma_X^{S^f}(t,T)-R^f(t,T)](T-t)} \N(-D_1^f) \\
			&& - \, \X S_t^f e^{-[R(t,T)+\Gamma_X^{S^f}(t,T)-R^f(t,T)](T-t)} \phi(D_1^f)  
			\frac {\partial D_1^f}{\partial S_t^f}\\ 
			&=& \X e^{-[R(t,T)+\Gamma_X^{S^f}(t,T) - R^f(t,T)](T-t)} \left \{
			\frac {\phi(D_1^f)}{\Sigma_{S^f}(t,T) \sqrt{T-t}} \left [ 1 - \frac{L^f}{K^f}\right ]
			- \N(-D_1^f)	
			\right \}
		\end{eqnarray*}
		\begin{flushright}
			$\boxslash$
		\end{flushright}	
	\end{dem}
	\begin{rem}$\quad$
	Lorsque $L^f = K^f$, on retrouve bien l'expression du delta d'un put quanto avec garantie de change.
	\end{rem}

\subsection*{Le Gamma}
	
	\begin{prop}[Gamma d'un put \guillemotleft \ gap-quanto \guillemotright \ avec garantie de change]$\quad$
		\begin{equation*}
			\boxed{
			\Gamma(Put^{quanto}_{gap}) \, = \, 
			\frac {\X \phi(D_1^f)}{S_t^f \Sigma_{S^f}(t,T) \sqrt{T-t}}e^{-[R(t,T)+\Gamma_X^{S^f}(t,T)-R^f(t,T)](T-t)}
			\left \{ \frac{D_1^f}{\Sigma_{S^f}(t,T) \sqrt{T-t}} \left [ \frac{L^f}{K^f}-1 \right ] +1 
			\right \} 
				}
		\end{equation*}
	\end{prop}
	\begin{dem}[Gamma d'un put \guillemotleft \ gap-quanto \guillemotright \ avec garantie de change]$\quad$
	Il suffit de d\'eriver $\Delta(Put^{quanto}_{gap})$ par rapport \`a $S_t^f$
		\begin{flushright}
			$\boxslash$
		\end{flushright}	
	\end{dem}

\subsection*{Sensibilit\'es par rapport aux taux d'int\'er�t}
	
	\begin{prop}[sensibilit\'e par rapport au taux domestique] $\quad$
		\begin{equation*}
			\boxed{
		\rho_1(Put^{quanto}_{gap}) \, = \, - (T-t) Put(t,T,K^f,S^f,L^f, \X)
			}
		\end{equation*}
	avec $Put(t,T,K^f,S^f,L^f,\X)$, le prix en $t$ de l'option, et $\rho_1(Put^{quanto}_{gap})$, 
	sa sensibilit\'e par rapport au taux domestique.
	\end{prop}
	\begin{dem}[sensibilit\'e par rapport au taux domestique]$\quad$
	Le r\'esultat est imm\'ediat en remarquant que le taux domestique n'intervient dans le prix du contrat que par actualisation 
du flux \`a l'\'ech\'eance ; autrement dit il est absent des expressions de $D_1^f$ et $D_2^f$.
		\footnote{
			En fait, les param\`etres intervenant dans les expressions de  $D_1^f$ et $D_2^f$, sont ceux intervenant 
			dans la diffusion de $S_t^f$ sous la probabilit\'e risque neutre domestique $\Q$ 
			} 
		\begin{flushright}
			$\boxslash$
		\end{flushright}
	\end{dem}

\subsubsection*{sensibilit\'e par rapport au taux \'etranger}
	
	\begin{prop}[Sensibilit\'e par rapport au taux \'etranger] $\quad$
		\begin{equation*}
			\boxed{
			\rho_2(Put^{quanto}_{gap}) \, = \,
			(T-t) \X S_t^f e^{-[R(t,T)+\Gamma_X^{S^f}-R^f(t,T)](T-t)} 
			\left \{ \frac{\phi(D_1^f)}{\Sigma_{S^f}(t,T)\sqrt{T-t}} 
				\left [ 1 -  \frac{L^f}{K^f} \right ] - \N(-D_1^f) 
			\right \}
			}
		\end{equation*}
	o\`u $\rho_2(Put^{quanto}_{gap}) \,$ est la sensibilit\'e par rapport \`a $R^f(t,T)$ du put \guillemotleft \ 
gap-quanto \guillemotright \ avec 
	garantie de change  
	\end{prop}
	\begin{dem}[Sensibilit\'e par rapport au taux \'etranger] $\quad$
		Le r\'esultat est imm\'ediat en utilisant : \\
		\begin{itemize}
			\item[$\bullet$]	
			\begin{eqnarray*}
				\frac{\partial D_2^f}{\partial R^f(t,T)} &=& \frac{\partial D_1^f}{\partial R^f(t,T)} \\
				&=& \frac{T-t}{\Sigma_{S^f}(t,T) \sqrt{T-t}}\\
			\end{eqnarray*}
			\item[$\bullet$]
			\begin{eqnarray*}
				\phi(-D_2^f) &=& \phi(D_2^f)\\
				&=&  \frac{S_t^f}{K^f} e^{(R^f(t,T)-\Gamma_{X}^{S^f}(t,T))(T-t)} \phi(D_1^f)\\
			\end{eqnarray*}
		\end{itemize}
		\begin{flushright}
			$\boxslash$
		\end{flushright}
	\end{dem}


%***********************************************************
\chapter{Produits structur\'es sur le p\'etrole : Exemple 1}
%***********************************************************

\section{Description de la structure }
	
	On consid\`ere une structure optionnelle sur un nominal de $ N \,$ \officialeuro \  dont le sous-jacent est 
le p\'etrole exprim\'e en monnaie \'etrang\`ere (\$ par baril).  
	\begin{description}
		 \item[$\bullet \quad$]On note :
			\begin{description}
				\item [$\quad t \quad$] La date de valorisation de l'option \\
				\item [$\quad T \quad$] L'\'ech\'eance de l'option \\
				\item [$\quad \left \{ S_t^f, t \right \} \quad $] 
					Le processus de prix du p\'etrole ( en \$ ) \\
			\end{description}
		\item[$\bullet \quad$] Le flux re\c cu \`a l'\'ech\'eance $T$ est : \\
			\begin{itemize}
				\item	si $\quad S_T^f < 70 \quad$ alors $\quad 0$ \\
				\item	si $\quad 70 \leq S_T^f < 75 \quad$ alors $\quad S_T^f - 70$\\
				\item	si $\quad 75 \leq S_T^f < 80 \quad$ alors $\quad 5 + 2 ( S_T^f - 75 )$ \\ 
				\item	si $\quad 80 \leq S_T^f < 85 \quad$ alors $\quad 15 + 3 ( S_T^f - 80 )$ \\
				\item	si $\quad S_T^f \geq 85 \quad$ alors $\quad 30$ \\
			\end{itemize}
		\item[$\bullet \quad$]Le taux de change appliqu\'e \`a l'\'ech\'eance est fix\'e \`a :
				 \, 1  \, \$ \, = \, 1  \, \officialeuro \\
	
		\item[$\bullet \quad$]Le nominal $N$ est exprim\'e en monnaie \officialeuro.
	\end{description}		

	\begin{center}
		\framebox{On est dans le cadre d'un produit structur\'e quanto avec garantie de change}
	\end{center}		
	
\section{Evaluation de la structure}
	

\subsection{D\'ecomposition du payoff}
	
	Le flux \`a l'\'ech\'eance $T$ peut \^etre dupliqu\'e par un portefeuille constitu\'e des produits suivants : \\
	\begin{itemize}
		\item	l'achat d'un call de strike $K_1^f = 70 $ \$ et d'\'ech\'eance $T$\\
		\item	l'achat d'un call de strike $K_2^f =75 $ \$ et d'\'ech\'eance $T$\\
		\item	l'achat d'un call de strike $K_3^f = 80  $ \$ et d'\'ech\'eance $T$	\\	
		\item	la vente de trois call de strike $K_4^f = 85$ \$ et d'\'ech\'eance $T$\\
	\end{itemize}

En effet, la valeur du portefeuille \`a l'\'ech\'eance est :\\
	\begin{eqnarray*}
	(S_T^f - 70)_+ + (S_T^f - 75)_+ + (S_T^f - 80)_+ + 3 (85 - S_T^f)_+ \\
	\end{eqnarray*}
	\begin{itemize}
		\item	{\bf si $S_T^f < 70$ ,} $\quad \text{payoff} = 0$ \\
		\item	{\bf si $70 \leq S_T^f < 75$ ,} $\quad \text{payoff} = S_T^f - 70 $\\
		\item	{\bf si $75 \leq S_T^f < 80$ ,}
			\begin{eqnarray*}
				\text{payoff} & = &(S_T^f - 70) + (S_T^f - 75)\\
				& = & 2 S_T^f -(75-5) -75 \\
				& = & 2 (S_T^f - 75 ) + 5
			\end{eqnarray*} \\
		\item	{\bf si $80 \leq S_T^f < 85$ ,}
			\begin{eqnarray*}
				\text{payoff} & = &(S_T^f - 70) + (S_T^f - 75) + (S_T^f - 80) \\
				& = & 3 S_T^f -(80-10) -(80-5)-80 \\
				& = & 3 (S_T^f - 80 ) + 15\\
			\end{eqnarray*} \\
		\item	{\bf si $85 \leq S_T^f$ ,}
			\begin{eqnarray*}
				\text{payoff} & = &(S_T^f - 70) + (S_T^f - 75) + (S_T^f - 80) + 3 (85 - S_T^f)\\
				& = & -70 -75 -80 + 3 * 85 \\
				& = & 30 \\
			\end{eqnarray*} \\
	\end{itemize}
			
	Par AOA\footnote{cf. chapitre \ref{AOA} p. \pageref{AOA} dans l'Annexe}, 
les valeurs du portefeuille et du contrat sont les m\^emes \`a toute date $t$ pr\'ec\'edant l'\'ech\'eance $T$ .\\
Il s'agit donc d'\'evaluer la valeur \`a la date $t$ de quatre {\bf calls europ\'eens quanto avec garantie de change}.\\
On note \X \, le taux de change fix\'e par le contrat.
 
\subsection{Formule de prix}
	La valeur \`a toute date $t$ pr\'ec\'edant l'\'ech\'eance $T$ est donn\'ee par : 
	\begin{eqnarray*}
		Call_t^{\X}( T, K_1^f, S^f ) + Call_t^{\X}( T, K_2^f, S^f ) + Call_t^{\X}( T, K_3^f, S^f ) - 3 \, Call_t^{\X}( T, K_4^f, S^f )
	\end{eqnarray*}
	
	En appliquant la formule de la premi\`ere partie (cf. proposition \ref{callQX} p. \pageref{callQX}) , on trouve \footnote{en multipliant par le nominal} :
	\begin{eqnarray*}
		& \X \, S_t^f e^{-\Gamma_X^{S^f}(t,T) (T-t)} e^{-\int_t^T (r_u-r_u^f) du}\left \{ \, \N(D_1^1) + \N(D_1^2) + \N(D_1^3) - 3\, \N(D_1^4) \, \right \} &\\
		& - \X \, e^{-\int_t^T r_u du} \, K \left \{ \, \N(D_2^1) + \N(D_2^2) + \N(D_2^3) - 3\, \N(D_2^4) \, \right \}&\\
	\end{eqnarray*}
avec : $ \quad \forall i \in \llbracket \, 1 \, , \, 4 \, \rrbracket \quad$, 
	\begin{eqnarray*}
		D_1^i &=& \frac{ \ln{\{ \frac{S_t^f}{K_i^f} \}} + (R^f(t,T)- \Gamma_X^{S^f}(t,T) + \frac 1 2 \Sigma^2_{S^f}(t,T) ) (T-t)}
					{\Sigma_{S^f}(t,T) \sqrt{T-t}}\\
		D_2^i &=& \frac{ \ln{\{ \frac{S_t^f}{K_i^f} \}} + (R^f(t,T)- \Gamma_X^{S^f}(t,T) - \frac 1 2 \Sigma^2_{S^f}(t,T) ) (T-t)}
					{\Sigma_{S^f}(t,T) \sqrt{T-t}}\\
			&=& D_1^i - \Sigma_{S^f}(t,T) \sqrt{T-t} \\
	\end{eqnarray*}

\subsection*{Estimation empirique des inputs de la formule}
En pratique, l'application de cette formule n\'ecessitera in\'eluctablement l'estimation de certains inputs tels que : \\
	\begin{itemize}
		\item 	Les taux sans risque moyens $ R(t,T) $ et $R^f(t,T)$ sur les march\'es domestique 
			et \'etranger \\
		\item	Les volatilit\'es et la corr\'elation moyennes de $X$ et $S^f$ entre $t$ et $T$\\
	\end{itemize}
Ces quantit\'es pourront \^etre remplac\'ees par leur estimateur empirique bas\'e sur des donn\'ees historiques.
%***********************************************************
\chapter{Produits structur\'es sur le p\'etrole : Exemple 2}
%***********************************************************


\section{Description}
	On consid\`ere la structure de swap d\'ecrite par les terms sheet suivants :\\
	\begin{description}
		\item[$\bullet \quad$] dur\'ee du contrat : 3 ans \\
		\item[$\bullet \quad$] nominal : $N$ \officialeuro \\
		\item[$\bullet \quad$] tous les 3 mois la contrepartie A paye l'EURIBOR 3 mois constat\'e 3 mois plus t\^ot\\	
		\item[$\bullet \quad$] la contrepartie B paye une fois par an  :\\
			\begin{itemize}
				\item	6,05\% si le cours du p\'etrole (en \$) d\'epasse de plus de 75\% la valeur de r\'ef\'erence 
					stipul\'ee \`a la signature du contrat \\
				\item	0 sinon \\
			\end{itemize}
		\item[$\bullet \quad$] la valeur de r\'ef\'erence est fix\'ee \`a 65,50 \$ \\
	\end{description}
	
	
	Ce contrat est un swap particulier car la partie B ne paye le taux fixe  que si le prix du p\'etrole
 \`a chaque date de flux d\'epasse  de plus de 75\% une valeur de r\'ef\'erence stipul\'ee dans le contrat. \\
	Tout d'abord d\'eterminons l'\'ech\'eancier correspondant \`a ce contrat.\\
	On note : \\
	\begin{itemize}
		\item	$t \quad$ La date de signature du contrat \\ 
		\item	$T_0 \quad$ la date de d\'ebut de composition   \\
		\item	$T_1 \quad$ la date de tomb\'ee du premier flux (trois mois plus tard) \\
		\item 	de mani\`ere plus g\'en\'erale $T_i \, , \, i \in \llbracket \, 1 \, , \, 12 \, \rrbracket \quad$ la date de tomb\'ee du i-\`eme flux \\
		\item	$\delta_i  \quad $ le temps \'ecoul\'e (exprim\'e en fraction d'ann\'ee)  
			entre les date $T_i$\  et \ $T_{i-1}$\ 
			\footnote{il s'agit de la dur\'ee de composition du i-\`eme flux pour A}  \\
		\item	$\delta_i' \quad $  le temps \'ecoul\'e (exprim\'e en fraction d'ann\'ee)  
			entre les date $T_{4i}$\  et \ $T_{4(i-1)}$\ \footnote{il s'agit de la dur\'ee de composition 
			du i-\`eme flux pour B}  \\
		\item	$L(T_{i-1},\delta_i) \quad $ taux EURIBOR entre $T_{i-1}$ et $T_{i-1} + \delta_i = T_i$ \\
	\end{itemize}
 
	Le flux \`a l'\'ech\'eance est \footnote{en multipliant bien-s\^ur par le nominal exprim\'e en \ \officialeuro } :
	\begin{eqnarray*}
		\sum_{i=1}^{i=12} \delta_i L(T_{i-1},\delta_i)  - \sum_{i=1}^{i=3} \delta_i'\, 6,05 \%  \, \1_{\{ \, S^f_{T_{4i}} \geq K^f \, \}}
	\end{eqnarray*}
	avec : \\
	\begin{eqnarray*}
		K^f &=& 75 \,\, \% \, \,  65,50 \quad \$ \\
	\end{eqnarray*}
		
\section{Evaluation}
	Il s'agit alors d'\'evaluer :
	\begin{eqnarray*}
		\E_{\Q} \,\left \lbrack  
			\sum_{i=1}^{i=12} \frac{\beta(t)}{\beta(T_i)} \,\delta_i L(T_{i-1},\delta_i)  - \sum_{i=1}^{i=3} \frac{\beta(t)}{\beta(T_{4i})} \, \delta_i'\, 6,05 \%  \, \1_{\{ \, S^f_{T_{4i}} \geq K^f \, \}} \, / \, \F_t 
			\, \right \rbrack \\
		= \,  \sum_{i=1}^{i=12} \E_{\Q} \,\left \lbrack  \frac{\beta(t)}{\beta(T_i)} \,\delta_i L(T_{i-1},\delta_i) \, / \, \F_t \, \right \rbrack 
		\, - \, \sum_{i=1}^{i=3} \E_{\Q} \,\left \lbrack  \frac{\beta(t)}{\beta(T_{4i})} \, \delta_i'\, 6,05 \%  \, \1_{\{ \, S^f_{T_{4i}} \geq K^f \, \}} \, / \, \F_t \, \right \rbrack \\
	\end{eqnarray*}
	En notant :
	\begin{eqnarray*}
		A_i &=& \E_{\Q} \,\left \lbrack  \frac{\beta(t)}{\beta(T_i)} \,\delta_i L(T_{i-1},\delta_i) \, / \, \F_t \, \right \rbrack \\
	\end{eqnarray*}
	et,
	\begin{eqnarray*}
		B_i &=& \E_{\Q} \,\left \lbrack  \frac{\beta(t)}{\beta(T_{4i})} \, \delta_i'\, 6,05 \%  \, \1_{\{ \, S^f_{T_{4i}} \geq K^f \, \}} \, / \, \F_t \, \right \rbrack \\
	\end{eqnarray*}
	On peut \'ecrire que la valeur du contrat en $t\, $ est :
	\begin{eqnarray*}
		\sum_{i=1}^{i=12} A_i - \sum_{i=1}^{i=3} B_i \\
	\end{eqnarray*}

\subsection*{Evaluons $A_i$}

		
\subsubsection*{Formule pour l'EURIBOR}

	Le taux EURIBOR appliqu\'e entre $T_{i-1} \, $ et $T_{i-1} + \delta_i  \, = T_i $ est donn\'e par la formule suivante :
		\begin{eqnarray*}
			L(T_{i-1},\delta_i) &=& \frac 1 {\delta_i} \left ( \frac 1 {B(T_{i-1} , T_i)} - 1 \right ) \\
		\end{eqnarray*}

	avec :
		\begin{eqnarray*}
			B(T_{i-1} , T_i) &=& \exp{\left \{ \, 
						- \int_{T_{i-1}}^{T_i} r_s ds
						\, \right \}} \\
					&=& \frac{\beta(T_{i-1})}{\beta(T_i)}
		\end{eqnarray*}
	\subsubsection*{}

	on a donc :
	\begin{eqnarray*}
		A_i &=& \E_{\Q} \,\left \lbrack  \frac{\beta(t)}{\beta(T_i)} \,\delta_i L(T_{i-1},\delta_i) \, / \, \F_t \, \right \rbrack \\
		&=& \E_{\Q} \,\left \lbrack \frac{\beta(t)}{\beta(T_i)}  \,\delta_i  \, \frac 1 {\delta_i} \left ( \frac 1 {B(T_{i-1} , T_i)} - 1 \right ) \, / \, \F_t \, \right \rbrack \\
		&=& \E_{\Q} \,\left \lbrack \frac{\beta(t)}{\beta(T_i)} \, \frac 1 {B(T_{i-1} , T_i)} \, / \, \F_t \, \right \rbrack \, 
			- \E_{\Q} \,\left \lbrack \frac{\beta(t)}{\beta(T_i)} \, / \, \F_t \, \right \rbrack \\
		&=& \E_{\Q} \,\left \lbrack \frac{\beta(t)}{\beta(T_i)} \,\frac{\beta(T_i)}{\beta(T_{i-1})} \, / \, \F_t \, \right \rbrack \,
			- B(t , T_i) \\
		&=& B( t , T_{i-1} ) - B(t , T_i) \\
	\end{eqnarray*}
	
\subsection*{Evaluons $B_i$}
		
	\begin{eqnarray*}
		B_i &=& \E_{\Q} \,\left \lbrack  \frac{\beta(t)}{\beta(T_{4i})} \, \delta_i'\, 6,05 \%  \, \1_{\{ \, S^f_{T_{4i}} \geq K^f \, \}} \, / \, \F_t \, \right \rbrack \\
		&=& \delta_i'\, 6,05 \%  \, B( t , T_{4i} ) \, \E_{\Q} \,\left \lbrack \, \1_{\{ \, S^f_{T_{4i}} \geq K^f \, \}} \, / \, \F_t \, \right \rbrack \\
		&=&  \delta_i'\, 6,05 \%  \, B( t , T_{4i} ) \, \Q  \,\left \lbrack \, S^f_{T_{4i}} \geq K^f \,  / \, \F_t \, \right \rbrack \\
	\end{eqnarray*}
	
	Le terme : $\quad \Q  \,\left \lbrack \, S^f_{T_{4i}} \geq K^f \,  / \, \F_t \, \right \rbrack \,$ s'identifie \`a 
la probabilit\'e d'exercice d'un call europ\'een de strike $\, K^f \, $ (en \$), d'\'ech\'eance $\, T_{4i} \, $ sur un sous-jacent exprim\'e 
en monnaie \'etrang\`ere.\\
 Cette quantit\'e a \'et\'e \'evalu\'ee en premi\`ere partie, et on a :\\
		\begin{eqnarray*}
			\Q  \,\left \lbrack \, S^f_{T_{4i}} \geq K^f \,  / \, \F_t \, \right \rbrack
		&=&  \N ( D^{T_{4i}}_2 )
		\end{eqnarray*}
 	 o\`u en posant :\\
		\begin{itemize}
		\item	$\Gamma_X^{S^f}(t,T_{4i}) = \frac 1 {T_{4i} - t} \int_t^{T_{4i}} (\rho_u(X,S^f) \sigma_X(u)  \sigma_{S^f}(u)) du $\\
		\item	$\Sigma^2_{S^f}(t,T_{4i}) = \frac 1 {T_{4i} - t} \int_t^{T_{4i}} \sigma_{S^f}^2(u) du $ \\
		\item	$R^f(t,{T_{4i}}) = \frac 1 {{T_{4i}} - t} \int_t^{T_{4i}} r_u du $ \\
		\end{itemize}

	on a :
		\begin{eqnarray*}
			D^{T_{4i}}_2 &=& \frac{ \ln{\{ \frac{S_t^f}{K^f} \}} + (R^f(t,{T_{4i}})- \Gamma_X^{S^f}(t,{T_{4i}}) - \frac 1 2 \Sigma^2_{S^f}(t,{T_{4i}}) ) ({T_{4i}}-t)}
					{\Sigma_{S^f}(t,{T_{4i}}) \sqrt{{T_{4i}}-t}}\\
		\end{eqnarray*}

		\subsubsection*{Evaluation du contrat}
	En rassemblant ces calculs, le prix du contrat est donn\'e par \footnote{en multipliant par le nominal} :
		\begin{eqnarray*}
		\sum_{i=1}^{i=12} \, \left \{ \,   B( t , T_{i-1} ) - B(t , T_i)    \, \right \} 
		- \sum_{i=1}^{i=3} \, \left \{ \,   \delta_i'\, 6,05 \%  \, B( t , T_{4i} ) \,  \N ( D^{T_{4i}}_2 )  \, \right \} \\
		 = \sum_{i=1}^{i=12} \, \left \{ \,   B( t , T_{i-1} ) - B(t , T_i)    \, \right \} 
		- \, 6,05 \%  \, \sum_{i=1}^{i=3} \, \left \{ \,   \delta_i'\, B( t , T_{4i} ) \,  \N ( D^{T_{4i}}_2 )  \, \right \} \\
		\end{eqnarray*}

%***********************************************************
\chapter{Produits structur\'es sur le p\'etrole : Exemple 3}
%***********************************************************

\section{Description}

	
	Le contrat d\'ecrit dans cette partie est un swap particulier car en plus de d\'ependre du taux variable 
consid\'er\'e (EURIBOR 6 mois), les flux d\'ependent d'un autre sous-jacent : le rendement du prix du p\'etrole 
entre le jour de la signature du contrat et l'\'ech\'eance \ ie. \,$\quad \frac{S^f_0 - S^f_T}{S^f_0} \quad $\\
	\begin{description}
		\item[$\bullet \quad$] dur\'ee du contrat : 6 mois \\
		\item[$\bullet \quad$] nominal : $N$ \officialeuro \\
		\item[$\bullet \quad$] dans 6 mois :
			\begin{itemize}
				\item la partie A paye 7,80 \% \\
				\item la partie B paye $Eur_6 + 0,02\% + CO$  si $S^f_T \leq 80\% S_0^f$  et $0$ sinon 
			\end{itemize}
	\end{description}
	\indent En notant: \\
	\begin{itemize}
		\item	$t = T_0 \quad$  la date de signature\\
		\item	$T \quad$  l'\'ech\'eance \\
		\item	$T_1 \quad$ la date de d\'ebut de composition des int\'er\^ets
		\item	$\delta \quad $ la dur\'ee de composition (en fraction d'ann\'ee) entre $T_1$ et l'\'ech\'eance \\
		\item	$S^f_0 \quad $  le cours du p\'etrole \`a la signature $T_0$  ($ S^f_0 \, =\, 55.90 \, \$ $)\\
		\item	$S^f_T \quad $  le cours du p\'etrole \`a  l'\'ech\'eance\\
		\item	$Eur_6 \quad $  le taux EURIBOR 6 mois valable pour toute la dur\'ee de composition $\delta \,$ 
			(fix\'e en $T_1$)\\
	\end{itemize}

	Le payoff est alors donn\'e par \footnote{en multipliant par le nominal exprim\'e en \ \officialeuro} :
	\begin{eqnarray*}
		\delta \left ( \, Eur_6   + 2 bps + CO  - 7.80 \% \, \right )   \\
	\end{eqnarray*}
	o\`u :
	\begin{eqnarray*}
		CO &=&  \frac{S^f_0 - S^f_T}{S^f_0} \, \1_{\{S^f_T \leq 80 \% \, S^f_0 \}}
	\end{eqnarray*}


	
\section{Formule de prix}

	Il s'agit d'\'evaluer :\\
	{
	\begin{eqnarray*}
&\E_{\Q} \,\left \lbrack \, \frac{\beta(t)}{\beta(T)}  \, \delta \left ( \, Eur_6   + 2 bps + \frac{S_0 - S_T}{S_0} \, \1_{\{S_T \leq 80 \% \, S_0 \}} - 7.80 \% \, \right ) \, / \, \F_t \, \right \rbrack \, = &\\
&\E_{\Q} \,\left \lbrack \, \frac{\beta(t)}{\beta(T)}  \, \delta \, Eur_6  \, / \, \F_t \, \right \rbrack
\, + \,  \E_{\Q} \,\left \lbrack \, \frac{\beta(t)}{\beta(T)}  \, \delta \,2 bps  \, / \, \F_t \, \right \rbrack   &\\
&\, + \,  \E_{\Q} \,\left \lbrack \, \frac{\beta(t)}{\beta(T)}  \, \delta  \frac{S_0 - S_T}{S_0} \, \1_{\{S_T \leq 80 \% \, S_0 \}}  \, / \, \F_t \, \right \rbrack 
\, - \, \E_{\Q} \,\left \lbrack \, \frac{\beta(t)}{\beta(T)}  \, \delta - 7.80 \% \, / \, \F_t \, \right \rbrack  &\\
	\end{eqnarray*}

et on a :
	\begin{eqnarray*}
		Eur_6 &=& \frac 1 \delta \, \left ( \,  \frac 1 {B( T_1 , \,T)} - 1 \, \right ) \\
		B( T_1 ,\, T) &=& \frac{\beta(T_1)}{\beta(T)} \\
	\end{eqnarray*}

Le premier terme nous donne donc :
	\begin{eqnarray*}
		\E_{\Q} \,\left \lbrack \, \frac{\beta(t)}{\beta(T)}  \, \delta \, Eur_6  \, / \, \F_t \, \right \rbrack &=& 
			\delta \, \E_{\Q} \,\left \lbrack \, \frac{\beta(t)}{\beta(T)}  \,\frac 1 \delta \, \left ( \,  \frac 1 {B( T_1 ,\,  T)} - 1 \, \right )  \, / \, \F_t \, \right \rbrack\\
		&=& \E_{\Q} \,\left \lbrack \, \frac{\beta(t)}{\beta(T)} \, \frac{\beta(T)}{\beta(T_1)}  \, / \, \F_t \, \right \rbrack \, 
		- \E_{\Q} \,\left \lbrack \, \frac{\beta(t)}{\beta(T)}	\, / \, \F_t \, \right \rbrack \\
		&=& B( t , \, T_1 ) - B( t , \, T ) \\
	\end{eqnarray*}

D'autre part , 
	\begin{eqnarray*}
		\E_{\Q} \,\left \lbrack \, \frac{\beta(t)}{\beta(T)}  \, \delta \,2 bps  \, / \, \F_t \, \right \rbrack  -  \E_{\Q} \,\left \lbrack \, \frac{\beta(t)}{\beta(T)}  \, \delta - 7.80 \% \, / \, \F_t \, \right \rbrack  
&= & 2 \, bps \, \delta \, B( t , T )  -  \delta \, 7.80 \% \, B( t , T ) \\
	\end{eqnarray*}

Il reste \`a d\'eterminer :
	\begin{eqnarray*}
	\E_{\Q} \,\left \lbrack \, \frac{\beta(t)}{\beta(T)}  \, \delta  \frac{S^f_0 - S^f_T}{S^f_0} \, \1_{\{S^f_T \leq 80 \% \, S^f_0 \}}  \, / \, \F_t \, \right \rbrack 
& = & \frac{\delta}{S^f_0} \, \E_{\Q} \,\left \lbrack \,\frac{\beta(t)}{\beta(T)} \, (S^f_0 - S^f_T)\, \1_{\{S^f_T \leq 80 \% \, S^f_0 \}}  \, / \, \F_t \, \right \rbrack \\
	\end{eqnarray*}
Et, on reconnait dans le terme :
	\begin{eqnarray*}
		\E_{\Q} \,\left \lbrack \,\frac{\beta(t)}{\beta(T)} \, (S^f_0 - S^f_T)\, \1_{\{S^f_T \leq 80 \% \, S^f_0 \}}  \, / \, \F_t \, \right \rbrack \\
	\end{eqnarray*} 
la formule d'\'evaluation risque-neutre domestique d'un put gap sur sous-jacent \'etranger dont les strikes sont :
	\begin{itemize}
		\item	$ L^f = S^f_0 \quad $ pour le strike intervenant dans le payoff \\
		\item	$ K^f = 80 \% \, \, S^f_0 \quad $ pour le strike intervenant dans la probabilit\'e d'exercice de l'option\\
	\end{itemize}
En adaptant la formule du put gap \`a notre cas et en prenant en compte la corr\'elation entre sous-jacent \'etranger et taux de 
change, on trouve que :
	\begin{eqnarray*}
		\E_{\Q} \,\left \lbrack \,\frac{\beta(t)}{\beta(T)} \, (S^f_0 - S^f_T)\, \1_{\{S^f_T \leq 80 \% \, S^f_0 \}}  \, / \, \F_t \, \right \rbrack
	&=& L^f \, e^{-\int_t^T r_s ds} \, \N(-D_2^f) - S_t \, \N(-D^f_1) \\
	\end{eqnarray*}
avec :
	\begin{eqnarray*}
		D_1^f &=& \frac{ \ln{\{ \frac{S_t^f}{K^f} \}} + (R^f(t,T)- \Gamma_X^{S^f}(t,T) + \frac 1 2 \Sigma^2_{S^f}(t,T) ) (T-t)}
					{\Sigma_{S^f}(t,T) \sqrt{T-t}}\\
		D_2^f &=& \frac{ \ln{\{ \frac{S_t^f}{K^f} \}} + (R^f(t,T)- \Gamma_X^{S^f}(t,T) - \frac 1 2 \Sigma^2_{S^f}(t,T) ) (T-t)}
					{\Sigma_{S^f}(t,T) \sqrt{T-t}}\\
	\end{eqnarray*}

\subsubsection*{prix du contrat}
	Le prix du contrat en $t$ est alors donn\'e par la formule suivante \footnote{en multipliant par le nominal} :
		\begin{eqnarray*}
	B( t , T_1 ) \, -\,  B( t , T ) \, + \,  2 \, bps \, \delta \, B( t , T ) \, - \,  \delta \, 7.80 \% \, B( t , T ) \, + \, \frac{\delta}{S^f_0} \, \left \{ \,  S_0^f \, e^{-\int_t^T r_s ds} \, \N(-D_2^f) - S_t \, \N(-D^f_1)   \, \right \}\\
= \quad B( t , T_1 ) \, + \,  B( t , T ) \, \left \{\, - 1 \,  + \, \delta \, \left ( \,  - 7.78 \% \, + \N(-D_2^f) \, \right ) \, \right \} \, + \, \frac{\delta \, S_t^f}{S^f_0} \, \N(-D^f_1)
		\end{eqnarray*}

avec :
	\begin{eqnarray*}
		D_1^f &=& \frac{ \ln{\{ \frac{100 \, S_t^f}{80 \, S_0^f} \}} + (R^f(t,T)- \Gamma_X^{S^f}(t,T) + \frac 1 2 \Sigma^2_{S^f}(t,T) ) (T-t)}
					{\Sigma_{S^f}(t,T) \sqrt{T-t}}\\
		D_2^f &=& \frac{ \ln{\{ \frac{100 \, S_t^f}{80 \, S_0^f} \}} + (R^f(t,T)- \Gamma_X^{S^f}(t,T) - \frac 1 2 \Sigma^2_{S^f}(t,T) ) (T-t)}
					{\Sigma_{S^f}(t,T) \sqrt{T-t}}\\
	\end{eqnarray*}

	

\appendix

%**************
\part*{Annexes \addcontentsline{toc}{part}{Annexes}}
%**************

%***********************
\chapter{Probabilit\'es}
%***********************

\section{Esp\'erance conditionnelle}
		
	\begin{prop}[Propri\'et\'e de l'Esp\'erance Conditionnelle] $\quad$ \\
		Soit $\mathcal G$ une tribu.\\
		Soit $Z$ une variable al\'eatoire $\mathcal G$-mesurable et $\P$-int\'egrable. \\
		Soit $\tilde Z$ une variable al\'eatoire $\P$-int\'egrable. \\
		\\
		On alors : \\
		\begin{itemize}
			\item [$\bullet$] $\E[ \, Z \, ] = \E \Big [\, \E [ Z / \, \mathcal G ] \, \Big ]$
			\item [$\bullet$] $\E[ \, Z \tilde Z \, / \, \mathcal G ] = Z \E [\tilde Z / \, \mathcal G ]$
		\end{itemize}
	\end{prop}

\section{Changement de probabilit\'e}

	\begin{df}[D\'eriv\'ee de Radon-Nikodym] $\quad$ 
		Une probabilit\'e $\Q$ sur un espace probabilis\'e $(\Omega, \, \mathcal A, \, \P)$ d\'efinit 
un changement de probabilit\'e s'il existe une variable al\'eatoire $\frac{\dif \Q}{\dif \P}$, 
d\'eriv\'ee de Radon-Nikodym, telle que pour tout \'ev\'enement $A \: \in \mathcal A$
		\begin{displaymath}
			\Q (A) = \E^{\P}[\frac{\dif \Q}{\dif \P} \text{\normalfont{\1}} _A ]	
		\end{displaymath}
		De plus si Z est une variable al\'eatoire $\Q$-int\'egrable on a
		\begin{displaymath}
			\E^{\Q} [Z] = \E^{\P} [\frac{\dif \Q}{\dif \P} Z]  
		\end{displaymath} 
	\end{df}
	\begin{prop}[R\`egle de Bayes]$\quad$ \label{RegleBayes} 
		On consid\`ere $\P \sim \Q$ (ie. $\P$ et $\Q$ \'equivalentes) sur 
l'espace probabilisable $(\Omega, \, \mathcal A)$. \\
		\indent Soit $\mathcal G$ une sous-tribu de $\mathcal A$ \\
		\indent Soit $\frac{\dif \Q}{\dif \P}|_{\mathcal G}$ la d\'eriv\'ee de de Radon-Nikodym de
$\Q$ par rapport \`a $\P$ sur la
 sous-tribu $\mathcal G$. \\
		\\
		On alors : 
		\begin{displaymath}
			\E^{\Q}[Z / \mathcal G] = \frac{\E^{\P}[\frac{\dif \Q}{\dif \P}|_{\mathcal G}\, Z / \mathcal G]}{\E^{\P}[\frac{\dif \Q}{\dif \P}|_{\mathcal G}/ \mathcal G]}
		\end{displaymath}
 
	\end{prop} 
	
	\begin{df}[Filtration] $\quad$
		Soit $(\Omega, \, \mathcal A, \, \P)$ un espace probabilis\'e. Une filtration sur cet
 espace est une famille croissante $(\F)_{0 \leq t \leq + \infty} $ de sous tribus de $\mathcal A$. 
	\end{df}
	\begin{prop}$\quad$
		Supposons que $\Q \ll \P$ (ie. $\Q$ est absolument continue par rapport \`a $\P$) sur 
$\F_{+\infty}$. Pour tout t $\in \: [0, \, + \infty]$, 
		\begin{displaymath}
			L_t := \frac{\dif \Q}{\dif \P}|_{\F_t}
		\end{displaymath}
		la d\'eriv\'ee de Radon-Nikodym $\Q$ par rapport \`a $\P$ sur $\F_t$ est un $(\F)$-martingale.
	\end{prop}
	\begin{prop} $\quad$
		Si on suppose de plus que $\Q \sim \P$ sur $\F_{+\infty}$ alors on a 
			\begin{displaymath}
				\forall t \geq 0 \,  \quad \,  L_t > 0 \quad p.s.
			\end{displaymath}
	\end{prop}
	\begin{prop}[Lemme d'ind\'ependance]$\quad$ \label{independance}
	Si $X$ est $\F$ -mesurable et $Y$ est ind\'ependante de $\F$ , alors :
		\begin{eqnarray*}
			\E\, \left \lbrack h(\,X ,  \, Y \, )\, /\, \F \, \right \rbrack &=& \gamma(X) \\
		\text{o\`u } \quad \gamma(X) &=& \E\, \left \lbrack h(\,X ,  \, Y \, )\,\right \rbrack
		\end{eqnarray*}		
	\end{prop}

	
%****************************
\chapter{Calcul stochastique} \label{CalculSto}
%****************************

\section{Mouvement Brownien}	

	\noindent On notera $W^{\P}$ un $\P$-mouvement brownien issu de 0 (i.e. $W^{\P}_0 = 0$) d\'efini sur l'espace probabilis\'e
 $(\Omega, \, \F, \, \P)$.\\
	\\
	On notera $\F:=(\F_t)_{\{0 \leq t \leq 
T\}}$ sa filtration canonique. \\
	\begin{prop}[Crochet oblique d'un mouvement brownien] $\quad$ 
		Le crochet oblique du mouvement brownien $W^{\P}$ est 
		\begin{displaymath}
			\langle W^{\P} \rangle _t \, = \, t
		\end{displaymath}
	\end{prop}
	\bigskip 
	On notera $W^{\P}_1$ et $W^{\P}_2$, deux browniens corr\'el\'es issus de 0.  	
	\begin{prop}[Crochet oblique de deux mouvement browniens] $\quad$
		Le crochet de deux mouvements browniens $W^{\P}_1$ et $W^{\P}_2$ est 
		\begin{displaymath}
			\langle W^{\P}_1, \, W^{\P}_2 \rangle _t = \rho \, t
		\end{displaymath}
	\end{prop}
	o\`u $\rho$ repr\'esente la corr\'elation entre les deux mouvements browniens. 

\section{Int\'egrale stochastique}

	\begin{df}[R\'egularit\'e] $\quad$ 
		On considerera qu'un processus $\{\theta(t) , \: t \geq 0 \}$ est r\'egulier 
par rapport au mouvement brownien $W^{\P}$ si l'int\'egrale stochastique d\'efinie par 
		\begin{displaymath}
			\int_0^t \theta_s \, \dif W^{\P}_s
		\end{displaymath}
		est bien d\'efinie. 
	\end{df}
	\bigskip
	\indent On consid\`ere deux processus $\{H_t , \: t \geq 0 \}$ et  $\{K_t , \: t \geq 0 \}$
 r\'eguliers par rapport au mouvement brownien $W^{\P}$ et on  d\'efinit deux int\'egrales stochastiques 
$M_t$ et $N_t$ 	
	\begin{displaymath}
		\left\{
		\begin{array}{l}
			M_t := \int_0^t H_s \, \dif W^{\P}_s \\
			     \\
			N_t := \int_0^t K_s \, \dif W^{\P}_s \\ 
			     \\
		\end{array}
		\right.	
	\end{displaymath}
	\begin{prop}[Crochet oblique de deux int\'egrales stochastiques] $\quad$
		Le crochet oblique des int\'egrales stochastiques $M_t$ et $N_t$ est 
		\begin{displaymath}
			\left < M, \, N \right > _t = \int_0^t H_s K_s \dif s
		\end{displaymath}    
	\end{prop}
	\begin{prop}[Crochet oblique d'une int\'egrale stochastique] $\quad$
		Le crochet oblique de l'int\'egrale stochastique $M_t$ est 
		\begin{displaymath}
			\langle M, \, M \rangle _t = \int_0^t H_s^2 \dif s
		\end{displaymath}    
	\end{prop} 
	\begin{rem} $\quad$ 
		 On notera dans ce cas $\langle M \rangle _t := \langle M, \, M \rangle _t$
	\end{rem}
	\bigskip
	\indent On consid\`ere deux processus $\{\tilde H_t , \: t \geq 0 \}$ et  
$\{\tilde K_t , \: t \geq 0 \}$ r\'eguliers respectivement par rapport aux mouvements browniens $W^{\P}_1$ et 
$W^{\P}_2$ et on  d\'efinit deux int\'egrales stochastiques $\tilde M_t$ et $\tilde N_t$  	
	\begin{displaymath}
		\left\{
		\begin{array}{l}
			\tilde M_t := \int_0^t \tilde H_s \, \dif W^{\P}_{1, s} \\
			     \\
			\tilde N_t := \int_0^t \tilde K_s \, \dif W^{\P}_{2, s} \\ 
			     \\
		\end{array}
		\right.	
	\end{displaymath}
	\begin{prop}[Crochet oblique de deux int\'egrales stochastiques avec browniens corr\'el\'es] $\quad$
		Le crochet oblique des int\'egrales stochastiques $\tilde M_t$ et $\tilde N_t$ est 
		\begin{displaymath}
			\langle \tilde M, \,  \tilde N \rangle_t = \int_0^t \tilde H_s \tilde K_s \, \rho \, \dif s
		\end{displaymath}    
	\end{prop}				


\section{Processus d'It\^o}

	\begin{df}[R\'egularit\'e] $\quad$
		On considerera qu'un processus $\{\theta(t) , \: t \geq 0 \}$ est r\'egulier par 
rapport \`a la mesure de Lebesgue si l'int\'egrale de Lebesgue d\'efinie par 
		\begin{displaymath}
			\int_0^t \theta_s \, \dif s
		\end{displaymath}
		est bien d\'efinie.
	\end{df}
	\begin{df}[Processus d'It\^ o] $\quad$
		Un processus $X$ est un processus d'It\^o si $X_t$ a la d\'ecomposition suivante
		\begin{equation}\label{ProcessusIto}
			\forall t \geq 0 \qquad X_t := x + \int_0^t \,b_s \, \dif s + \int_0^t \sigma _s \dif W^{\P}_s  
		\end{equation}		
	\end{df}
	\begin{rem}[Equation Diff\'erentielle Stochastique] $\quad$
		L'\'equation (\ref{ProcessusIto}) ci-dessus peut s'\'ecrire \'egalement : 
		\begin{displaymath}
			\left\{
			\begin{array}{l}
				\dif X_t = b_t \dif t + \sigma_t \dif W^{\P}_t \\
				X_0 = x 
			\end{array}
			\right.
		\end{displaymath}
	\end{rem}
	\begin{prop}[Crochet oblique d'un processus d'It\^o] $\quad$
		Le crochet oblique d'un processus d'It\^o est 
		\begin{displaymath}
			\langle X \rangle _t = \int_0^t \sigma_s^2 \dif s
		\end{displaymath}		
	\end{prop}
	
\section{Formule d'It\^o multidimensionnelle}

	\begin{prop}[Lemme d'It\^o] $\quad$ \label{LemmeIto}
		Soient $X_{t}^{1}, \, X_{t}^{2}, \cdots, \, X_{t}^{p}$, des processus d'It\^o. \\
		\\
		\indent Posons $X_{t}:=(X_{t}^{1},\, X_{t}^{2}, \,\ldots, \, X_{t}^{p})$. \\
		\\
		\indent Soit $f$ une fonction de $\R^p$ dans $\R$.\\
		\\
		\indent Alors $f(X_t)$ est un processus de It\^o, et on a :
		\begin{eqnarray*}
			f(X_t) &=& f(X_0) + \sum_{i=1}^{p} {\int_0^t \frac{\partial f}{\partial x_i}(X_s) \, \dif X_s^j} +
					\frac 1 2 \sum_{i,j=1}^p {\int_0^t \frac{\partial_2 f}{\partial x_i 
					\partial x_j}(X_s) \, \dif \langle X^i, \, X^j \rangle _s}
		\end{eqnarray*}
	\end{prop}

\section{Th\'eor\`eme de Girsanov}

	\noindent Soit un processus $\{\theta_t , \: t \geq 0 \}$ r\'egulier par rapport \`a la mesure
 de Lebesgue et par rapport au mouvement brownien $W^{P}$ \\
	\begin{df}[Exponentielle de Dol\'eans-Dade] $\quad$
		On d\'efinit l'exponentielle de Dol\'eans-Dade par 
		\begin{displaymath}
			\mathcal E_t (\theta \bullet W^{\P}  ) := \exp \big \{ \int_0^t \theta_s \, \dif s - \frac 1 2 \int_0^t \theta_s \, \dif W^{\P}_s \big \}      
		\end{displaymath}
	\end{df}
	\begin{prop}[Th\'eor\`eme de Girsanov] $\quad$  \label{TheoremGirsanov}
		On consid\`ere le processus $\{L_t , \: t \geq 0 \}$ d\'efini par  
		\begin{displaymath}
			L_t := \mathcal E_t (\theta \bullet W^{\P}) 
		\end{displaymath}
		Sous des conditions suppl\'ementaires de r\'egularit\'e sur le processus $\{\theta_t , \: t \geq 0 \}$
 on a alors \\
		\begin{itemize}
			\item [$\bullet$] $\{L_t , \: t \geq 0 \}$ est une $(\P, \, \F_t)$-martingale. \\
			\item [$\bullet$] Il existe un probabilit\'e $\Q$ telle que sa d\'eriv\'ee de
 Radon-Nikodym par rapport \`a $\P$ sur la filtration $\F_T$ est 
				\begin{displaymath}
					\frac {\dif \Q} {\dif \P} = L_T
				\end{displaymath}  		
			\item [$\bullet$] Il existe un $\Q$-mouvement brownien $\tilde W^{\Q}_t := W^{\P}_t + \int_0^t \theta _s 
\dif s$.   
		\end{itemize}
	\end{prop}
	
\section{Equation Diff\'erentielle Stochastique et Mouvement \mbox{brownien} g\'eom\'etrique}

	\begin{df}[Mouvement brownien g\'eom\'etrique]$\quad$
		On dit qu'un proccessus $\left \{ X_t, t \leq T \right\}$ est un mouvement brownien g\'eom\'etrique 
s'il satisfait \`a une \'equation diff\'erentielle stochastique (sous une certaine probabilit\'e) du type :
		\begin{eqnarray*}\
			\frac{\dif X_t}{ X_t} &=& \mu_t \, \dif t + \sigma_t \, \dif W_t 
		\end{eqnarray*}  
	dont la solution est donn\'ee par\footnote{En appliquant le Lemme d'It\^o cf. proposition \ref{LemmeIto} p. \pageref{LemmeIto}} :
		\begin{eqnarray*}
			X_t &=& X_0 \, \exp{ \left \{ \int_0^t \left ( \mu_s - \frac 1 2 \sigma^2_s \right ) \dif s  +  
			\int_0^t \sigma_s \dif W_s \right \}}
		\end{eqnarray*}
	\end{df}
	\begin{rem}$\quad$
		\begin{itemize}
			\item[$\bullet \quad$] En mod\'elisation des march\'es financiers, on suppose souvent que 
				les instruments \'etudi\'es sont des proccesus r\'egis par ce type d'EDS \\
			\item[$\bullet \quad$] Le terme $\mu_t$ est appel\'e {\bf terme de tendance} ou {\bf drift}\\
			\item[$\bullet \quad$] $\sigma_t$ repr\'esente la {\bf volatitilit\'e instantan\'ee 
				du processus en t }\\ 
		\end{itemize}
	\end{rem}
	\begin{prop}$\quad$ \label{MvBrGeoMartingale}
		\begin{itemize}
			\item La volatilit\'e est invariante par changement de probabilit\'e \\
			\item $\left \{ X_t , t \leq T \right \}$ est une $\left ( \Omega, \F, \P \right )$-martingale, si 
			et seulement si son terme de tendance est nul sous $\P$  
		\end{itemize}
	\end{prop}	

%**********************************************
\chapter{Simulation de processus stochastiques} \label{Simulation}
%**********************************************

\section{Simulation d'une loi uniforme sur $\left [ 0 , 1 \right ]$}
	
	On simule une suite de variables al\'eatoires uniformes sur  $\left [ 0 , 1 \right ]$ par l'algorithme suivant :\\
		\begin{displaymath}
			\left \{
			\begin{array}{rcl}
				x_0 &=& \text{valeur initiale} \\
				x_{n+1} &=& a x_n + b ( \text{mod m} ) 
			\end{array}
			\right.
		\end{displaymath}
	La valeur initiale $x_0$ doit \^etre un entier compris entre $0$ et $m-1$, on peut utiliser le g\'en\'erateur de 
nombre al\'eatoire ( rand() en C++). Il faut choisir $a$, $b$ et $m$ de mani\`ere \`a optimiser la r\'epartition des valeurs. \\
	Typiquement, pour une machine \`a 32 bits : \\
		\begin{displaymath}
			\left \{
			\begin{array}{rcl}
				m &=& 2^{32}\\
				a &=& 1103515245 \\
				b &=& 12345
			\end{array}
			\right.
		\end{displaymath}
	La suite de nombres $\left \{ u_n = \frac{x_n}{m} , \, n \geq 1 \right \} $ obtenue simule une loi uniforme sur 
$\left [ 0 , 1 \right ]$ .

\section{Simulation  de variables gaussiennes : M\'ethode de \mbox{Box \& Muller}} \label{BoxMuller}

	Cette m\'ethode est bas\'ee sur le fait que si $U_1$ et $U_2$ sont deux variables uniformes ind\'ependantes alors :
		\begin{displaymath}
			\sqrt{- 2 log(U_1)} \, cos(2 \pi U_2)
		\end{displaymath}
	suit une loi gaussienne centr\'ee et r\'eduite. \\
	\newline
\indent
	Pour simuler une gaussienne $X$ de moyenne $\mu$ et de variance $\sigma$, il suffit de poser :
		\begin{displaymath}
			X \, = \, \mu + \sigma Y
		\end{displaymath}
	avec $Y$ une gausienne centr\'ee et r\'eduite

\section{Simulation du mouvement brownien}
	
	
	Pour simuler un mouvement brownien $( W_t)_{t \geq 0}$ il s'agit de remarquer que si $( y_n )_{n \geq 0}$ est 
une suite de gaussiennes ind\'ependantes, centr\'ees et r\'eduite, si $\Delta_t > 0 $ et si on construit $( S_n )_{n \geq 1}$ 
comme suit :
		\begin{displaymath}
			\left \{
			\begin{array}{rcl}
				S_0 &=& 0 \\
				S_{n+1} &=& y_n + S_n \\
			\end{array}
			\right.
		\end{displaymath} 
 alors les lois de $( \sqrt{\Delta_t} S_0, \sqrt{\Delta_t} S_1, \cdots , \sqrt{\Delta} S_n )$ et de 
$( W_0, W_{\Delta_t}, W_{2 \Delta_t}, \cdots , W_{n \Delta_t})$ sont identiques.\\
On peut alors approcher le brownien par $X_t^n \, = \,  \sqrt{\Delta_t} S_{[t/\Delta_t]}$

\section{Simulation des \'equations diff\'erentielles stochastiques}
	
	On cherche \`a simuler l'\'equation diff\'erentielle stochastique :
		\begin{displaymath}
			\left \{
			\begin{array}{rcl}
				X_0 &=& x \\
				\dif X_t &=& b(X_t) \dif t + \sigma(X_t) \dif W_t 
			\end{array}
			\right.
		\end{displaymath}
	On se fixe un pas de discr\'etisation en temps $\Delta_t$ et on consid\`ere une suite $( y_n )_{n \geq 0}$ de 
gaussiennes centr\'ees r\'eduites  ind\'ependantes . \\
	On construit le processus $( S_n )_{n \geq 0}$ en posant :
		\begin{displaymath}
			\left \{
			\begin{array}{rcl}
				S_0 &=& x \\
				S_{n+1} &=& S_n + \Delta_t b(S_n) + \sigma(S_n) y_n \sqrt{\Delta_t}
			\end{array}
			\right.
		\end{displaymath}
	Alors le processus $( X_t^n \, = \, S_{[t/\Delta_t]})_{t \geq 0}$ approxime $( X_t )_{t \geq 0}$, la solution de 
l'\'equation diff\'erentielle consid\'er\'ee.

%********************************************************
\chapter{March\'e en absence d'opportunit\'e d'arbitrage} \label{AOA}
%*******************************************************

\section{AOA}

	Une hypoth\`ese fondamentale en mod\'elisation des march\'es  financiers est : {\bf l'absence d'opportunit\'e 
d'arbitrage}\\
\indent Il s'agit de consid\'erer un march\'e dans lequel il n'est pas possible de gagner de l'argent \`a coup s\^ur 
partant d'un investissement nul.\\
Une cons\'equence directe de l'AOA est l'unicit\'e des prix des produits d\'eriv\'es,  dans le sens o\`u deux strat\'egies 
qui donnent le m\^eme flux \`a l'horizon dans tous les \'etats du  monde ont la m\^eme valeur \`a toute date interm\'ediare.

\section{Cons\'equences de l'AOA}

On rappelle les d\'efinitions du z\'ero-coupon et du prix forward d'un titre : \\

	\begin{df}[Z\'ero-coupon]$\quad$ 
		Un z\'ero-coupon de maturit\'e $T$ not\'e $B(t,T)$, est la quantit\'e d'argent \`a investir en t pour 
disposer d'exactement  1 \officialeuro \  \`a la date $T$, sans flux interm\'ediaire.
	\end{df}

	\begin{df}[Forward] $\quad$ 
		Un contrat \`a terme (forward) est un accord entre deux parties d'acheter ou de ventre un sous-jacent $S$ 
\`a la date {\bf T}, \`a un prix {\bf K} fix\'e \`a l'avance ; tous les paiement ayant lieu en T.\\
Le strike K est le prix forward (ou \`a terme) de S en T, on le note en g\'en\'eral $F_t(S,T)$ :\\
c'est le prix auquel on est pr\^et \`a acheter le sous-jacent S en T.
	\end{df}

Les deux r\'esultats suivants sont des cons\'equences directes de l'AOA :
\newpage
	\begin{rem}[Prix forward] $\quad$ 
		Pour garantir la d\'etention de S en T on peut :\\
		\begin{itemize}
			\item	soit acheter S aujourd'hui (date t)  et le garder jusqu'en T \\
			\item	soit acheter le contrat forward. Pour pouvoir le payer en $T$, 
			il faut investir $F_t(S,T)$ en z\'ero-coupon.\\
		\end{itemize}
	Les flux en T \'etant \'egaux, l'AOA nous dit que les prix en t sont \'egaux, donc :\\
		\begin{eqnarray*}
			F_t(S,T) \, B(t,T) &=& S_t\\
		\end{eqnarray*}
		\begin{eqnarray}\label{prix forward}
			\boxed{ F_t(S,T) = \frac{S_t}{B(t,T)} }
		\end{eqnarray}
		\end{rem}
		
		\begin{rem}[Parit\'e Call-Put] $\quad$ \label{Parite}
		L'achat d'un call et la vente d'un put de m\^emes caract\'eristiques assurent la d\'etention de la valeur de 
l'actif \`a l'\'ech\'eance et la vente du prix d'exercice :
			\begin{eqnarray*}
				S_T - K
			\end{eqnarray*}
Le m\^eme flux peut alors \^etre obtenu en achetant le sous-jacent aujourd'hui et en remboursant $K B(t,T)$.
Par AOA, on a :\\
			\begin{eqnarray*}\label{call-put}
				\boxed{ Call_t(T,K,S) - Put_t(T,K,S) = S_t - K B(t,T) }
			\end{eqnarray*}
		\end{rem}

%**************************************************************
\chapter{Num\'eraire et probabilit\'e risque-neutre associ\'ee} \label{Numeraire}
%**************************************************************

\section{D\'efinitions}
		
	\begin{df}[Num\'eraire]	$\quad$
	Tout actif dont le processus de prix est strictement positif peut \^etre consid\'er\'e comme un num\'eraire.
Pour un tel choix, les autres actifs sont alors d\'etermin\'es en quantit\'e d'unit\'e de ce num\'eraire.
	\end{df}	

	\begin{ex}[Num\'eraire du facteur d'accumulation] $\quad$
	Soit $\{S_t, \, t\}$, un processus de prix associ\'e \`a un actif $S$. Consid\'erons le facteur d'accumulation 
$\{\beta(t), \, t\}$ comme num\'eraire. Le prix en $t$ du sous-jacent $S$ associ\'e \`a ce num\'eraire est : $\frac{S_t}{\beta(t)}$. 
De mani\`ere analogue, le prix d'un z\'ero-coupon de maturit\'e $T$ sous ce num\'eraire est : $\frac{B(t,T)}{\beta(t)}$
	\end{ex}

	\begin{ex}[Num\'eraire du z\'ero-coupon] $\quad$
	Le prix en $t$ du sous-jacent $S$ associ\'e au num\'eraire du z\'ero-coupon est : $ F_t(S,T)$
	\end{ex}
	
	\begin{df}[Probabilit\'e risque-neutre associ\'ee \`a un num\'eraire]$\quad$
	On dit que $\P _N$ est la mesure risque-neutre associ\'ee au num\'eraire $N( . )$ si pour tout processus de prix $\{S_t,\, t\}$, 
le processus $\{\frac{S(t)}{N(t)},\,  t\}$ est une $\{\Omega ,\, \F, \, \P_N\}$ -martingale.  
	\end{df}

	\begin{rem}$\quad$
	Par abus de langage, lorsque le num\'eraire n'est pas pr\'ecis\'e, il s'agit de la probabilt\'e risque-neutre 
associ\'ee au num\'eraire facteur d'accumulation $\beta(t)$. Cette probabilit\'e est en g\'en\'erale not\'ee $\Q$
	\end{rem}
	
	\begin{prop}$\quad$
	Soit $N$ un num\'eraire, alors la mesure de probabilt\'e $\P_N$ d\'efinie par:
		\begin{eqnarray*}
			\P_N(A) : = \frac{1}{N(0)}\, \int_A{\frac{N(T)}{\beta(T)} \, \dif\Q} \quad ,\forall A \in \F_T
		\end{eqnarray*}
est risque-neutre pour $N$
	\end{prop}
	
	\begin{rem} $\quad$
	Les mesures $\P_N$ et $\Q$ sont \'equivalentes, i.e. : 
		\begin{eqnarray*}
			\Q \, ( A ) = 0 & \Leftrightarrow & \P_N \, ( A ) = 0	\\	
		\end{eqnarray*}
Et, on a :
		\begin{eqnarray*}
			\Q \, (A) = N(0)\, \int_A{\frac{\beta(T)}{N(T)}\, \dif \P_N} \quad  , \, \forall A \in \F_T	
		\end{eqnarray*}
	\end{rem}
 
	\begin{rem} $\quad$
	Soit $\{S_t, \, t\}$, un processus de prix associ\'e \`a un actif $S$.\\
\indent - Sous $\Q$,  $\{\frac{S_t}{\beta(t)},\,  t\}$ est une $\{\Omega ,\, \F,\, \Q\}$ -martingale\\
\indent - Sous $\P_N$,  $\{\frac{S_t}{\beta(t)},\,  t\}$ est une $\{\Omega ,\F,\, \P_N\}$ -martingale\\
	\end{rem}
		
	\begin{ex}[Probabilit\'e forward-neutre]$\quad$
	On consid\`ere le num\'eraire du z\'ero-coupon de maturit\'e T \\
La probabilit\'e risque neutre associ\'ee \`a ce num\'eraire est :
		\begin{eqnarray*}
			\P_T\, (A) &=& \frac{1}{B(0,T)} \int_A{\frac{B(T,T)}{\beta(T)} \dif \Q}\\
							&=& \frac{1}{B(0,T)} \int_A{\frac{1}{\beta(T)}\dif \Q} \quad ,\forall A \in \F_T	 
		\end{eqnarray*}
$\P_T$ est appel\'ee {\bf probabilit\'e T-forward}.\\
\\
Ainsi $F_t(S,T) = \frac{S_t}{B(t,T)}$ est une $\{\Omega ,\,\F, \,\P_T\}$ -martingale
	\end{ex}

		\begin{ex}[Num\'eraire associ\'e \`a un actif] $\quad$
		Soit $S$ un actif. On consid\`ere $S(.)$ comme num\'eraire. Alors, le prix de l'actif sous ce num\'eraire est 
naturellement 1. La probabilit\'e risque-neutre associ\'e \`a ce num\'eraire est :
			\begin{eqnarray*}
				\P_s(A) &=& \frac{1}{S(0)} \int_A{\frac{S(T)}{\beta(T)}\, \dif \Q} \quad ,\forall A \in \F_T	 
			\end{eqnarray*}
		\end{ex}


\section{Changement de num\'eraire}

	Soient $N$ un num\'eraire , S, le prix d'un actif et $\P_N$ probabilit\'e risque-neutre associ\'ee \`a $N$. \\
$\P_N$ admet une densit\'e de Radon-Nikodym par rapport \`a $\Q$ , et on a :
		\begin{eqnarray*}
			\frac{\dif \P_N}{\dif \Q} = \frac{N(T)\,\beta(t)}{N(t)\,\beta(T)}
		\end{eqnarray*}
	\\
	\\
	Soit $M$, un autre num\'eraire, on a la formule de changement de num\'eraire pour tout flux $\phi(S_T)\, \, \,\\  
\F_t \,$ -mesurable :\\
		\begin{eqnarray}\label{chgt num} 
			 N(T)\,\E_{\P_N} \left \lbrack \, M(T)\, \phi(S_T) \,  / \, \F_t \, \right \rbrack &=& M(T)\,\E_{\P_M} \, \left \lbrack \, N(T)\, \phi(S_T) \, / \, \F_t \, \right \rbrack 
		\end{eqnarray}
\newpage
\subsection*{ Prix en t d'un flux $\phi(S_T)$ en $T$}

\indent Soit $\phi(S_T)$ un flux engendr\'e par un produit d\'eriv\'e en T\\
Alors la valeur du produit aujourd'hui est :\\		
	\begin{eqnarray}\label{prix RN}
\boxed{\E_{\Q}\, \left \lbrack \, \frac{\beta(t)}{\beta(T)} \,\phi(S_T)\,/\,\F_t \, \right \rbrack}
	\end{eqnarray}

\fbox{
	\begin{minipage}{\textwidth}
	\begin{rem}[Important]$\quad$
		Pour \'evaluer les prix de certains contrats, il sera parfois n\'ecessaire de changer de num\'eraire.
		Un changement de num\'eraire correspondant \`a un changement de probabilit\'e, la r\`egle de Bayes 
donn\'ee \`a la proposition \ref{RegleBayes} p. \pageref{RegleBayes} reste encore valable.
	\end{rem}
	\end{minipage}
	}

\begin{thebibliography}{20}
\addcontentsline{toc}{part}{Bibliographie}
	\bibitem{1} D. Brigo \& F. Mercurio - \textit{Interest Rate Models Theory and Practice} - Springer Finance, 2001 \\
	\bibitem{2} L. Martellini \& P. Priaulet - \textit{Produits de Taux d'Int\'er\^et} - Economica, 2004 \\
	\bibitem{3} D. Lamberton \& B. Lapeyre - \textit{Introduction au Calcul Stochastique Appliqu\'e \`a la Finance} - 
			Ellipses, 1997 \\
	\bibitem{4} P. Wilmott - \textit{Derivatives : The Theory and Practice of Financial Engineering} - University, 1998 \\
	\bibitem{5} A. Brace, M. Musiela \& E. Schl\"ogl - \textit{A Simulation Algorythm Based on Measure Relationships 
			in the Lognormal Market Models} - December 1998 \\
	\bibitem{6} S. Shreve - \textit{Stochastic Calculus and Finance} - University of Illinois, Juillet 1997 \\
	\bibitem{7} A. Ruttiens - \textit{Futures, Swaps, Options : Les produits financiers d\'eriv\'es.} Edipro, 2006\\
	\bibitem{8} JM. Dalbarade - \textit{Math\'ematiques des March\'e Financiers} - Eska, 2005\\
	\bibitem{9} F. Riva - \textit{Applications Financi\`eres sous Excel en Visual Basic} - Economica, 2005 \\
\end{thebibliography}

\end{document}